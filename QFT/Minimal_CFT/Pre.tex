\subsection{Symmetry and Conserved Current}

\subsubsection{A more standard definition of Symmetry}

In Prof. Rattazzi's lecture the symmetry of a field theory is defined as:
\defi{
  \textbf{Symmetry(QFT EPFL)}

  Consider a field theory with a series of dynamical fields $ \phi_a(x) $ and a dynamical Lagrangian $ \mathcal{L}[\phi] $. A symmetry of the action is a transformation with parameters $ \alpha $:
  \begin{align}
    &x^{\prime\mu}=f^{\mu}(x,\alpha)\\ 
    &\phi_a^{\prime}(x^{\prime})=F_a(\phi(x),\alpha)
  \end{align}
  so that satisfy that:
  \begin{align}
  d^4x\left[\mathcal{L}(\phi_a(x),\partial\phi_a(x))\right]=d^4x^{\prime}\left[\mathcal{L}(\phi_a^{\prime}(x^{\prime}),\partial^{\prime}\phi_a^{\prime}(x^{\prime}))+\partial_\mu^{\prime}K^\mu(\phi^{\prime})\right]
  \end{align}
}
\rmk{
  We haven't say anything about what \textbf{transformation} physically means. Because it has 2 understandings with exactly the same mathematical expression:
  \begin{itemize}
    \item \textbf{Active Transformation}: The coordinates are fixed, but the fields are transformed.
    \item \textbf{Passive Transformation}: The coordinates are transformed, and the field transform covariantly.
  \end{itemize}
}
This definition can be rewrite a bit more elegantly when we define the change of Action under this coordinate transformtaion:
\defi{
  \textbf{Induced Change of Field and Acton under Transformation}

  Consider a field theory with dynamical fields $ \phi_a $ and a action $ S $ that takes the form:
\begin{align}
    S=\int d^dx\ \mathcal{L}(\phi(x),\partial_\mu\phi(x))
\end{align}
Form a transformation:
  \begin{align}
    &x^{\prime\mu}=f^{\mu}(x,\omega)\\ 
    &\phi^{\prime}(x^{\prime})=F(\phi(x),\omega)
  \end{align}
  We can induce a \textbf{change of the field configuration}:
\begin{align}
  \phi(x)\to \phi^{\prime}(x)=F(\phi(f^{-1}(x)),\omega)
\end{align}
We define the action after the as:
  \begin{align}
    S^{\prime}=\int d^dx\ \mathcal{L}(\phi^{\prime}(x),\partial_\mu\phi^{\prime}(x))
  \end{align}
}
 We define Symmetry as:
\defi{
  \textbf{Symmetry(QFT standard)}

  A transformation is a symmetry of the action if the action is invariant up to a total derivative:
  \begin{align}\label{eq:Symmetry definition standard}
    S^{\prime}=S+\int d^dx\ \partial_\mu K^\mu
  \end{align}
  Pluging in the definition of action we have:
  \begin{align}
    \delta S[\phi] = \text{Boundary Term}
  \end{align}
  For we can see that $ S^\prime - S $ is the standard definition of variation of action functional induced by the transformation.Thus we can rephrase as:
  \textbf{A transformation is a symmetry if the induced variation of the action functional is a boundary term.}
}
We can Prove that these two definitions are equivalent.We consider \cref{eq:Symmetry definition standard} and do a variable substitution $ x\to x^{\prime} $, we have:
\begin{align}
  S^{\prime}=\int d^dx^{\prime} \mathcal{L}(\phi^{\prime}(x^{\prime}),\partial_\mu^{\prime}\phi^{\prime}(x^{\prime}))+\int d^dx^{\prime}\ \partial_\mu^{\prime} K^\mu = S = \int d^dx\ \mathcal{L}(\phi(x),\partial_\mu\phi(x))
\end{align}
Thus we compare the integrands and get the first definition.

\rmk{
  Why do we use the Second definition?
  Because in the path integral formalism this is more natural and make the derivation more clean. However, the first definition is more intuitive easier to understand.
}

\subsubsection{Noether's Theorem and Conserved Current}

Consider a transformation that is a \textbf{Symmetry of Rigid Parameters}, we can have a Coserved Current and the form of the current is related to the variation of action functional. 

\thm{
  \textbf{Noether's Theorem(Field Theory Standard)}

  Consider a field theory with dynamical fields $ \phi_a $ and a action $ S $ that takes the standard form. Assume that we have a transformation that is a symmetry:
  \begin{align}
    x^{\prime\mu}&=x^\mu+\omega_a\frac{\delta x^\mu}{\delta\omega_a}\\ 
    \phi^{\prime}(x^{\prime})&=\phi(x)+\omega_a\frac{\delta\mathcal{F}}{\delta\omega_a}(x)
  \end{align}
  where $ \omega_a $ are \textbf{rigid parameters}. Then there exists a conserved current $ J^\mu{}_{a} $ that satisfies, if we \textbf{lift the rigid parameters to local parameters} $ \omega_a(x) $:
  \begin{align}
    \delta S=-\int dxJ^\mu{}_a\partial_\mu\omega_a + \text{Boundary Terms}
  \end{align}
  We only have the derivative term $ \partial_{\mu} \omega_a $ this is because the transformation is a symmetry for rigid parameters. One explicit form of $ J^\mu{}_a $ is (\textbf{Canonical Form}):
  \begin{align}
    J^\mu{}_a=\left\{\frac{\partial\mathcal{L}}{\partial(\partial_\mu\phi)}\partial_\nu\phi-\delta_\nu^\mu\mathcal{L}\right\}\frac{\delta x^\nu}{\delta\omega_a}-\frac{\partial\mathcal{L}}{\partial(\partial_\mu\phi)}\frac{\delta\mathcal{F}}{\delta\omega_a}
  \end{align}
If the EoM is satisfied, we have the conservation equation:
  \begin{align}
    \partial_\mu J^\mu{}_a =0
  \end{align}
}
\textbf{Proof:}

\YL{[leave it for later. Its real standard]}


\subsubsection{Series of Conserved Currents}

  The Conserved Current is not Unique. We can always add a term of the form:
  \begin{align}
    J^\mu{}_a\to J^\mu{}_a + \partial_\nu B^{\mu\nu}{}_a \quad \text{where}\quad B^{\mu\nu}{}_a = -B^{\nu\mu}{}_a
  \end{align}
  Then it is also a conserved current and Satisfy the relation with the variation of action functional. This is because:
  \begin{align}
    \int d^dx\ \partial_\nu B^{\mu\nu}{}_a \partial_\mu \omega_a = \int d^dx\ \partial_\mu\left(B^{\mu\nu}{}_a \partial_\nu \omega_a\right) - \int d^dx\ B^{\mu\nu}{}_a \partial_\mu\partial_\nu \omega_a
  \end{align}
  The first term is a boundary term and the second term is symmetric in $ \mu\nu $ while $ B^{\mu\nu}{}_a $ is antisymmetric in $ \mu\nu $ thus it vanishes.  

\subsection{Energy-Momentum Tensor}

\subsubsection{Canonical Energy-Momentum Tensor}

Consider the spacetime translation symmetry:
\begin{align}
  &x^{\prime\mu}=x^\mu + \epsilon^\mu\\ 
  &\phi^{\prime}(x^{\prime})=\phi(x)
\end{align}
We can construct its Canonical Conserved Current which is defined as the canonical
\defi{
  \textbf{Canonical Energy-Momentum Tensor}

  The Canonical Energy-Momentum Tensor is defined as the conserved current of spacetime translation symmetry:
  \begin{align}
    T_c^\mu{}_\nu = \frac{\partial\mathcal{L}}{\partial(\partial_\mu\phi)}\partial_\nu\phi - \delta^\mu_\nu \mathcal{L}
  \end{align}
}
According to the Noether's Theorem, we can see that if a theory has translation symmetry, we lift the rigid translation parameters $ \epsilon^\mu $ to local parameters $ \epsilon^\mu(x) $, we have:
\begin{align}
  \delta S = -\int d^dx\ T_c^\mu{}_\nu \partial_\mu \epsilon^\nu + \text{Boundary Terms}
\end{align}
and if the EoM is satisfied, we have the conservation equation:
\begin{align}
  \partial_\mu T_c^\mu{}_\nu =0
\end{align}
This is generally, the canonical E-M tensor is \textbf{not symmetric} in its indices and has a \textbf{non-vanishing trace}.

\subsubsection{Symmetric Traceless Energy-Momentum Tensor}

If the theory has Lorentz Symmetry and Scale Symmetry, we can always construct a symmetric traceless energy-momentum tensor from the canonical one by adding an improvement term. 

\thm{
  \textbf{Symmetric Traceless Energy-Momentum Tensor}

  Consider a field theory with Lorentz and Scale Symmetry. We can construct a \textbf{Symmetric and Traceless} Energy-Monentum Tensor:
  \begin{align}
    T^{\mu\nu}=T_c^{\mu\nu}+\partial_\rho B^{\rho\mu\nu}+\frac{1}{2}\partial_\lambda\partial_\rho X^{\lambda\rho\mu\nu}
  \end{align}
  where $ B $ and $ X $ are constructed tensors. Moreover, the Lorentz and Scale Conserved Currents can be constructed from this symmetric traceless energy-momentum tensor as:
  \begin{align}
    J_{Lorentz}^{\mu\rho\sigma} &= T^{\mu\rho}x^\sigma - T^{\mu\sigma}x^\rho\\ 
    J_{Scale}^\mu &= T^{\mu\nu}x_\nu
  \end{align}
}
\rmk{
  The above statement is a bit rouph. 
  \begin{itemize}
    \item If the system has \textbf{Lorentz Symmetry}, we can always construct a symmetric energy-momentum tensor by adding an improvement term $ \partial_\rho B^{\rho\mu\nu} $ where $ B^{\rho\mu\nu}=-B^{\mu\rho\nu} $. This is called \textbf{Belinfante-Rosenfeld procedure}.
    \item If the system has \textbf{Scale Symmetry}, we can only consturct a traceless energy-momentum tensor when the \textbf{Virial Current} is a total derivative.
\begin{align}
  V^\mu=\partial_\alpha\sigma^{\alpha\mu}
\end{align}
      In this case we can add the second improvement term $ \frac{1}{2}\partial_\lambda\partial_\rho X^{\lambda\rho\mu\nu} $ to make the energy-momentum tensor traceless.
  \end{itemize}
}

Proof: See yellow book chap 4.2.2 and chap 2.5.1. 


\begin{itemize}
  \item \textbf{From now on we will always use the symmetric traceless energy-momentum tensor when we talk about energy-momentum tensor.}
\end{itemize}

\subsection{Euclidean Formalism of Path Integral}


\subsubsection{Correlation Functions in QFT}

In QFT, we focus on the vaccum expectation value of operators as observables. Instead of working canonically we can use Euclidean path integral formalism to compute these vaccum expectation values. 
\defi{
  \textbf{Correlation Function in Canonical Formalism}

  In canonical formalism, we define the correlation function of operators \{quantized classical observables\} $ \mathcal{O}_1(x_1),\mathcal{O}_2(x_2),\cdots,\mathcal{O}_n(x_n) $ as the vaccum expectation value:
  \begin{align}
    \langle \mathcal{O}_1(x_1)\mathcal{O}_2(x_2)\cdots\mathcal{O}_n(x_n)\rangle = \langle 0 | T\{\mathcal{O}_1(x_1)\mathcal{O}_2(x_2)\cdots\mathcal{O}_n(x_n)\} | 0 \rangle
  \end{align}
}
Note it is important to have the time-ordering operator $ T\{\cdots\} $ here, without it we cannot have a well-defined interpretation of correlation function.



\subsubsection{Euclidean Path Integral as vaccum Expectation Value}

We can obtain the correlation function in Euclidean path integral formalism as:
\thm{\label{eq:correlation function path integral}
  \textbf{Correlation Function in Euclidean Path Integral Formalism}

  In Euclidean path integral formalism, we can compute the correlation function from classical observables $ \mathcal{O}_1(x_1),\mathcal{O}_2(x_2),\cdots,\mathcal{O}_n(x_n) $ as:
  \begin{align}
    \langle \mathcal{O}_1(x_1)\mathcal{O}_2(x_2)\cdots\mathcal{O}_n(x_n)\rangle = \frac{\int D\phi\ \mathcal{O}_1(x_1)\mathcal{O}_2(x_2)\cdots\mathcal{O}_n(x_n) e^{-S_E[\phi]}}{\int D\phi\ e^{-S_E[\phi]}}
  \end{align}
  where $ S_E[\phi] $ is the Euclidean action functional and $ \mathcal{O}(x) $ on the right hand side are classical functionals of fields $ \phi $ instead of operators.
}
From now on we will always use the Euclidean path integral formalism to compute correlation functions and we will use $ S $ which should denote the Euclidean action functional.


\subsubsection{Wick Theorem}

In canonical formalism, we define the correlation function with a time ordering or operators. However, it is difficult to calculate the time ordering operators vaccum expectation values. However, we can define another ordering of operators called \textbf{Normal Ordering} 

\defi{
  \textbf{Normal Ordering}

  Consider a set of creation and annihilation operators $ a_i^\dagger,a_i $ as the canonical value of the theory.

  The normal ordering of a operator product $ \mathcal{O} $ denoted as $ :\mathcal{O}: $ is defined as the operator product with all the creation operators moved to the left of all the annihilation operators.
}
\begin{itemize}
  \item Vaccum expectation value of a normal ordered operator product is zero.
\end{itemize}
The Wick Theorm states that :
\thm{
  \textbf{Wick Theorem} 

  We first define the notation of contraction:
  \begin{align}
    :\wick{\c1\phi_1 \phi_2 \c1\phi_3 \phi_4}:
    = :\phi_1\phi_3:\,\langle\phi_2\phi_4\rangle
  \end{align}
  The relation between time ordering and normal ordering of operator products is given by:
  \begin{align}
    T\{\phi_1\phi_2\cdots\phi_n\} = :\phi_1\phi_2\cdots\phi_n: + \text{all possible contractions} 
  \end{align}
}


\subsection{Symmetry Transformation of Correlation Functions}

\subsubsection{Transformation of Correlation Functions under Symmetry}

Consider a correlation function of fields of the following form:
\begin{align}
  \langle\Phi(x_1)\cdots\Phi(x_n)\rangle=\frac{1}{Z}\int[d\Phi]\Phi(x_1)\cdots\Phi(x_n)\exp-S[\Phi]
\end{align}
We have a theorem of its transformation under a symmetry transformation:
\thm{\label{thm:correlation function symmetry}
  \textbf{Correlation Functions under Symmetry}

  If we have a symmetry transformation of classical fields $ \Phi(x)\to\Phi^{\prime}(x^{\prime})=F(\Phi(x),\alpha) $ with $ x^{\prime\mu}=f^\mu(x,\alpha) $, then the correlation function have the following identity:
  \begin{align}
    \langle\Phi(x_1^{\prime})\cdots\Phi(x_n^{\prime})\rangle=\langle F(\Phi(x_1))\cdots F(\Phi(x_1))\rangle
  \end{align}
}
If we rewrite $ x_i = f^{-1} x_i' $, then it means that:
\begin{align}
  \langle \Phi(x_1)\cdots\Phi(x_n)\rangle = \langle F(\Phi(f^{-1}(x_1)))\cdots F(\Phi(f^{-1}(x_n)))\rangle
\end{align}
In the infinitesimal form, we can define the quantum variation operator as:

\defi{
  \textbf{Quantum Variation Operator}

  Consider a symmetry transformation of classical fields $ \Phi(x)\to\Phi^{\prime}(x^{\prime})=F(\Phi(x),\omega) $ with $ x^{\prime\mu}=f^\mu(x,\omega) $. We define the generator of the transformation as $ G_a $ where:
  \begin{align}
    \Phi^\prime(x) - \Phi(x) = -i \omega_a G_a \Phi(x)
  \end{align}
  Then we define the quantum variation of the correlation function as:
  \begin{align}
  \delta_\omega\langle\Phi(x_1)\cdots\Phi(x_n)\rangle\equiv-i\omega_a\sum_{i=1}^n\langle\Phi(x_1)G_a\Phi(x_i)\cdots\Phi(x_n)\rangle
  \end{align}
}
Then we can rewrite \cref{thm:correlation function symmetry} in the infinitesimal form as:
\begin{align}\label{eq:preward}
  \delta_\omega \langle \Phi(x_1)\cdots\Phi(x_n)\rangle = 0
\end{align}
\rmk{
  Note that here we \textbf{only vary the field configuration like the classical case. We do not vary the action functional} because the transformation is a symmetry of the action, the variation of action functional is a boundary term which does not contribute to the path integral.
}

\subsubsection{Ward Identity}

The above identity \cref{eq:preward} can be rewrite in another form when we use the trick of lifting the rigid parameters to local parameters. We have:
\thm{
  \textbf{Ward Identity}\label{eq:Ward Identity}

  Consider a symmetry transformation of classical fields $ \Phi(x)\to\Phi^{\prime}(x^{\prime})=F(\Phi(x),\omega) $ with $ x^{\prime\mu}=f^\mu(x,\omega) $. If we lift the rigid parameters $ \omega_a $ to local parameters $ \omega_a(x) $, then the variation of action functional is given by:
  \begin{align}
    \delta_\omega S[\Phi] = -\int d^dx\ J^\mu{}_a \partial_\mu \omega_a(x)
  \end{align}
  where $ J^\mu{}_a $ is the conserved current associated to this symmetry. Then the Ward Identity states that:
  \begin{align}
    \partial_\mu \langle J^\mu{}_a(x) \Phi(x_1)\cdots\Phi(x_n)\rangle = i\sum_{i=1}^n \delta(x-x_i) \langle \Phi(x_1)\cdots G_a\Phi(x_i)\cdots\Phi(x_n)\rangle
  \end{align}
}
Proof: See yellow book chap 2.4.4. we use the fact that:
\begin{align}
  \langle X\rangle=\frac{1}{Z}\int[d\Phi^{\prime}]\left(X+\delta X\right)\exp-\left\{S[\Phi]+\int dx\mathrm{~}\partial_\mu j_a^\mu\omega_a(x)\right\}
\end{align}

\subsubsection{Conserved Charge as Generator of Symmetry}

From the Canonical Formalism of QFT, we know that the conserved charge is the generator of symmetry transformation. We can also see this from the Ward Identity.

To match the canonical formalism, we first have to define "time", and then we define the conserved charge as: 
\begin{align}
  Q_a = \int d^{d-1}x\ J^0{}_a(t,\vec{x})
\end{align}
Then we integrate the Ward Identity over the box between $ t-\epsilon $ and $ t+\epsilon $ and take the limit $ \epsilon\to 0^+ $, where $ t = x_1^0 $ we have:
\begin{align}
  \langle Q_a(t_+)\Phi(x_1)Y\rangle-\langle Q_a(t_-)\Phi(x_1)Y\rangle=-i\langle G_a\Phi(x_1)Y\rangle
\end{align}
This is exactly the canonical formalism statement if we use \cref{eq:correlation function path integral}:
\begin{align}
  [Q_a,\Phi]=-iG_a\Phi
\end{align}
Where we come back to the standard result of canonical formalism.

