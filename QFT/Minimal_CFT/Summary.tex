\subsection{Hilbert Space and Operator Contents}

In a General CFT, the Hilbert Space and all Operator Contents can be classified into the Primaries and Their Descendent.
\imp{Hilbert Space}{
  The Hilbert Space $ \mathcal{H} $ of a CFT is spanned by the highest weight representations of the Virasoro Algebra. Each highest weight representation is generated by a Primary State $ |h,\bar{h}\rangle $ and its Descendent States $ L_{-n_1}L_{-n_2}...|h,\bar{h}\rangle  $ ($ n_i>0 $). 
  \begin{align}
    \mathcal{H} = \bigoplus_{h,\bar{h}} \mathcal{V}_h \otimes \bar{\mathcal{V}}_{\bar{h}} ,
  \end{align}
}

And the Operators are classified similarly
\imp{Operator Content}{
The Operator Content of a CFT is classified by the Primary Operators $ \phi_{h,\bar{h}}(z,\bar{z}) $ and their Descendent Operators $ \phi_{h,\bar{h}}^{(-k_1,...,-k_n)}(z,\bar{z})  $ ($ k_i>0 $).
}


\subsection{Correlation Functions}

In fact, the full information of a CFT correlation can be extracted if we know the following data:
\begin{itemize}
  \item Central Charge $ c $ 
  \item Hilbert Space or the Operator Content
  \item Structural Constants $ C_{abc} $
\end{itemize}
Here I give a biref discussion of the general procedure:
\begin{itemize}
  \item \textbf{Step 1:} Use the Conformal Ward Identity to reduce the N-point Correlation Functions of Descendents to N-point Correlation Functions of Primaries.
  \item \textbf{Step 2:} Use the OPE of Primaries to N-point Correlation Functions of Primaries to 2-Point Correlation Functions of Primaries. 
\end{itemize}
The conformal ward identity is fully defined with the knowledge of central charge and the Hilbert space (or operator content). The OPE of Primaries is fully defined with the knowledge of the Hilbert space (or operator content) and the Structural Constants. Thus, the full information of a CFT correlation function is encoded in these three data.


\subsection{Conformal Boostrap}

The core idea is of conformal bootstrap is to ask what kinds of combination of the three data (central charge, Hilbert space/operator content, structural constants) can give a self-consistent CFT. The self-consistency conditions include:
\begin{itemize}
  \item Unitary of the Hilbert Space
  \item Crossing Symmetry of the 4-Point Correlation Functions
  \item Modular Invariance of the Torus Partition Function
\end{itemize}

