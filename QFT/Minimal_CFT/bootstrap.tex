\YL{[want to follow a math book]}

The first three sections give a minimal introduction to concepts in CFT from a normal QFT style. How, I'll stress how to organize these conceptes and think in a CFT way. btw, the name "Operator Algebra Formalism" is a word I coined in 3 seconds, so don't take it too seriously. 

The above three sections we implicitly made some assumptions apart from the assumption of conformal symmetry :
\begin{itemize}
  \item A CFT should contain Primary Fields
  \item General CFT should contain a E-M Tensor with a suitable OPE
  \item Conformal Ward identity holds beyond primary fields
  \item A global conformal invariant vacuum $ |0\rangle $ 
  \item Hilbert Space is spanned by the highest weight representations of the Virasoro Algebra
  \item ...
\end{itemize}
These assumptions are really messy. However, they made the final structure of CFT beautiful and clear. Thus, we generally won't use these explicit and implicit assumptions in the above sections to define a CFT (or we will fall into mess again), but we will make the clear results (OPE Structure, Virasoro Algebra Representation Spectrum , state operator correspondence...) as a defining feature of a CFT. These feature gives a abstract but clear definition of a CFT, which I call it the Operator Algebra Formalism. 




