% main.tex 
% !TeX root = main.tex
%%%%%%%%%%%%%%%%%%%%%%%%%%%%%% DOCUMENT 
\documentclass[12pt]{article}

%%%%%%%%%%%%%%%%%%%%%%%%%%%%%% PACKAGES

% 中文支持(XeLaTeX 编译)
% \usepackage[UTF8]{ctex}
% \usepackage{xeCJKfntef} 

% \setCJKmainfont{HanziPen SC}
% \setmainfont{HanziPen SC}


% 页面设置
\usepackage[a4paper, left=15mm, right=15mm, top=15mm, bottom=15mm]{geometry}

\PassOptionsToPackage{dvipsnames,svgnames,x11names}{xcolor}
\usepackage{xcolor}


% 数学环境及符号
\usepackage{amsmath, amssymb, amsfonts, amsthm,amsopn}
\usepackage{tensor}              % 张量指标管理
\usepackage{mathtools}           % amsmath增强
\usepackage{physics}             % 物理公式快捷命令
\usepackage{simpler-wick}
\usepackage{bbold}               % 数学黑体
\usepackage{dsfont}              % 另一种数字体
\usepackage[mathscr]{eucal}     % 花体字母
\usepackage{tensor}              % 张量指标管理
\usepackage{simpler-wick}       % Wick记号
\usepackage{mathrsfs}            % 另一种花体字母

% 颜色与图形相关
\usepackage{graphicx}           % 插图支持
\usepackage{float}              % 浮动体控制
\usepackage{tikz}               % 绘图库
\usetikzlibrary{math}           % tikz数学扩展
\usepackage{geometry}
% 表格与列表
\usepackage{makecell}           % 表格多行换行
\usepackage{multicol}           % 多栏排版
\usepackage{colortbl}           % 表格颜色
\usepackage{enumitem}           % 列表自定义

% 其他辅助
\usepackage{framed}             % 有边框环境
\usepackage{tcolorbox}          % 灵活盒子环境
\tcbuselibrary{breakable}       % 盒子内容分页
\usepackage{thmtools}           % 定理环境管理
\usepackage{thm-restate}        % 定理重述
\usepackage{showlabels}         % 显示标签,调试用(完成后可注释)
\usepackage[normalem]{ulem}     % 下划线、删除线
\usepackage{hyperref}           % 超链接(最后加载)
\usepackage{cleveref}           % 智能引用(紧跟hyperref)
\usepackage{soul}
\definecolor{lightred}{rgb}{1,.8,.8} 
\sethlcolor{lightred}   


% 自定义宏包
\usepackage{macros}

% 一个中文可以高亮的包
\usepackage{cjkhl}
\definecolor{lightblue}{rgb}{.8,.8,1}

%%%%%%%%%%%%%%%%%%%%%%%%%%%%%% 自定义命令
\newcommand{\tml}{Teichmüller space}
\newcommand{\hil}{Hilbert space}
\newcommand{\mtc}{Modular Tensor Category}

%%%%%%%%%%%%%%%%%%%%%%%%%%%%%% BEGINNING OF THE DOCUMENT

\begin{document}
\title{\boldmath A Minimal Introduction to CFT}
\author{Yu Liu}

\maketitle

\begin{abstract}
  I recently came to review some basic concepts in Conformal Field Theory (CFT) for some semester projects. When going through the renowned textbook "Conformal Field Theory" by Di Francesco, Mathieu, and Sénéchal, I found it structures poorly, without a clear logical flow and indication of the core concepts. Thus, I write this note for myself to extract the essential ideas from the book and emplasize the core concepts in CFT. This note is by no means complete or rigorous, but I hope it can serve as a concise reference for myself and others who want to get a quick overview of CFT.
\end{abstract}

\tableofcontents

\newpage
\section{Fundamental QFT}\label{sec:Fundamental QFT} % (fold)
\subsection{Symmetry and Conserved Current}

\subsubsection{A more standard definition of Symmetry}

In Prof. Rattazzi's lecture the symmetry of a field theory is defined as:
\defi{
  \textbf{Symmetry(QFT EPFL)}

  Consider a field theory with a series of dynamical fields $ \phi_a(x) $ and a dynamical Lagrangian $ \mathcal{L}[\phi] $. A symmetry of the action is a transformation with parameters $ \alpha $:
  \begin{align}
    &x^{\prime\mu}=f^{\mu}(x,\alpha)\\ 
    &\phi_a^{\prime}(x^{\prime})=F_a(\phi(x),\alpha)
  \end{align}
  so that satisfy that:
  \begin{align}
  d^4x\left[\mathcal{L}(\phi_a(x),\partial\phi_a(x))\right]=d^4x^{\prime}\left[\mathcal{L}(\phi_a^{\prime}(x^{\prime}),\partial^{\prime}\phi_a^{\prime}(x^{\prime}))+\partial_\mu^{\prime}K^\mu(\phi^{\prime})\right]
  \end{align}
}
\rmk{
  We haven't say anything about what \textbf{transformation} physically means. Because it has 2 understandings with exactly the same mathematical expression:
  \begin{itemize}
    \item \textbf{Active Transformation}: The coordinates are fixed, but the fields are transformed.
    \item \textbf{Passive Transformation}: The coordinates are transformed, and the field transform covariantly.
  \end{itemize}
}
This definition can be rewrite a bit more elegantly when we define the change of Action under this coordinate transformtaion:
\defi{
  \textbf{Induced Change of Field and Acton under Transformation}

  Consider a field theory with dynamical fields $ \phi_a $ and a action $ S $ that takes the form:
\begin{align}
    S=\int d^dx\ \mathcal{L}(\phi(x),\partial_\mu\phi(x))
\end{align}
Form a transformation:
  \begin{align}
    &x^{\prime\mu}=f^{\mu}(x,\omega)\\ 
    &\phi^{\prime}(x^{\prime})=F(\phi(x),\omega)
  \end{align}
  We can induce a \textbf{change of the field configuration}:
\begin{align}
  \phi(x)\to \phi^{\prime}(x)=F(\phi(f^{-1}(x)),\omega)
\end{align}
We define the action after the as:
  \begin{align}
    S^{\prime}=\int d^dx\ \mathcal{L}(\phi^{\prime}(x),\partial_\mu\phi^{\prime}(x))
  \end{align}
}
 We define Symmetry as:
\defi{
  \textbf{Symmetry(QFT standard)}

  A transformation is a symmetry of the action if the action is invariant up to a total derivative:
  \begin{align}\label{eq:Symmetry definition standard}
    S^{\prime}=S+\int d^dx\ \partial_\mu K^\mu
  \end{align}
  Pluging in the definition of action we have:
  \begin{align}
    \delta S[\phi] = \text{Boundary Term}
  \end{align}
  For we can see that $ S^\prime - S $ is the standard definition of variation of action functional induced by the transformation.Thus we can rephrase as:
  \textbf{A transformation is a symmetry if the induced variation of the action functional is a boundary term.}
}
We can Prove that these two definitions are equivalent.We consider \cref{eq:Symmetry definition standard} and do a variable substitution $ x\to x^{\prime} $, we have:
\begin{align}
  S^{\prime}=\int d^dx^{\prime} \mathcal{L}(\phi^{\prime}(x^{\prime}),\partial_\mu^{\prime}\phi^{\prime}(x^{\prime}))+\int d^dx^{\prime}\ \partial_\mu^{\prime} K^\mu = S = \int d^dx\ \mathcal{L}(\phi(x),\partial_\mu\phi(x))
\end{align}
Thus we compare the integrands and get the first definition.

\rmk{
  Why do we use the Second definition?
  Because in the path integral formalism this is more natural and make the derivation more clean. However, the first definition is more intuitive easier to understand.
}

\subsubsection{Noether's Theorem and Conserved Current}

Consider a transformation that is a \textbf{Symmetry of Rigid Parameters}, we can have a Coserved Current and the form of the current is related to the variation of action functional. 

\thm{
  \textbf{Noether's Theorem(Field Theory Standard)}

  Consider a field theory with dynamical fields $ \phi_a $ and a action $ S $ that takes the standard form. Assume that we have a transformation that is a symmetry:
  \begin{align}
    x^{\prime\mu}&=x^\mu+\omega_a\frac{\delta x^\mu}{\delta\omega_a}\\ 
    \phi^{\prime}(x^{\prime})&=\phi(x)+\omega_a\frac{\delta\mathcal{F}}{\delta\omega_a}(x)
  \end{align}
  where $ \omega_a $ are \textbf{rigid parameters}. Then there exists a conserved current $ J^\mu{}_{a} $ that satisfies, if we \textbf{lift the rigid parameters to local parameters} $ \omega_a(x) $:
  \begin{align}
    \delta S=-\int dxJ^\mu{}_a\partial_\mu\omega_a + \text{Boundary Terms}
  \end{align}
  We only have the derivative term $ \partial_{\mu} \omega_a $ this is because the transformation is a symmetry for rigid parameters. One explicit form of $ J^\mu{}_a $ is (\textbf{Canonical Form}):
  \begin{align}
    J^\mu{}_a=\left\{\frac{\partial\mathcal{L}}{\partial(\partial_\mu\phi)}\partial_\nu\phi-\delta_\nu^\mu\mathcal{L}\right\}\frac{\delta x^\nu}{\delta\omega_a}-\frac{\partial\mathcal{L}}{\partial(\partial_\mu\phi)}\frac{\delta\mathcal{F}}{\delta\omega_a}
  \end{align}
If the EoM is satisfied, we have the conservation equation:
  \begin{align}
    \partial_\mu J^\mu{}_a =0
  \end{align}
}
\textbf{Proof:}

\YL{[leave it for later. Its real standard]}


\subsubsection{Series of Conserved Currents}

  The Conserved Current is not Unique. We can always add a term of the form:
  \begin{align}
    J^\mu{}_a\to J^\mu{}_a + \partial_\nu B^{\mu\nu}{}_a \quad \text{where}\quad B^{\mu\nu}{}_a = -B^{\nu\mu}{}_a
  \end{align}
  Then it is also a conserved current and Satisfy the relation with the variation of action functional. This is because:
  \begin{align}
    \int d^dx\ \partial_\nu B^{\mu\nu}{}_a \partial_\mu \omega_a = \int d^dx\ \partial_\mu\left(B^{\mu\nu}{}_a \partial_\nu \omega_a\right) - \int d^dx\ B^{\mu\nu}{}_a \partial_\mu\partial_\nu \omega_a
  \end{align}
  The first term is a boundary term and the second term is symmetric in $ \mu\nu $ while $ B^{\mu\nu}{}_a $ is antisymmetric in $ \mu\nu $ thus it vanishes.  

\subsection{Energy-Momentum Tensor}

\subsubsection{Canonical Energy-Momentum Tensor}

Consider the spacetime translation symmetry:
\begin{align}
  &x^{\prime\mu}=x^\mu + \epsilon^\mu\\ 
  &\phi^{\prime}(x^{\prime})=\phi(x)
\end{align}
We can construct its Canonical Conserved Current which is defined as the canonical
\defi{
  \textbf{Canonical Energy-Momentum Tensor}

  The Canonical Energy-Momentum Tensor is defined as the conserved current of spacetime translation symmetry:
  \begin{align}
    T_c^\mu{}_\nu = \frac{\partial\mathcal{L}}{\partial(\partial_\mu\phi)}\partial_\nu\phi - \delta^\mu_\nu \mathcal{L}
  \end{align}
}
According to the Noether's Theorem, we can see that if a theory has translation symmetry, we lift the rigid translation parameters $ \epsilon^\mu $ to local parameters $ \epsilon^\mu(x) $, we have:
\begin{align}
  \delta S = -\int d^dx\ T_c^\mu{}_\nu \partial_\mu \epsilon^\nu + \text{Boundary Terms}
\end{align}
and if the EoM is satisfied, we have the conservation equation:
\begin{align}
  \partial_\mu T_c^\mu{}_\nu =0
\end{align}
This is generally, the canonical E-M tensor is \textbf{not symmetric} in its indices and has a \textbf{non-vanishing trace}.

\subsubsection{Symmetric Traceless Energy-Momentum Tensor}

If the theory has Lorentz Symmetry and Scale Symmetry, we can always construct a symmetric traceless energy-momentum tensor from the canonical one by adding an improvement term. 

\thm{
  \textbf{Symmetric Traceless Energy-Momentum Tensor}

  Consider a field theory with Lorentz and Scale Symmetry. We can construct a \textbf{Symmetric and Traceless} Energy-Monentum Tensor:
  \begin{align}
    T^{\mu\nu}=T_c^{\mu\nu}+\partial_\rho B^{\rho\mu\nu}+\frac{1}{2}\partial_\lambda\partial_\rho X^{\lambda\rho\mu\nu}
  \end{align}
  where $ B $ and $ X $ are constructed tensors. Moreover, the Lorentz and Scale Conserved Currents can be constructed from this symmetric traceless energy-momentum tensor as:
  \begin{align}
    J_{Lorentz}^{\mu\rho\sigma} &= T^{\mu\rho}x^\sigma - T^{\mu\sigma}x^\rho\\ 
    J_{Scale}^\mu &= T^{\mu\nu}x_\nu
  \end{align}
}
\rmk{
  The above statement is a bit rouph. 
  \begin{itemize}
    \item If the system has \textbf{Lorentz Symmetry}, we can always construct a symmetric energy-momentum tensor by adding an improvement term $ \partial_\rho B^{\rho\mu\nu} $ where $ B^{\rho\mu\nu}=-B^{\mu\rho\nu} $. This is called \textbf{Belinfante-Rosenfeld procedure}.
    \item If the system has \textbf{Scale Symmetry}, we can only consturct a traceless energy-momentum tensor when the \textbf{Virial Current} is a total derivative.
\begin{align}
  V^\mu=\partial_\alpha\sigma^{\alpha\mu}
\end{align}
      In this case we can add the second improvement term $ \frac{1}{2}\partial_\lambda\partial_\rho X^{\lambda\rho\mu\nu} $ to make the energy-momentum tensor traceless.
  \end{itemize}
}

Proof: See yellow book chap 4.2.2 and chap 2.5.1. 


\begin{itemize}
  \item \textbf{From now on we will always use the symmetric traceless energy-momentum tensor when we talk about energy-momentum tensor.}
\end{itemize}

\subsection{Euclidean Formalism of Path Integral}


\subsubsection{Correlation Functions in QFT}

In QFT, we focus on the vaccum expectation value of operators as observables. Instead of working canonically we can use Euclidean path integral formalism to compute these vaccum expectation values. 
\defi{
  \textbf{Correlation Function in Canonical Formalism}

  In canonical formalism, we define the correlation function of operators \{quantized classical observables\} $ \mathcal{O}_1(x_1),\mathcal{O}_2(x_2),\cdots,\mathcal{O}_n(x_n) $ as the vaccum expectation value:
  \begin{align}
    \langle \mathcal{O}_1(x_1)\mathcal{O}_2(x_2)\cdots\mathcal{O}_n(x_n)\rangle = \langle 0 | T\{\mathcal{O}_1(x_1)\mathcal{O}_2(x_2)\cdots\mathcal{O}_n(x_n)\} | 0 \rangle
  \end{align}
}
Note it is important to have the time-ordering operator $ T\{\cdots\} $ here, without it we cannot have a well-defined interpretation of correlation function.



\subsubsection{Euclidean Path Integral as vaccum Expectation Value}

We can obtain the correlation function in Euclidean path integral formalism as:
\thm{\label{eq:correlation function path integral}
  \textbf{Correlation Function in Euclidean Path Integral Formalism}

  In Euclidean path integral formalism, we can compute the correlation function from classical observables $ \mathcal{O}_1(x_1),\mathcal{O}_2(x_2),\cdots,\mathcal{O}_n(x_n) $ as:
  \begin{align}
    \langle \mathcal{O}_1(x_1)\mathcal{O}_2(x_2)\cdots\mathcal{O}_n(x_n)\rangle = \frac{\int D\phi\ \mathcal{O}_1(x_1)\mathcal{O}_2(x_2)\cdots\mathcal{O}_n(x_n) e^{-S_E[\phi]}}{\int D\phi\ e^{-S_E[\phi]}}
  \end{align}
  where $ S_E[\phi] $ is the Euclidean action functional and $ \mathcal{O}(x) $ on the right hand side are classical functionals of fields $ \phi $ instead of operators.
}
From now on we will always use the Euclidean path integral formalism to compute correlation functions and we will use $ S $ which should denote the Euclidean action functional.


\subsubsection{Wick Theorem}

In canonical formalism, we define the correlation function with a time ordering or operators. However, it is difficult to calculate the time ordering operators vaccum expectation values. However, we can define another ordering of operators called \textbf{Normal Ordering} 

\defi{
  \textbf{Normal Ordering}

  Consider a set of creation and annihilation operators $ a_i^\dagger,a_i $ as the canonical value of the theory.

  The normal ordering of a operator product $ \mathcal{O} $ denoted as $ :\mathcal{O}: $ is defined as the operator product with all the creation operators moved to the left of all the annihilation operators.
}
\begin{itemize}
  \item Vaccum expectation value of a normal ordered operator product is zero.
\end{itemize}
The Wick Theorm states that :
\thm{
  \textbf{Wick Theorem} 

  We first define the notation of contraction:
  \begin{align}
    :\wick{\c1\phi_1 \phi_2 \c1\phi_3 \phi_4}:
    = :\phi_1\phi_3:\,\langle\phi_2\phi_4\rangle
  \end{align}
  The relation between time ordering and normal ordering of operator products is given by:
  \begin{align}
    T\{\phi_1\phi_2\cdots\phi_n\} = :\phi_1\phi_2\cdots\phi_n: + \text{all possible contractions} 
  \end{align}
}


\subsection{Symmetry Transformation of Correlation Functions}

\subsubsection{Transformation of Correlation Functions under Symmetry}

Consider a correlation function of fields of the following form:
\begin{align}
  \langle\Phi(x_1)\cdots\Phi(x_n)\rangle=\frac{1}{Z}\int[d\Phi]\Phi(x_1)\cdots\Phi(x_n)\exp-S[\Phi]
\end{align}
We have a theorem of its transformation under a symmetry transformation:
\thm{\label{thm:correlation function symmetry}
  \textbf{Correlation Functions under Symmetry}

  If we have a symmetry transformation of classical fields $ \Phi(x)\to\Phi^{\prime}(x^{\prime})=F(\Phi(x),\alpha) $ with $ x^{\prime\mu}=f^\mu(x,\alpha) $, then the correlation function have the following identity:
  \begin{align}
    \langle\Phi(x_1^{\prime})\cdots\Phi(x_n^{\prime})\rangle=\langle F(\Phi(x_1))\cdots F(\Phi(x_1))\rangle
  \end{align}
}
If we rewrite $ x_i = f^{-1} x_i' $, then it means that:
\begin{align}
  \langle \Phi(x_1)\cdots\Phi(x_n)\rangle = \langle F(\Phi(f^{-1}(x_1)))\cdots F(\Phi(f^{-1}(x_n)))\rangle
\end{align}
In the infinitesimal form, we can define the quantum variation operator as:

\defi{
  \textbf{Quantum Variation Operator}

  Consider a symmetry transformation of classical fields $ \Phi(x)\to\Phi^{\prime}(x^{\prime})=F(\Phi(x),\omega) $ with $ x^{\prime\mu}=f^\mu(x,\omega) $. We define the generator of the transformation as $ G_a $ where:
  \begin{align}
    \Phi^\prime(x) - \Phi(x) = -i \omega_a G_a \Phi(x)
  \end{align}
  Then we define the quantum variation of the correlation function as:
  \begin{align}
  \delta_\omega\langle\Phi(x_1)\cdots\Phi(x_n)\rangle\equiv-i\omega_a\sum_{i=1}^n\langle\Phi(x_1)G_a\Phi(x_i)\cdots\Phi(x_n)\rangle
  \end{align}
}
Then we can rewrite \cref{thm:correlation function symmetry} in the infinitesimal form as:
\begin{align}\label{eq:preward}
  \delta_\omega \langle \Phi(x_1)\cdots\Phi(x_n)\rangle = 0
\end{align}
\rmk{
  Note that here we \textbf{only vary the field configuration like the classical case. We do not vary the action functional} because the transformation is a symmetry of the action, the variation of action functional is a boundary term which does not contribute to the path integral.
}

\subsubsection{Ward Identity}

The above identity \cref{eq:preward} can be rewrite in another form when we use the trick of lifting the rigid parameters to local parameters. We have:
\thm{
  \textbf{Ward Identity}\label{eq:Ward Identity}

  Consider a symmetry transformation of classical fields $ \Phi(x)\to\Phi^{\prime}(x^{\prime})=F(\Phi(x),\omega) $ with $ x^{\prime\mu}=f^\mu(x,\omega) $. If we lift the rigid parameters $ \omega_a $ to local parameters $ \omega_a(x) $, then the variation of action functional is given by:
  \begin{align}
    \delta_\omega S[\Phi] = -\int d^dx\ J^\mu{}_a \partial_\mu \omega_a(x)
  \end{align}
  where $ J^\mu{}_a $ is the conserved current associated to this symmetry. Then the Ward Identity states that:
  \begin{align}
    \partial_\mu \langle J^\mu{}_a(x) \Phi(x_1)\cdots\Phi(x_n)\rangle = i\sum_{i=1}^n \delta(x-x_i) \langle \Phi(x_1)\cdots G_a\Phi(x_i)\cdots\Phi(x_n)\rangle
  \end{align}
}
Proof: See yellow book chap 2.4.4. we use the fact that:
\begin{align}
  \langle X\rangle=\frac{1}{Z}\int[d\Phi^{\prime}]\left(X+\delta X\right)\exp-\left\{S[\Phi]+\int dx\mathrm{~}\partial_\mu j_a^\mu\omega_a(x)\right\}
\end{align}

\subsubsection{Conserved Charge as Generator of Symmetry}

From the Canonical Formalism of QFT, we know that the conserved charge is the generator of symmetry transformation. We can also see this from the Ward Identity.

To match the canonical formalism, we first have to define "time", and then we define the conserved charge as: 
\begin{align}
  Q_a = \int d^{d-1}x\ J^0{}_a(t,\vec{x})
\end{align}
Then we integrate the Ward Identity over the box between $ t-\epsilon $ and $ t+\epsilon $ and take the limit $ \epsilon\to 0^+ $, where $ t = x_1^0 $ we have:
\begin{align}
  \langle Q_a(t_+)\Phi(x_1)Y\rangle-\langle Q_a(t_-)\Phi(x_1)Y\rangle=-i\langle G_a\Phi(x_1)Y\rangle
\end{align}
This is exactly the canonical formalism statement if we use \cref{eq:correlation function path integral}:
\begin{align}
  [Q_a,\Phi]=-iG_a\Phi
\end{align}
Where we come back to the standard result of canonical formalism.


% section Fundamental QFT (end)

\newpage
\section{CFT Path-Integral Formalism}\label{sec:Fundamental CFT with Path Integral Approach} % (fold)
Follow chapter 5 of the yellow book.

\subsection{2D Conformal Transformation}

\subsubsection{Holomorphic function as conformal transformation}

We consider a general transformation on 2D metric:
\begin{align}
  g^{\mu\nu}\to\left(\frac{\partial w^\mu}{\partial z^\alpha}\right)\left(\frac{\partial w^\nu}{\partial z^\beta}\right)g^{\alpha\beta}
\end{align}
Then we say this transformation is conformal if:
\begin{align}
  g_{\mu\nu}\to\Lambda(z)g_{\mu\nu}
\end{align}
\thm{
  In 2D, any transformation that is holomorphic (antiholomorphic) is conformal.
}


\subsubsection{Holomorphic Coordinate}

We use the holomorphic coordinate to express everything in 2D:
\begin{align}
  & z = z^0 + i z^1, \quad \bar{z} = z^0 - i z^1, \quad z^0 = \frac{z + \bar{z}}{2}, \quad z^1 = \frac{z - \bar{z}}{2i};\\
& \partial = \frac{1}{2} (\partial_0 - i \partial_1), \quad 
\bar{\partial} = \frac{1}{2} (\partial_0 + i \partial_1), \quad \partial_0 = \partial + \bar{\partial}, \quad 
\partial_1 = i (\partial - \bar{\partial}).
\end{align}

\subsection{Primary Fields}

In the context of conformal field theory, we call every observable a field, not just the dynamical field in normal QFT.

\subsubsection{Definition of Primary Fields}
There are fields transform covariantly under all conformal transformation, which are called \textbf{Primary Fields}. They should be a generalization of fields within lorentz representation with a dimension.
\defi{
  \textbf{Primary Field (normal coordinate)}
  
  A field $ \phi(x) $ is called Primary field if under conformal transformation $ x\to x^{\prime} $, it transforms as:
  \begin{align}
    \phi^{\prime}(x^{\prime})=\Omega(x)^{-\Delta}e^{-is\Theta(x)}\phi(x).
  \end{align}
  with $ \Delta $ the scaling dimension and $ s $ the conformal spin. And:
  \begin{align}
    J^{\mu}{}_{\nu}(x)=\frac{\partial x^{\prime\mu}}{\partial x^{\nu}}.\quad \Omega(x)\equiv\sqrt{\det J(x)}\quad,
  \end{align}
  $ \Theta $ can be get by solving $ \operatorname{Tr}J=2\Omega\cos\Theta\quad \varepsilon^{\alpha\beta}J_{\alpha\beta}=2\Omega\sin\Theta $
}

If a field has scaling dimension $ \Delta $ and conformal spin $ s $, we can define its holomorphic and antiholomorphic dimensions as:
\begin{align}
  h=\frac{1}{2}(\Delta+s) \quad \bar{h}=\frac{1}{2}(\Delta-s)
\end{align}
\defi{
  \textbf{Primary Field (holomorphic coordinate)}
  
  A field $ \phi(z,\bar{z}) $ is called Primary field if under conformal transformation $ z\to w(z) $, it transforms as:
  \begin{align}
    \phi^{\prime}(w,\bar{w})=\left(\frac{dw}{dz}\right)^{-h}\left(\frac{d\bar{w}}{d\bar{z}}\right)^{-\bar{h}}\phi(z,\bar{z})
  \end{align}
}

We can write the infinitesimal version of the transformation as:
\begin{align}
  \delta_{\epsilon,\bar{\epsilon}}\phi\equiv\phi^{\prime}(z,\bar{z})-\phi(z,\bar{z})=-(h\phi\partial\epsilon+\epsilon\partial\phi)- (\bar{h}\phi\bar\partial\bar{\epsilon}+\bar{\epsilon}\bar\partial\phi)
\end{align}
If we write it in the normal coordinate $ (z^0,z^1) $, we have:
\begin{align}
  \delta\phi=\epsilon^\mu\partial_\mu\phi-\frac{\Delta}{2}\phi\partial_\mu\epsilon^\mu-\frac{is}{2}\phi\varepsilon^{\alpha\beta}\partial_\alpha\epsilon_\beta.
\end{align}
Which is exactly a field with scaling dimension $ \Delta $ and spin $ s $ under general coordinate transformation. Thus we can see that the two definitions are equivalent.


\subsubsection{Correlation Function of Primary Fields}

If the theory has conformal symmetry then we know that the correlation function of Primary Fields should satisfy the condition \cref{thm:correlation function symmetry} or the infinitesti,al version \cref{eq:preward}. These forms a constraint on the correlation functions of Primary fields. 
\thm{\label{thm:correlation function symmetry}
  \textbf{Correlation Function of Primary Fields}
  
  If a theory has conformal symmetry, then the correlation function of Primary fields should satisfy:
  \begin{align}
    \langle\phi_1(z_1,\bar{z}_1)\phi_2(z_2,\bar{z}_2)\rangle&=\frac{C_{12}}{(z_1-z_2)^{2h}(\bar{z}_1-\bar{z}_2)^{2\bar{h}}}\quad\text{if}\quad\begin{cases}h_1=h_2=h\\\bar{h}_1=\bar{h}_2=\bar{h}&\end{cases} \\ 
    \langle\phi_{1}(x_{1})\phi_{2}(x_{2})\phi_{3}(x_{3})\rangle&=C_{123}\frac{1}{z_{12}^{h_{1}+h_{2}-h_{3}}z_{23}^{h_{2}+h_{3}-h_{1}}z_{13}^{h_{3}+h_{1}-h_{2}}}\times\frac{1}{\bar{z}_{12}^{\bar{h}_{1}+\bar{h}_{2}-\bar{h}_{3}}\bar{z}_{23}^{\bar{h}_{2}+\bar{h}_{3}-\bar{h}_{1}}\bar{z}_{13}^{\bar{h}_{3}+\bar{h}_{1}-\bar{h}_{2}}}
  \end{align}
}
These are standard results. We note tht the function are fully fixed up to a factor $ C_{12} $ and $ C_{123} $. In fact these two factor are not independent, if we make $ \phi_3 $ a identity operator and put it to infinity then we can find that $ C_{123} $ reduces to $ C_{12} $.

These constant $ C_{123} $ are called the \textbf{Structure Constants} of the theory.

\subsection{Ward Identity of Primary Fields}

\subsubsection{Ward Identity in Normal Coordinate}

If a theory contains Primary field has conformal symmetry, then we can write down the Ward identity for the correlation function of Primary fields from \cref{eq:Ward Identity}, where $ X=\prod_{i=1}^n\phi_i(x_i) $ is a series of Primary fields with scaling dimension $ \Delta_i $ and conformal spin $ s_i $:
\begin{align}\label{eq:Ward Identitycft}
  \frac{\partial}{\partial x^\mu}\langle T_\nu^\mu(x)X\rangle&=-\sum_{i=1}^n\delta(x-x_i)\frac{\partial}{\partial x_i^\nu}\langle X\rangle\\
  \varepsilon_{\mu\nu}\langle T^{\mu\nu}(x)X\rangle&=-i\sum_{i=1}^n\delta(x-x_i) s_i\langle X\rangle\\
  \langle T_\mu^\mu(\mathbf{x})X\rangle&=-\sum_{i=1}^n\delta(x-x_i)\Delta_i\langle X\rangle
\end{align}
Remember that we use the symmetric traceless energy-momentum tensor here. This is a standard result which can be found in yellow book 5.2.1

\subsubsection{Ward Identity in Holomorphic Coordinate}

\YL{[here I make it more concise than the yellow book]}

We can also rewrite them in holomorphic coordinate. First we rewrite the energy-momentum tensor in holomorphic coordinate, we take the following definition:
\begin{align}
  T(z)&= -2\pi T_{zz}(z)=-\pi(T_{00}-T_{11}-2iT_{01})\\
  \bar{T}(\bar{z})&=-2\pi T_{\bar{z}\bar{z}}(\bar{z})=-\pi(T_{00}-T_{11}+2iT_{01})\\
  T_{z\bar{z}}&=-\frac{\pi}{2}T^\mu{}_\mu
\end{align}
The Traceless condition $ T^\mu{}_\mu=0 $ means $ T_{z\bar{z}}=0 $. And the conservation condition $ \partial_\mu T^{\mu\nu}=0 $ means:
\begin{align}
  \bar{\partial}T(z,\bar{z})=0,\quad \partial\bar{T}(z,\bar{z})=0
\end{align}
Thus we see that $ T(z) $ is holomorphic and $ \bar{T}(\bar{z}) $ is antiholomorphic. 

\rmk{
  These are all classical results, till now we don't know that in coorelation functions these properties still hold. But we will see that they only need little modification.
}

Then we wish to put the ward identity in holomorphic coordinate. We need some important complex analysis results:
\begin{itemize}
  \item Total derivative in holomorphic coordinate:
    \begin{align}
      \int_{M}d^{2}x\partial_{\mu}F^{\mu}=\frac{1}{2}i\int_{\partial M}\left(-dzF^{\bar{z}}+d\bar{z}F^{z}\right)
    \end{align}
    \item Detla function times a function from residue theorem:
      \begin{align}
  \frac{1}{2\pi i}\oint_{C_i}dw\frac{\epsilon(w)}{(w-w_i)^2}=\partial\epsilon(w_i),\quad\frac{1}{2\pi i}\oint_{C_i}dw\frac{\epsilon(w)}{w-w_i}=\epsilon(w_i)
      \end{align}
\end{itemize}
With these in mind we notice that the LHS of the above three ward identities can be combined into a total derivative according to the properties of conformal transoformation:
\begin{align}\label{eq:derivative trick}
  \partial_\mu(\epsilon_\nu T^{\mu\nu})=\epsilon_\nu\partial_\mu T^{\mu\nu}+\frac{1}{2}(\partial_\rho\epsilon^\rho)\eta_{\mu\nu}T^{\mu\nu}+\frac{1}{2}\varepsilon^{\alpha\beta}\partial_\alpha\epsilon_\beta\varepsilon_{\mu\nu}T^{\mu\nu}
\end{align}
\rmk{
  Here we use the fact that in 2D conformal transformation we have:
  \begin{align}
    \partial_\mu\epsilon_\nu+\partial_\nu\epsilon_\mu=\eta_{\mu\nu}\partial_\rho\epsilon^\rho
  \end{align}
}
\rmk{We notice that classically $ T^\mu{}_\mu=0 $ and $ \varepsilon_{\mu\nu}T^{\mu\nu}=0 $, thus the last two terms vanish. However, in quantum case these two terms may not vanish thus we never should ingnore them.}
Then we integrate the LHS of \cref{eq:derivative trick} over a region $ M $ the left hand side becomes:
\begin{align}
  LHS=\int_Md^2x\mathrm{~}\partial_\mu\langle T^{\mu\nu}(x)\epsilon_\nu(x)X\rangle
\end{align}
we use the above trick to rewrite it as a contour integral:
\begin{align}
  LHS = \frac{1}{2}i\int_{C}\left(-dz\langle T^{\bar{z}\bar{z}}\epsilon_{\bar{z}}X\rangle+d\bar{z}\langle T^{zz}\epsilon_{z}X\rangle\right)
\end{align}
We note that according to \cref{eq:Ward Identitycft} the $ \langle T^{\mu}{}_\mu X\rangle $ vanishes on the contour, thus we don't have the $ T_{z\bar{z}} $ term here. With a rewrite we can have:
\begin{align}
 LHS = -\frac{1}{2\pi i}\oint_{C}dz\mathrm{~}\epsilon(z)\langle T(z)X\rangle+\frac{1}{2\pi i}\oint_{C}d\bar{z}\mathrm{~}\bar{\epsilon}(\bar{z})\langle\bar{T}(\bar{z})X\rangle
\end{align}
Then we Tackle the RHS of \cref{eq:derivative trick}, we plug in the ward identities \cref{eq:Ward Identitycft} and get:
\begin{align}
  RHS=&\sum_{i=1}^n\int_Md^2x\delta(x-x_i)\left[-\epsilon^\nu(x)\frac{\partial}{\partial x_i^\nu}-\frac{\Delta_i}{2}\partial_\mu\epsilon^\mu(x)-\frac{is_i}{2}\varepsilon^{\alpha\beta}\partial_\alpha\epsilon_\beta(x)\right]\langle X\rangle\\
  =&\sum_{i=1}^n\left[-\epsilon^\nu(x_i)\frac{\partial}{\partial x_i^\nu}-\frac{\Delta_i}{2}\partial_\mu\epsilon^\mu(x_i)-\frac{is_i}{2}\varepsilon^{\alpha\beta}\partial_\alpha\epsilon_\beta(x_i)\right]\langle X\rangle  \\ 
  = & \sum_{i=1}^n\left[-\epsilon(z_i)\partial-h_i\partial\epsilon(z_i)-\bar{\epsilon}(\bar{z}_i)\bar\partial-\bar{h}_i\bar{\partial}\bar{\epsilon}(\bar{z}_i)\right]\langle X\rangle
\end{align}
If we use the second mathematical trick we mentioned above, we can rewrite it in a contour integral form:
\begin{align}
  RHS=&-\frac{1}{2\pi i}\oint_{C}dw\mathrm{~}\epsilon(w)\sum_{i=1}^n\left[\frac{1}{w-z_i}\partial_{z_i}+\frac{h_i}{(w-z_i)^2}\right]\langle X\rangle\\
  &-\frac{1}{2\pi i}\oint_{C}d\bar{w}\mathrm{~}\bar{\epsilon}(\bar{w})\sum_{i=1}^n\left[\frac{1}{\bar{w}-\bar{z}_i}\partial_{\bar{z}_i}+\frac{\bar{h}_i}{(\bar{w}-\bar{z}_i)^2}\right]\langle X\rangle
\end{align}
\begin{itemize}
  \item \textbf{Decouple of Holomorphic and Antiholomorphic Part}

    We know that the holomorphic and antiholomorphic parts are dependent, and we have $ \epsilon(z)^* = \bar\epsilon(\bar{z}) $. However, we manually decouple them and keep in mind that the theory is meaningful only when we combine them back. 
\end{itemize}

Thus we have the Conformal Ward Identity for Primary fields decompled into holomorphic and antiholomorphic parts:
\thm{
  \textbf{Conformal Ward Identity for Primary Fields}

  For a series of primary fields $ X=\prod_{i=1}^n\phi_i(z_i,\bar{z}_i) $ with holomorphic and antiholomorphic dimensions $ (h_i,\bar{h}_i) $, we have:
\begin{align}
  -\frac{1}{2\pi i}\oint_{C}dz\mathrm{~}\epsilon(z)\langle T(z)X\rangle&=-\frac{1}{2\pi i}\oint_{C}dw\mathrm{~}\epsilon(w)\sum_{i=1}^n\left[\frac{1}{w-z_i}\partial_{z_i}+\frac{h_i}{(w-z_i)^2}\right]\langle X\rangle\\
  -\frac{1}{2\pi i}\oint_{C}d\bar{z}\mathrm{~}\bar{\epsilon}(\bar{z})\langle \bar{T}(\bar{z})X\rangle&=-\frac{1}{2\pi i}\oint_{C}d\bar{w}\mathrm{~}\bar{\epsilon}(\bar{w})\sum_{i=1}^n\left[\frac{1}{\bar{w}-\bar{z}_i}\partial_{\bar{z}_i}+\frac{\bar{h}_i}{(\bar{w}-\bar{z}_i)^2}\right]\langle X\rangle
\end{align}
More conveniently, we can write them in the following local form:
\begin{align}
  \langle T(z)X\rangle&=\sum_{i=1}^n\left[\frac{h_i}{(z-z_i)^2}+\frac{1}{z-z_i}\partial_{z_i}\right]\langle X\rangle\\
  \langle \bar{T}(\bar{z})X\rangle&=\sum_{i=1}^n\left[\frac{\bar{h}_i}{(\bar{z}-\bar{z}_i)^2}+\frac{1}{\bar{z}-\bar{z}_i}\partial_{\bar{z}_i}\right]\langle X\rangle
\end{align}
However, we should keep in mind that this local form is only valid within contour integrals of arbitrary holomorphic (antiholomorphic) functions.
}



\subsection{Conformal Ward Identity}

Our above discussion is limited to Primary fields. However, we can extend it to general fields we take the following as a axiom or assumption. We define the infinitesimal variation of the correlation containing general fields:
\axm{\label{axm:conformal ward identity}
  \textbf{Conformal Ward Identity for General Fields}

  For a series of general fields $ X=\prod_{i=1}^n\phi_i(z_i,\bar{z}_i) $ with holomorphic and antiholomorphic dimensions $ (h_i,\bar{h}_i) $, we have:
  \begin{align}
    \delta_{\epsilon}\langle X\rangle=-\frac{1}{2\pi i}\oint_Cdz\epsilon(z)\langle T(z)X\rangle \\ 
    \delta_{\bar{\epsilon}}\langle X\rangle=-\frac{1}{2\pi i}\oint_Cd\bar{z}\bar{\epsilon}(\bar{z})\langle \bar{T}(\bar{z})X\rangle
  \end{align}
}
We will see that in a CFT, we may have other fields that we only know their correlation function but have no idea of their transformation properties, thus we will turn to this definition.


\subsection{OPE Structure}

\subsubsection{Definition of OPE}

Above we notice from the conformal Ward Identity of Primary fields that the correlation function of primary fields and E-M tensor has a divergence structure when they approach each other. This is a general feature of QFT, which can be captured by the Operator Product Expansion (OPE). 
\rmk{
  In fact, we will finally notice that this structure can both give us the data of the correlation functions and the operator commutation relation between fields.
}

\defi{
  \textbf{Operator Product Expansion (OPE)}

  In a QFT, when two fields $ A(z) $ and $ B(w) $ in a correlation function approach each other, their product can be expanded as:
  \begin{align}
    A(z)B(w)=\sum_{n=-\infty}^N\frac{\left\{AB\right\}_n(w)}{(z-w)^n}
  \end{align}
  We understand this equation in the sense of correlation functions. Normally, we negelect the regular part $ n\leq 0 $ and only keep the singular part $ n>0 $. and we write $ \sim $ instead of $ = $ to emphasize this point.
}


\subsubsection{OPE in CFT}

From the above discussions of E-M tensor and Primary fields, we can write down th e EM-Primary OPE and Primary-Primary OPE in CFT. 

\thm{
  \textbf{E-M Tensor - Primary Field OPE}

  For a CFT the OPE between E-M tensor and Primary field $ \phi(z,\bar{z}) $ with holomorphic and antiholomorphic dimensions $ (h,\bar{h}) $ is:
  \begin{align}
    T(z)\phi(w,\bar{w})\sim\frac{h}{(z-w)^2}\phi(w,\bar{w})+\frac{1}{z-w}\partial_w\phi(w,\bar{w})\\
\bar{T}(\bar{z})\phi(w,\bar{w})\sim\frac{\bar{h}}{(\bar{z}-\bar{w})^2}\phi(w,\bar{w})+\frac{1}{\bar{z}-\bar{w}}\partial_{\bar{w}}\phi(w,\bar{w})
  \end{align}
}
This is a direct result of the conformal Ward identity for Primary fields. Moreover we can write down the Primary-Primary OPE. We observe the correlation function between Primary fields \cref{thm:correlation function symmetry} and we notice that we have a natural normalization for Primary fields:
\defi{
  \textbf{Nomralization of Primary Fields}

  We often take the normalization of Primary fields such that:
  \begin{align}
    \langle\phi_j(z_j,\bar{z}_j)\phi_k(z_k,\bar{z}_k)\rangle=\delta_{jk}/(z_j-z_k)^{2h_j}(\bar{z}_j-\bar{z}_k)^{2\bar{h}_j}.
  \end{align}
}
Thus, under this normalization, if we want \cref{thm:correlation function symmetry} to hold, we have to take the Primary-Primary OPE as:
\thm{
  \textbf{Primary Field - Primary Field OPE}

  For a CFT the OPE between two Primary fields $ \phi_i(z,\bar{z}) $ and $ \phi_j(z,\bar{z}) $ with holomorphic and antiholomorphic dimensions $ (h_i,\bar{h}_i) $ and $ (h_j,\bar{h}_j) $ is:
  \begin{align}
    \phi_i(z_1,\bar{z}_1)\cdot\phi_j(z_2,\bar{z}_2) \sim \sum_kC_{ijk}(z_1-z_2)^{-h_i-h_j+h_k}(\bar{z}_1-\bar{z}_2)^{-\bar{h}_i-\bar{h}_j+\bar{h}_k}\phi_k(z_1,\bar{z}_1)+\cdots,
  \end{align}
}
Proof: \textbf{I notice that in many textbooks this proof is missing, so I write it down here for completeness.}

Consider the three-Point function, we expend it by assuming $ z_{12} $ is very small:
\begin{align}
  G_{123}(z_1,z_2,z_3)=C_{123}\ z_{12}^{h_3-h_1-h_2}z_{23}^{h_1-h_2-h_3}z_{13}^{h_2-h_1-h_3},
\end{align}
we make a change of variable $ z_{13}=z_1-z_3=(z_2-z_3)+z_{12}=z_{23}+z_{12}. $Then we can expand it as:
\begin{align}
  z_{13}^{h_2-h_1-h_3}=(z_{23}+z_{12})^{h_2-h_1-h_3}=z_{23}^{h_2-h_1-h_3}\left(1+\frac{z_{12}}{z_{23}}\right)^{h_2-h_1-h_3}
\end{align}
We plug it back to the three-point function and consider the leading order term:
\begin{align}
  G_{123}=C_{123}z_{12}^{h_3-h_1-h_2}z_{23}^{h_1-h_2-h_3}z_{23}^{h_2-h_1-h_3}
\end{align}
then through an observation that we can see $ (h_1-h_2-h_3)+(h_2-h_1-h_3)=-2h_3. $So we have in the leading order:
\begin{align}
  G_{123}=C_{123}\left.z_{12}^{h_3-h_1-h_2}\right.z_{23}^{-2h_3} 
\end{align}
There is another way to calculate this three-point function. We can first use the OPE between $ \phi_1(z_1) $ and $ \phi_2(z_2) $ and calculate the two-point function between the result and $ \phi_3(z_3) $. 

The OPE of $ \phi_1, \phi_2 $ to produce $ \phi_3 $ must have the position dependence $ z_{12}^{-h_1-h_2+h_3} $ to make the three-point function conformally invariant. So we take an ansatz for the OPE:
\begin{align}
  \phi_1(z_1)\cdot\phi_2(z_2)\sim\sum_k K_{12k}\ z_{12}^{-h_1-h_2+h_k}\phi_k(z_2)+\cdots
\end{align}
We plug the ansatz into the three-point function and get:
\begin{align}
  G_{123}&=\langle\phi_1(z_1)\phi_2(z_2)\phi_3(z_3)\rangle\sim\sum_k K_{12k}\ z_{12}^{-h_1-h_2+h_k}\langle\phi_k(z_2)\phi_3(z_3)\rangle\\ 
         &=\sum_k K_{12k}\ z_{12}^{-h_1-h_2+h_k} \delta_{k3}z_{23}^{-2h_3}\\ 
         &= K_{123}\ z_{12}^{-h_1-h_2+h_3} z_{23}^{-2h_3}
\end{align}
Thus we can conclude that $ K_{123}=C_{123} $. 

\qed 

\rmk{
  The Above proof I only use the holomorphic part, but the antiholomorphic part is completely analogous.
}

\rmk{
  Sadly, not all paper take this normalization, so we have to becareful to distingguish the difference between OPE coefficients and correlation function structure constants. So sometimes we see:
  \begin{align}
    \phi_i \cdot \phi_j \sim \displaystyle\frac{C_{ijk}}{C_{kk}}\phi_k+\cdots
  \end{align}
}

\subsection{Complementary Remarks}

\subsubsection{Transformation Rules of E-M Tensor}

From the definition of OPE between E-M tensors and the definition of general conformal Ward identity, we can deduce the transformation rules of E-M tensor under conformal transformation. We take an infinitesimal conformal transformation $ z\to z+\epsilon(z) $, then we have:
\thm{
  \textbf{Transformation Rules of E-M Tensor}

  Under an infinitesimal conformal transformation $ z\to w(z) $, the E-M tensor transforms as:
  \begin{align}
    T^{\prime}(w)=\left(\frac{dw}{dz}\right)^{-2}\left[T(z)-\frac{c}{12}\{w;z\}\right]
  \end{align}
  where $ \{w;z\} $ is the Schwarzian derivative defined as:
  \begin{align}
    \{w;z\}=\frac{(d^3w/dz^3)}{(dw/dz)}-\frac{3}{2}\left(\frac{d^2w/dz^2}{dw/dz}\right)^2
  \end{align}
}
We notice that the schwarzian term vanishes for global conformal transformation, thus the E-M tensor transforms as a quasi-primary field with holomorphic dimension $ h=2 $.





% section Fundamental CFT with Path Integral Approach (end)

\newpage 
\section{CFT Operator Formalism}\label{sec:CFT Operator Formalism} % (fold)
\subsection{Radial Quantization}

The above discussion is purely from the path-integral formalism. However, we can also understand CFT from the canonical quantization approach. This approach is called Radial Quantization and the result is called Operator Formalism. 

\subsubsection{Radial Quantization Setup}

\axm{
  \textbf{Radial Quantization}

  In 2D CFT we use radial direction as the time direction. The Hilbert Space is defined on a circle with fixed radius. 
}
If you want to ask why we do this, well, because this is the only way that gives us amazing results in CFT... 
Now we can make everything look like normal canonical quantization:
\begin{itemize}
  \item We make all field an operator on the Hilbert Space! though knowing nothing about their commutation relation or even the Hilbert Space structure yet.
  \item We make the Diation Operator (though we doesn't know what the hell it is) the Hamiltonian in radial quantization. 
  \item We define time ordering as radial ordering, which orders operators according to their radius (the smaller radius operator is on the right).
  \item ...
\end{itemize}

In the radial quantization, the time direction is the radial direction and the equal time slice is a circle with fixed radius. If we write the coordinate into the holomorphic coordinate and anti-holomorphic then it is shown by the following diagram:
\begin{figure}[H]
  \centering
  \includegraphics[width=0.55\textwidth]{assets/holocoor.png}
  \caption{Radial Quantization in Holomorphic Coordinate}
  \label{fig:holocoor}
\end{figure}

And just like normal QFT, we often define operators at a radius by a contour integral of operators on the radius (for example the mode operators in QFT is defined by a integral on a time slice). For example, we can define an operator $ A_1 $ from an operator $ a(z) $ as
\begin{align}
  A_1=\oint a(z)dz
\end{align}

\subsubsection{Hermite Conjugation in Radial Quantization}\label{sec:Hermite Conjugation in Radial Quantization}

In normal canonical quantization, we define the Hermite Conjugation of operators. In radial quantization, we can define the Hermite Conjugation as well but a natural definition is a bit different:
\defi{
  \textbf{Hermite Conjugation of Primary Field in Radial Quantization}

  For Primary Field $ phi(z,\bar{z}) $ with conformal dimension $ (h,\bar{h}) $, its Hermite Conjugation is defined as:
  \begin{align}
    [\phi(z,\bar{z})]^\dagger=\bar{z}^{-2h}z^{-2\bar{h}}\phi(1/\bar{z},1/z)
  \end{align}
}
This is a strange definition, but it works in many places, so I won't claim more on it. \YL{[I don't think I fully understand this]}


\subsubsection{Mode Expansion in Radial Quantization}

In normal QFT, we often expand field operators in terms of mode operators. In radial quantization, we can do the same thing. For example, for a Primary Field $ \phi(z,\bar{z}) $ with conformal dimension $ (h,\bar{h}) $, we can make a Laurent expansion:
\defi{\label{defi:Mode Expansion of Primary Field in Radial Quantization}
  \textbf{Mode Expansion of Primary Field in Radial Quantization}

  The Mode Expansion of a Field $ \phi(z,\bar{z}) $ with conformal dimension $ (h,\bar{h}) $ is given by:
\begin{align}
  &\phi(z,\bar{z})=\sum_{m\in\mathbf{Z}}\sum_{n\in\mathbf{Z}}z^{-m-h}\bar{z}^{-n-\bar{h}}\phi_{m,n}\\
  &\phi_{m,n}=\frac{1}{2\pi i}\oint dzz^{m+h-1}\frac{1}{2\pi i}\oint d\bar{z}\bar{z}^{n+\bar{h}-1}\phi(z,\bar{z})
\end{align}
}
\begin{itemize}
  \item Attention!! the integral contour is conter-clockwise for the $ z $ integral but clockwise for the $ \bar{z} $ integral. This is adopted to keep the same orientation as the normal integral on $ x-y $ plane.
\end{itemize}
We use this definition for convention, we will then see its convenience through the Hermite Conjugation of Mode Operators and commutation calculations. We then calculate the Hermite Conjugation of Mode Operators:
\thm{\label{thm:Hermite Conjugation of Mode Operators}
  \textbf{Hermite Conjugation of Mode Operators}

  The Hermite Conjugation of the Mode Operators $ \phi_{m,n} $ of a Field $ \phi(z,\bar{z}) $ with conformal dimension $ (h,\bar{h}) $ is given by:
  \begin{align}
    \phi_{m,n}^\dagger=\phi_{-m,-n}
  \end{align}
}


\subsubsection{OPE in Radial Quantization}

In path-integral formalism we define OPE within the correlation functions and say nothing about operators (because we don't have operators in path-integral formalism). However, in radial quantization we have operators, thus we can define OPE as an operator relation:
\thm{
  \textbf{OPE as Operator Relation}

  Consider two operators $ A(z) $ and $ B(w) $, their OPE is defined as the operator relation
  \begin{align}
    \mathcal{R}A(z)B(w)=\sum_{n=-\infty}^N\frac{\left\{AB\right\}_n(w)}{(z-w)^n}
  \end{align}
  where $ \mathcal{R} $ is the radial ordering operator, which orders operators according to their radius (the smaller radius operator is on the right).
}
Proof: due to the relation between correlation functions in path integral formalism and canonical formalism, we can see that this result is straightforward.

The Radial Ordering property of OPE is very important for it naturally generates a commutator structure, which leads to a pronound result.

\subsubsection{Commutator from OPE}

The first amazing result of radial quantization is that we can get the commutator of two operators from their OPE. We first consider an equality:
\thm{
  \textbf{Commutators of local operators from OPE}

Consider two local operators $ a(z) $ and $ b(w) $, and an operator $ A $ defined as:
\begin{align}
  A=\oint dz a(z)
\end{align}
Then we have:
\begin{align}\label{eq:Commutator from OPE basic}
  \oint_wdz \mathcal{R}a(z)b(w)=\oint_{C_1}dza(z)b(w)-\oint_{C_2}dzb(w)a(z)=[A,b(w)]
\end{align}
}
The Proof is quite straightforward. We note that in the middow part we have to change the order of the operators because of the definition of radial ordering. We can generalize this to get the commutator of two operators from their OPE:
\thm{
  \textbf{Commutator from OPE}

Consider two operators on a radial quantization Hilbert Space $ A(z) $ and $ B(w) $, their commutator is given by:
\begin{align}
  [A,B]=\oint_0dw\oint_wdz \mathcal{R}a(z)b(w)
\end{align}
Where $ \mathcal{R}a(z)b(w) $ is exatly the OPE of these two local operaators.
}


\subsection{Symmetry Operator of CFT}

\subsubsection{Symmetry Charges and Algebra}

In canonical quantization, we define the symmetry charge as the quantization of the noether charge which is also the generator of symmetry transformation by commutators. The properyty of OPE between general fields \cref{axm:conformal ward identity} leads to that:
\begin{align}
  \delta_\epsilon\Phi(w)=-[Q_\epsilon,\Phi(w)]
\end{align}
where $ Q_\epsilon $ is:
\begin{align}
  Q_\epsilon=\frac{1}{2\pi i}\oint dz\epsilon(z)T(z)
\end{align}
Proof: we just plug in the conformal ward identity to \cref{eq:Commutator from OPE basic}.

Then we get the theorem:
\thm{
  \textbf{Symmetry Chages in Radial Quantized CFT}

  The Symmetry Charge associated to a conformal transformation $ z\to z+\epsilon(z) $ is given by:
  \begin{align}
    Q_\epsilon=\frac{1}{2\pi i}\oint dz\epsilon(z)T(z)
  \end{align}
  We can as well analogue the following as sort of "Symmetry Current":
  \begin{align}
    J_\epsilon(z)=\epsilon(z)T(z)
  \end{align}
}
We notice that this is a superposition of infinite many symmetry charges. Thus, to make its mathematical structure clear, we can make a Laurent expansion of $ T(z) $, canonically we use the lorant expansion defined above \cref{defi:Mode Expansion of Primary Field in Radial Quantization}:
\begin{align}
  T(z)=\sum_{n\in\mathbf{Z}}z^{-n-2}L_n\quad L_n=\frac{1}{2\pi i}\oint dzz^{n+1}T(z)\\
\bar{T}(\bar{z})=\sum_{n\in\mathbf{Z}}\bar{z}^{-n-2}\bar{L}_n\quad\bar{L}_n=\frac{1}{2\pi i}\oint d\bar{z}\bar{z}^{n+1}\bar{T}(\bar{z})
\end{align}
\begin{itemize}
  \item Again we note that the integral contour is conter-clockwise for the $ z $ integral but clockwise for the $ \bar{z} $ integral. This is adopted to keep the same orientation as the normal integral on $ x-y $ plane.
\end{itemize}
Thus the Symmetry Charge can be written as:
\begin{align}
  Q_\epsilon=\sum_{n\in\mathbf{Z}}\epsilon_nL_n \quad \epsilon(z)=\sum_{n\in\mathbf{Z}}\epsilon_nz^{n+1}
\end{align}
We thus can see that the true internal structure of the conformal symmetry is given by the operators $ L_n $ and $ \bar{L}_n $. 
\thm{
  \textbf{Geniune Symmetry Chages and Symmetry Algebra}

  The Symmetry Charge of Radial Quantized CFT is generated by the operators $ L_n $ and $ \bar{L}_n $, which satisfy the Virasoro Algebra:
  \begin{align}
    &[L_n,L_m]=(n-m)L_{n+m}+\frac{c}{12}n(n^2-1)\delta_{n+m,0}\\
&[L_n,\bar{L}_m]=0\\
&[\bar{L}_n,\bar{L}_m]=(n-m)\bar{L}_{n+m}+\frac{c}{12}n(n^{2}-1)\delta_{n+m,0}
  \end{align}
}
Proof, we just use the OPE trick of EM Tensor to calculate the commutators.

\bigskip
Another important statement is the Hermite Conjugation of Symmetry Charges. From \cref{thm:Hermite Conjugation of Mode Operators} we can directly get:
\thm{
  \textbf{Hermite Conjugation of Symmetry Charges}

  The Hermite Conjugation of the Symmetry Charges $ L_n $ and $ \bar{L}_n $ are given by:
\begin{align}
  L_n^\dagger=L_{-n}\quad \bar{L}_n^\dagger=\bar{L}_{-n}
\end{align}
}

\subsubsection{Symmetry Generators in Radial Quantization}

In a QFT the Hamiltonian is the Time Direction Generator. Thus in the content of radial quantization the Hamiltonian is the Diation Generator.
\defi{
  \textbf{Diation Generator in 2D QFT}

  In a 2D QFT, if we have constructed a \textbf{Traceless and Symmetric} E-M Tensor $ T_{\mu\nu} $, then the Diation Current is given by:
  \begin{align}
    J_D^\mu=x_\nu T^{\mu\nu}
  \end{align}
  Thus the diation charge is given by:
  \begin{align}
    D=\int_\Sigma dS_\mu j_D^\mu=\int_\Sigma dS_\mu x_\nu T^{\mu\nu}(x)
  \end{align}
}
In the content of Radial Quantization we take $ \Sigma $ as a circle and the integral measure as $ dS_\mu=n_\mu ds,\quad n_\mu=\frac{x_\mu}{r},\quad ds=rd\theta. $ in the polar coordinate. Thus we have:
\begin{align}
  D(r)=r^2\oint_0^{2\pi}d\theta\left.n_\mu n_\nu T^{\mu\nu}(r,\theta)\right..
\end{align}
If we transform this result to holomorphic coordinate, we can get:
\thm{
  \textbf{Hamiltonian (Dialation Operator) in Radial Quantization} 

  The Hamiltonian (Dialation Operator) in Radial Quantization is given by:
  \begin{align}
    H=L_0+\bar{L}_0
  \end{align}
  In terms of EM Tensor, it is given by:
  \begin{align}
    H = \frac{1}{2\pi i}\oint_C dz \; z \, T(z)
  \;+\; \frac{1}{2\pi i}\oint_C d\bar z \; \bar z \, \bar T(\bar z)
  = L_0 + \bar L_0
  \end{align}
}
Proof: its just straightforward calculation. We can also confirm this by calculating its commutation relation with Pimary fields and see that it indeed gives the generator of diation transformation.

\bigskip

\YL{[I have to add the normal QFT of angular momentum operators!!]}

Similarly, we can see that the Momentum Operator in radial quantization is given by:
\thm{
  \textbf{Momentum Operator in Radial Quantization} 

  The Momentum Operator in Radial Quantization is given by:
  \begin{align}
    P=(L_0-\bar{L}_0)
  \end{align}
  Which corresponds to rotation around the origin. 
}
\rmk{
  In fact this is the angular momentum operator in normal QFT, but now we have a circle as a equal time slice then it is natural to call it momentum operator.
}
Moreover, we can see that the space-time translation operators in radial quantization are given by:
\thm{
  \textbf{Translation Operators in Radial Quantization} 

  The Translation Operators in Radial Quantization are given by:
  \begin{align}
    P_z=L_{-1}\quad P_{\bar{z}}=\bar{L}_{-1}
  \end{align}
  Which corresponds to translation in $ z $ and $ \bar{z} $ direction respectively.
}
These operators can be understand as a combination of radial translation and rotation. and in the holomorphic coordinate they are exactly the translation in $ z $ and $ \bar{z} $ direction.

\subsection{Hilbert Space of CFT}

\subsubsection{Hilbert Space Structure}

From usual QM we know that the Hilbert Space shall form a representation of the symmetry algebra of the theory. This is sort of a empirical fact, but it works very well in QFT. Thus, we can make it a axiom in CFT:
\axm{
  \textbf{Hilbert Space as Virasoro Algebra Representation}

  The Hilbert Space of a 2D CFT forms a representation of the Virasoro Algebra generated by $ L_n $ and $ \bar{L}_n $. Which may be constructed by the highest weight representation of the Virasoro Algebra.
}
The representation of 2 virasoro algebras can be constructed by the tensor product of two single Virasoro Algebra representations. Thus, we only need to consider the representation of a single Virasoro Algebra. 

We further introduce the concept of Heighest Weight Representation to build the representation of the Virasoro Algebra:
\begin{itemize}
  \item \textbf{Highest Weight State}: assume there exists a state $ |h,\bar{h}\rangle $ in the Hilbert Space such that:
    \begin{align}
      L_0|h\rangle=h|h\rangle \quad L_n|h\rangle=0\quad n>0
    \end{align}
\item \textbf{Highest Weight Representation}: we can build a representation of the Virasoro Algebra by acting the lowering operators $ L_{-n} $ ($ n>0 $) on the highest weight state $ |h\rangle $. The states in this representation are called Descendant States.
\item \textbf{General Representation Space}: The representation space of the Virasoro Algebra is given by the direct sum of all highest weight representations:
  \begin{align}
    \mathcal{R}=\bigoplus_{h,\bar{h}}\mathcal{V}_{h}\otimes\bar{\mathcal{V}}_{\bar{h}}
  \end{align} 
\end{itemize}
\rmk{
  This subsubsection what we discuss is pure math and $ L_n $ is just abstract lie algebra. By "state" we just mean an abstract vector in a vector space. We haven't yet connect this math structure to the physical operators and states in CFT. This will be done in the next subsubsection.
}
Then we want to find a pysical realization of this math structure in CFT.

\subsubsection{Hilbert Space Realization}

From normal QFT, we know that we can assume a well defined vacuum state $ |0\rangle $ in the Hilbert Space. And then we act all possible operators on the vacuum to get all kinds of states in the Hilbert Space. In this general CFT construction, we know that the only operators we have now is:
\begin{itemize}
  \item The Symmetry Charges $ L_n $ and $ \bar{L}_n $
  \item The Primary Fields $ \phi(z,\bar{z}) $ 
\end{itemize}
\textbf{
The question is that can we get the structure of direct sum over highest weight representations (Virasoro Algebra Representation) by acting these operators on some vacuum? The answer is yes, we can. 
}

\begin{itemize}
  \item \textbf{Step 1: Define the Vacuum State}
\end{itemize}
we first define the vacuum state $ |0\rangle $ as the state invariant under all global conformal transformations:
\defi{
  \textbf{Vacuum State in CFT}  

  The Vacuum State $ |0\rangle $ in CFT is defined as the state invariant under all global conformal transformations and well defined when acted on by all local operators. In terms of Symmetry Charges, it satisfies:
  \begin{align}
    L_{-1}|0\rangle=L_0|0\rangle=L_1|0\rangle=0\\
\bar{L}_{-1}|0\rangle=\bar{L}_0|0\rangle=\bar{L}_1|0\rangle=0
  \end{align}
  If we assume that the E-M Tensor act well on vacuum, then we have:
  \begin{align}
    T(0)|0\rangle=\bar{T}(0)|0\rangle=0
  \end{align}
  which gives us:
  \begin{align}
    L_n|0\rangle=\bar{L}_n|0\rangle=0\quad n\geq -1
  \end{align}
}
This also means that the vacuum energy is zero:
\begin{align}
  \langle0|T(z)|0\rangle=\langle0|\bar{T}(\bar{z})|0\rangle=0
\end{align}

\begin{itemize}
  \item \textbf{Step 2: Define Primary States}
\end{itemize}

We then try to act Primary Fields on the vacuum to get some states. We define:
\defi{
  \textbf{Primary State in CFT}

  The Primary State $ |h,\bar{h}\rangle $ associated to a Primary Field $ \phi(z,\bar{z}) $ with conformal dimension $ (h,\bar{h}) $ is defined as:
  \begin{align}
    |h,\bar{h}\rangle=\phi(0,0)|0\rangle
  \end{align} 
}
This state has real unique properties. To show this we first calculate the operator commutator between $ L_n $ and $ \phi(z,\bar{z}) $:
\begin{align}
  [L_n,\phi(z,\bar{z})]=z^{n}\left( (n+1)h+z\partial_z \right)\phi(z,\bar{z}) \quad n \geq -1\\ 
  [\bar{L}_n,\phi(z,\bar{z})]=\bar{z}^{n}\left( (n+1)\bar{h}+\bar{z}\partial_{\bar{z}} \right)\phi(z,\bar{z}) \quad n \geq -1
\end{align}
With these commutators we can calculate the action of $ L_n $ and $ \bar{L}_n $ on the Primary State:
\begin{align}
  L_0|h,\bar{h}\rangle=h|h,\bar{h}\rangle\quad\bar{L}_0|h,\bar{h}\rangle=\bar{h}|h,\bar{h}\rangle \\ 
  L_n|h,\bar{h}\rangle=\bar{L}_n|h,\bar{h}\rangle=0\quad n>0
\end{align}
This means that:
\thm{
  \textbf{Primary State as Highest Weight State}

  The Primary State $ |h,\bar{h}\rangle $ associated to a Primary Field $ \phi(z,\bar{z}) $ with conformal dimension $ (h,\bar{h}) $ behaves like a Highest Weight State of the Virasoro Algebra with highest weight $ (h,\bar{h}) $.
}

\begin{itemize}
  \item \textbf{Step 3: Build the Hilbert Space}
\end{itemize}
Naturally, we then find all the primary states and act the lowering operators onto them to build a representation of the Virasoro Algebra. 
States corresponds to Virasoro Generators acting on Primary States are called Descendant States:
\thm{
  \textbf{Descendant State in CFT}

  The Descendant States in CFT are defined as the states obtained by acting the lowering operators $ L_{-n} $ ($ n>0 $) and $ \bar{L}_{-n} $ ($ n>0 $) on Primary States.
  \begin{align}
    L_{-k_1}L_{-k_2}\cdots L_{-k_n}|h\rangle\quad(1\leq k_1\leq\cdots\leq k_n)\\ 
    \bar{L}_{-l_1}\bar{L}_{-l_2}\cdots \bar{L}_{-l_m}|\bar{h}\rangle\quad(1\leq l_1\leq\cdots\leq l_m)
  \end{align}
}

We then take the direct sum of all these representations to build the Hilbert Space. 
\defi{
  \textbf{Hilbert Space of CFT}

  The Hilbert Space $ \mathcal{H} $ of a CFT is given by the direct sum of all highest weight representations of the Virasoro Algebra built from all Primary States:
  \begin{align}
    \mathcal{H}=\bigoplus_{h,\bar{h}}\mathcal{V}_{h}\otimes\bar{\mathcal{V}}_{\bar{h}} = \bigoplus_{h,\bar{h}}\left\{L_{-k_1}L_{-k_2}\cdots L_{-k_n}|h,\bar{h}\rangle\quad(1\leq k_1\leq\cdots\leq k_n)\right\}
  \end{align} 
}

\begin{itemize}
  \item \textbf{Final Step: Test? is it done?}
\end{itemize}

We finally show that all states generated by primary fields and E-M tensor acting randomly on the vacuum can be expressed as linear combinations of the above defined states in the Hilbert Space. I want to discuss a few cases:
\begin{enumerate}
  \item \textbf{Case 0: Single Primary and many Virasoro Generators}  

    We only have to commute them to the right order to get the canonical form.
  \item \textbf{Case 1: Many Primaries} consider a state generated by many primary fields acting on the vacuum:
    \begin{align}
      |\psi\rangle=\phi_1(z_1,\bar{z}_1)\phi_2(z_2,\bar{z}_2)\cdots\phi_n(z_n,\bar{z}_n)|0\rangle
    \end{align}
    We can use the OPE of primary fields (Though we haven't yet discussed it) but consistenely make it a sum of single primary field and descendents acting on the vacuum.
  \item \textbf{Case 2: Only E-M Tensor} consider a state generated only by E-M Tensor acting on the vacuum. In fact by definition:
    \begin{align}
      T(0)|0\rangle = L_{-2}|0\rangle 
    \end{align}
    In fact we can view the vaccum as a primary field with conformal dimension $ (0,0) $, thus E-M tensor is the 2nd level descendent of the vacuum primary field. Thus it has nothing different from the zero and first case.
  \item \textbf{Case 3: Field on Points other than 0} consider a state generated by a field acting on a point other than 0:
    \begin{align}
      |\psi\rangle=\phi(z,\bar{z})|0\rangle\quad z\neq 0
    \end{align}
    We can use the translation symmetry generated by $ L_{-1} $ and $ \bar{L}_{-1} $ to translate the point to 0:
    \begin{align}
      \phi(z+w,\bar{z}+\bar{w})=e^{zL_{-1}+\bar{z}\bar{L}_{-1}}\phi(w,\bar{w})e^{-zL_{-1}-\bar{z}\bar{L}_{-1}}
    \end{align}
    Then due to the vacuum is invariant under translation, we have:
    \begin{align}
      \phi(z,\bar{z})|0\rangle=e^{zL_{-1}+\bar{z}\bar{L}_{-1}}\phi(0,0)|0\rangle
    \end{align}
    which is in the Hilbert Space.
\end{enumerate}
Thus we can argue that all states generated by primary fields and E-M tensor acting randomly on the vacuum can be expressed as linear combinations of the states in the Hilbert Space defined above.

\YL{[Nearly all textbook skips this part, but it is really important ]}
\rmk{
  Though we write the Hilbert Space as a theorem. Its in fact more like something we only know because we only have the knowledge of primary field and E-M tensor. So in fact it is more like an axiom than a theorem.
}

\subsubsection{Inner Product}

As a valid Hilbert Space, we also need to define an inner product. We find that the above definition of Hermite Conjugation in radial quantization \cref{sec:Hermite Conjugation in Radial Quantization} gives us a natural inner product:
\defi{
  \textbf{Inner Product in Radial Quantization}

  The Inner Product between two states $ |\psi_1\rangle $ and $ |\psi_2\rangle $ in Radial Quantization is defined as:
  \begin{align}
    \langle\psi_1|\psi_2\rangle = \langle0|[\psi_1]^\dagger \psi_2|0\rangle
  \end{align}
}
Pratically we can calculate the inner product by commuting all the Virasoro Generators and make the Lowering Operators to act on Primary States. Then we compute the inner product between Primary States:
\thm{
  \textbf{Inner Product between Primary States}

  The inner product between two primary states 
  $|h_1,\bar{h}_1\rangle=\phi_1(0,0)|0\rangle$ 
  and 
  $|h_2,\bar{h}_2\rangle=\phi_2(0,0)|0\rangle$
  is determined by the two--point function of their corresponding primary fields:
  \begin{align}
    \langle h_1,\bar{h}_1 \mid h_2,\bar{h}_2\rangle
    &=\lim_{z,\bar z,w,\bar w\to 0}
      \langle 0|\phi_1(z,\bar z)^\dagger\,\phi_2(w,\bar w)|0\rangle \\[6pt]
    &=\lim_{z,\bar z,w,\bar w\to 0}
      z^{-2h_1}\bar z^{-2\bar h_1}
      \Big\langle 0\Big|
        \phi_1\!\Big(\tfrac1{\bar z},\tfrac1z\Big)
        \phi_2(w,\bar w)
      \Big|0\Big\rangle \\[6pt]
    &=\lim_{\xi,\bar\xi\to\infty}
      \bar\xi^{2h_1}\xi^{2\bar h_1}\,
      \langle 0|\phi_1(\bar\xi,\xi)\phi_2(0,0)|0\rangle ,
      \qquad
      (\xi=1/\bar z,\;\bar\xi=1/z) \\[6pt]
    &=C_{12},
  \end{align}
  where $C_{12}$ is the coefficient of the two--point function 
  $\langle \phi_1(z,\bar z)\phi_2(0,0)\rangle
    =C_{12}\,z^{-2h_1}\bar z^{-2\bar h_1}$.
  In particular, primary states are orthogonal unless they share the same conformal dimensions:
  \begin{align}
    \langle h_1,\bar h_1 \mid h_2,\bar h_2\rangle
    = C_{12}\,\delta_{h_1,h_2}\,\delta_{\bar h_1,\bar h_2}.
  \end{align}
}



\subsection{Primary and Descendent Fields}
We can see that the Hilbert space is made by Primaries and Descendents. In fact, the operators in CFT also have the same structure. We then further this in this subsection.

\subsubsection{State-Operator Correspondence}

We now notice that we assume the spectrum is exactly given by states at 0 point acting on the vaccum. This leads to a amazing result called State-Operator Correspondence:
\thm{
  \textbf{State-Operator Correspondence in CFT}

  There is a one-to-one correspondence between local operators $ \mathcal{O}(z,\bar{z}) $ in CFT and states $ |\mathcal{O}\rangle $ in the Hilbert Space given by the following relation:
  \begin{align}
    |\mathcal{O}\rangle=\mathcal{O}(0,0)|0\rangle
  \end{align}
}

\subsubsection{Descendent Operators}
We then notice that we can define an operator correspong to the descendant states as well:
\defi{
  \textbf{Descendant Operator in CFT}

  The Descendant Field of a Primary Field $ \phi(z,\bar{z}) $ with conformal dimension $ (h,\bar{h}) $ is defined as the operator obtained by acting the Virasoro Generators $ L_{-n} $ ($ n>0 $) and $ \bar{L}_{-n} $ ($ n>0 $) on the Primary Field:\begin{align}
    \phi^{(-n)}(w)\equiv(L_{-n}\phi)(w)=\frac{1}{2\pi i}\oint_wdz\frac{1}{(z-w)^{n-1}}T(z)\phi(w)
  \end{align}
  Generally we can act many Virasoro Generators to get higher level descendant fields:
  \begin{align}
    \phi^{(-k,-n)}(w)\equiv(L_{-k}L_{-n}\phi)(w)=\frac{1}{2\pi i}\oint_wdz\left(z-w\right)^{1-k}T(z)(L_{-n}\phi)(w)
  \end{align}
}
In particular, we have:
\begin{align}
  \phi^{(0)}(w)=h\phi(w)\quad\mathrm{and}\quad\phi^{(-1)}(w)=\partial\phi(w)
\end{align}

\subsubsection{Operator Data of CFT}

Thus more as a assumption, but with good reason (just like the Hilbert Space is), we define the Operator content of CFT as:
\thm{
  \textbf{Operator Content of CFT}

  The Operator Content of a CFT is given by all Primary Fields and their Descendant Fields.
}

\subsubsection{Correlation Functions with Descendant Operators}

Using the conformal ward identity of primary fields we can construct a relation between correlation functions with descendant operators and correlation functions with only primary fields:
\thm{
  \textbf{Correlation Functions with Descendant Operators(1)}

  The Correlation Functions with Descendant Operators can be expressed as differential operators acting on correlation functions with only primary fields. 

  For a descendant operator $ \phi^{(-n)}(z) $ of a primary field $ \phi(z) $ with conformal dimension $ h $, and a series of Pirmary Fields $ X $ we have:
  \begin{align}
\langle \phi^{(-n)}(w)\,X \rangle = \mathcal{L}_{-n}\,\langle \phi(w)\,X\rangle ,
\qquad (n\ge 1) 
  \end{align}
  Where:
  \begin{align}
    \mathcal{L}_{-n}=\sum_i\left\{\frac{(n-1)h_i}{(w_i-w)^n}-\frac{1}{(w_i-w)^{n-1}}\partial_{w_i}\right\}
  \end{align}
}
Proof: see the following calculation:
  \begin{align}
\langle \phi^{(-n)}(w)\,X \rangle
&= \frac{1}{2\pi i}\oint_{w} dz\,(z-w)^{\,1-n}\,
   \langle T(z)\,\phi(w)\,X \rangle \\[6pt]
&= -\,\frac{1}{2\pi i}
   \oint_{\{w_i\}} dz\,(z-w)^{\,1-n}
   \sum_i \left[
      \frac{1}{z-w_i}\,\partial_{w_i}\langle \phi(w)\,X\rangle
      + \frac{h_i}{(z-w_i)^2}\,\langle \phi(w)\,X\rangle
   \right] \\[6pt]
& = \mathcal{L}_{-n}\,\langle \phi(w)\,X\rangle ,
\qquad (n\ge 1) 
  \end{align}
We can generalize to higher level descendant operators and even to the case $ X $ containing descendant operators as well:
\thm{
  \textbf{Correlation Functions with Descendant Operators(2)}

  The Correlation Functions with Descendant Operators can be expressed as differential operators acting on correlation functions with only primary fields. 

  For a descendant operator $ \phi^{(-k,-n)}(z) $ of a primary field $ \phi(z) $ with conformal dimension $ h $, and a series of Pirmary Fields $ X $ we have:
  \begin{align}
    \langle \phi^{(-k,-n)}(w)\,X \rangle = \mathcal{L}_{-k}\mathcal{L}_{-n}\,\langle \phi(w)\,X\rangle ,
\qquad (k,n\ge 1)
  \end{align}
  Where $ \mathcal{L}_{-n} $ is defined as above. More generally, we have:
  \begin{align}
    \langle\phi^{(-k_1,...,-k_n)}(w)X\rangle=\mathcal{L}_{-k_1}\cdots\mathcal{L}_{-k_n}\langle\phi(w)X\rangle
  \end{align}
}
In general, a correlation function we many descendant operators can be expressed as differential operators acting on correlation functions with only primary fields, but may take more detailed consideration.

\subsubsection{OPE with Descendant Operators}

Similar to the Primaries, we can calculate the OPE of Descendant fields and the E-M Tensor. Please refer to yellow book 6.6.2 for details. 

We can also calculate the full OPE between two Primary Fields including all descendant fields. Please refer to yellow book 6.6.3 for details.


\subsection{Complementary Remarks}

\subsubsection{Radial Quantization on Cylinder}

We use the holomorphic coordinate to do the radial quantization. In fact another more physical and meaningful way is to do the radial quantization on a cylinder coordinate system. Which is related to the holomorphic coordinate by the conformal transformation:





% section CFT Operator Formalism (end)

\newpage
\section{Summary: General Structure of CFT}\label{sec:Summery: General Structure of CFT} % (fold)
\subsection{Hilbert Space and Operator Contents}

In a General CFT, the Hilbert Space and all Operator Contents can be classified into the Primaries and Their Descendent.
\imp{Hilbert Space}{
  The Hilbert Space $ \mathcal{H} $ of a CFT is spanned by the highest weight representations of the Virasoro Algebra. Each highest weight representation is generated by a Primary State $ |h,\bar{h}\rangle $ and its Descendent States $ L_{-n_1}L_{-n_2}...|h,\bar{h}\rangle  $ ($ n_i>0 $). 
  \begin{align}
    \mathcal{H} = \bigoplus_{h,\bar{h}} \mathcal{V}_h \otimes \bar{\mathcal{V}}_{\bar{h}} ,
  \end{align}
}

And the Operators are classified similarly
\imp{Operator Content}{
The Operator Content of a CFT is classified by the Primary Operators $ \phi_{h,\bar{h}}(z,\bar{z}) $ and their Descendent Operators $ \phi_{h,\bar{h}}^{(-k_1,...,-k_n)}(z,\bar{z})  $ ($ k_i>0 $).
}


\subsection{Correlation Functions}

In fact, the full information of a CFT correlation can be extracted if we know the following data:
\begin{itemize}
  \item Central Charge $ c $ 
  \item Hilbert Space or the Operator Content
  \item Structural Constants $ C_{abc} $
\end{itemize}
Here I give a biref discussion of the general procedure:
\begin{itemize}
  \item \textbf{Step 1:} Use the Conformal Ward Identity to reduce the N-point Correlation Functions of Descendents to N-point Correlation Functions of Primaries.
  \item \textbf{Step 2:} Use the OPE of Primaries to N-point Correlation Functions of Primaries to 2-Point Correlation Functions of Primaries. 
\end{itemize}
The conformal ward identity is fully defined with the knowledge of central charge and the Hilbert space (or operator content). The OPE of Primaries is fully defined with the knowledge of the Hilbert space (or operator content) and the Structural Constants. Thus, the full information of a CFT correlation function is encoded in these three data.


\subsection{Conformal Boostrap}

The core idea is of conformal bootstrap is to ask what kinds of combination of the three data (central charge, Hilbert space/operator content, structural constants) can give a self-consistent CFT. The self-consistency conditions include:
\begin{itemize}
  \item Unitary of the Hilbert Space
  \item Crossing Symmetry of the 4-Point Correlation Functions
  \item Modular Invariance of the Torus Partition Function
\end{itemize}


% section Summery: General Structure of CFT (end)

\newpage 
\section{Operator Algebra Formalism}\label{sec:Bootstrap Idea} % (fold)
\input{Bootstrap.tex}
% section Bootstrap Idea (end)

\end{document}
