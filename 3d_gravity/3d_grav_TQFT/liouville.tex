这个部分主要就是follow hung1法师的文章:“Quantum 2D Liouville Path-Integral Is a Sum over Geometries in AdS3 Einstein Gravity”的内容。我突然意识到我能看懂了!!

\section{主要内容介绍}
这个文章之中,我们使用三角剖分以及缩小动动的经典的TN的操作给出了一个Liouville的配分函数。并且这个配分函数可以理解为3D的state-sum(在一定的边界条件下)。这在c十分大的时候可以理解为以Einstein-Hilbert作为weight的一个求和。

\imp{CFT边界条件信息}{
    有一个基本的结论是RCFT里面,如果我们知道CFT的所有允许的边界条件,那么我们可以给出这个理论的全部的信息!!
}
Liouville理论虽然并不是RCFT但是他的基本上所有boundary的信息全都已经有了,所以可以仿照RCFT进行操作。

RCFT可以写成TN或者strange correlator的样子,其实是non-invertible symmetry语境下面的holographic "sandwich"的技术。其中,non-invertible symmetry可以很清晰的使用TDL进行一个描述!!

从RCFT推广到irCFT的时候最大的问题是,从一个不连续的求和并且是对于有限多的数字求和变成了一个连续的对于无限的数字的积分。所以可能积分的时候会有发散的问题。但这篇文章我们可以规避这个问题因为我们并不是考虑一个irrational推广的TV TQFT的波函数$ \ket{\Psi_{\mU_q(SL(2,\mathcal{R}))}} $而是考虑一个strange correlator的数值,也就是$ \braket{\Omega}{\Psi_{\mU_q(SL(2,\mathcal{R}))}} $ 在这个情况下正交的条件可以成立,并且积分是收敛的。

\section{构建缩小的洞洞}

