这里记录一些补充的知识,首先我们分成专题的补充很多的前置的知识:

\section{Classical 3D Gravity and Chern-Simons}
我们知道经典的三维引力可以使用TQFT进行表达。也就是经典的三维引力和一个$ PSL(2,\mathbb{R}) \times  PSL(2,\mathbb{R})$ 的Chern-Simons theory有联系。具体的说就是:
\imp{经典三维引力}{
    三维引力的1st order formulism可以进行redefine并使得其EoM和action对应一个$ SL(2,\mathbb{R})_k \times SL(2,\mathbb{R})_{-k} $的Chern-Simons理论。 
}


\subsection{Vielbein and spin connection Formalism}
我们考虑的是first-order formalism。也就是说我们并不会使用metric $ g_{\mu\nu} $作为一个基本的场,而是选择另外一个场$ e_\mu^a $作为一个基本的场,也就是\textbf{frame field}或者在三维里面我们成为dreibein。
\imp{Dreibein Formalism}{
    我们定义一些矩阵称之为dreibein,如下:
    \eq{
        g_{\mu\nu}(x)=e_\mu^a(x)\eta_{ab}e_\nu^b(x),
    }
    下面我们给出一些性质:
    \itm{
        \pt{$ e_\mu^a $矩阵可以理解为一个坐标变换矩阵,是对偶矢量的基的坐标变换矩阵。$ e^a = e_\mu^a dx^\mu $  }
        \pt{e矩阵的行列式一定是非零的,因为这是行列式定义决定的,我们把定义式写成矩阵形式就是:$ g = e^T \eta e $ 
        所以行列式的关系就是:$ \det(g) = - \det(e)^2 $ 
        我们可以如下进行定义:
        \eq{
            e = \det(e) = \sqrt{- \det(g)}
        }
        }
        \pt{由于横列式不为0,我们可以定义逆矩阵,也就是inverse frame field满足下面关系:
        \eq{
            e_\mu^ae_b^\mu=\delta_b^a\mathrm{~and~}e_a^\mu e_\nu^a=\delta_\nu^\mu.
        }
        }
        \pt{
            很容易发现对于$ e_\mu^a $的构造并不是唯一的。由于洛伦兹群并不改变minkovski度规,所以我们对$ e_\mu^a $进行一个洛伦兹变换其实完全也满足定义式子,所以我们会发现不同的$ e_\mu^a $满足下面关系:
            \eq{
                e_\mu^{\prime a}=\Lambda_b^{-1a}(x)e_\mu^b(x)\mathrm{~with~}\Lambda\in SO(2,1)
            }   
        }
    }
}  
下面我们希望做流形上的微积分,毕竟所有的action其实都是流行上的微积分的结果。为此,我们需要通过dreibein写出微分形式之中的重要组成部分。我们有下面的构造:
\itm{
    \pt{\textbf{1-form}:首先很显然可以构造一个1-form就相当于新的坐标的基:
    \eq{
        e^a\equiv e_\mu^adx^\mu
    }
    }
    \pt{
        \textbf{levi-civita Symbol}:这个可以显然通过$ e \epsilon_{\mu\nu\rho} $ 是一个张量,并进行坐标变换得到!
        \eq{
\epsilon_{\mu\nu\rho}&\equiv e^{-1}\epsilon_{abc}e_{\mu}^{a}e_{\nu}^{b}e_{\rho}^{c},\\\epsilon^{\mu\nu\rho}&\equiv e\operatorname{\epsilon}^{abc}e_{a}^{\mu}e_{b}^{\nu}e_{c}^{\rho}.
        }
        \textbf{注意,我们这里虽然都没有加帽子,这些$ \epsilon_{\mu\nu\rho} $都是symbol并不是tensor!!!!}
    }
}
下面我们定义新的坐标下面的协变导数(注意我们的spin connection在坐标变换下面并不是按照tensor变换的。)。我们可以定义connection coefficient:
\defi{
    connection coefficient

    对于某一个度规我们可以给出一个协变导数的connection coefficient,也就是我们的Christoffel symbol在坐标变换下面的结果:(注意我们这里的a,b是抽象指标,不是基矢量的指标!!)
    \eq{
        \left(e_{\tau}\right)^{b}\nabla_{b}(e_{\mu})^{a}=\gamma^{\sigma}{}_{\mu\tau}(e_{\sigma})^{a}
    }
    或者说:
    \eq{
        \gamma^\nu{}_{\mu\tau}=(e^\nu)_a(e_\tau)^b\nabla_b(e_\mu)^a.
    }
    这个式子等价于Chirstoffel symbol的坐标变换关系,也就是说,$ \gamma^\nu{}_{\mu\tau} $本质上就是新的基的联络:
    \eq{
        \Gamma_{\mu^{\prime}\lambda^{\prime}}^{\nu^{\prime}}=\frac{\partial x^\mu}{\partial x^{\mu^{\prime}}}\frac{\partial x^\lambda}{\partial x^{\lambda^{\prime}}}\frac{\partial x^{\nu^{\prime}}}{\partial x^\nu}\Gamma_{\mu\lambda}^\nu-\frac{\partial x^\mu}{\partial x^{\mu^{\prime}}}\frac{\partial x^\lambda}{\partial x^{\lambda^{\prime}}}\frac{\partial^2x^{\nu^{\prime}}}{\partial x^\mu\partial x^\lambda}.
    }
    \textbf{还有另一套记号我觉得更方便一点点:(这里我们并不使用抽象指标,a啥的都是基的指标!!)}
    我们定义connection coefficient满足下面的关系:
    \eq{
        e_a^\mu\nabla_\mu e_b=\nabla_ae_b=e_c\omega_{ab}^c
    }
    或者说:
    \eq{
        \omega^{c}{}_{ab}=e^{c}{}_{\nu}e_{a}{}^{\mu}(\partial_{\mu}e_{b}{}^{\nu}+e_{b}{}^{\alpha}\Gamma_{\mu\alpha}^{\nu})=e^{c}{}_{\nu}e_{a}{}^{\mu}\nabla_{\mu}e_{b}{}^{\nu}.
    }
}
\imp{Spin Connection}{
    从connection coefficient出发我们可以定义一个1-form,我们称之为spin connection 1-form:
\defi{
    spin connection

    我们定义一个1-form为:
    \eq{
        \omega^c{}_b = \omega^c{}_{ab} e^a
    }
    \textbf{注意!!!对于Torsion Free的导数算符,我们的联络下面两个指标是Symmetric的!!!!}
    这个1-form和Dreibein的关系是:
    \eq{
        \omega^c{}_b =  e^{c}{}_{\nu}e_{a}{}^{\mu}(\nabla_{\mu}e_{b}{}^{\nu} ) e^a_\eta dx^\eta = e^{c}{}_{\nu}(\nabla_{\mu}e_{b}{}^{\nu} )  dx^\mu
    }
}

我们会发现另外的一个性质,由于我们的dreibein有一个lorenzian的对称性,所以进行洛伦兹变换之后的spin connection也是可以求出来的。通过下面的推导我们会发现满足下面的变换关系:
\eq{
    \omega^{\prime a}{}_{b}=\Lambda^{a}{}_{c}\omega^{c}{}_{d}(\Lambda)^{d}{}_{b}+\Lambda^{a}{}_{c}(d\Lambda)^{c}{}_{b}.
}
推导过程如下:
\pict{2025-03-10-18-15-14.png}{1}
这里提示一下,这个变换关系显然满足一个
}

下面我们给出两个Cartan Structure的结论,我们会发现second cartan structure其实和曲率张量有关系。我们下面进行说明:
\thm{first Cartan structure equation

我们可以通过组合spin connection和Dreibein组合出来一个2-form。并且可以证明这个2-form:
\eq{
    T^a\equiv de^a+\omega^a{}_b\wedge e^b\mathrm{~,}
}
很容易发现这个式子满足下面的性质:
\itm{
    \pt{$ T^a = 0 $恒成立。这个就是第一cartan结构方程}
    \pt{这个量按照lorenzian变换:
    \eq{
        T^a\to\Lambda^{a}{}_bT^b
    }
    }
}
}

下面一些证明:
\pict{2025-03-10-18-22-47.png}{0.85}

\thm{
    second Cartan structure equation

    % 这个方程说的是我们的spin connection和Riemann tensor之间的关系,是这样的:

    % 我们首先可以从黎曼张量在某一个坐标系下面的分量构建出来一个量:
    % \eq{
    %     R_{\mu\nu b}{}^{a} =  R_{\mu\nu \rho}{}^{\eta} e^a_\eta e^\rho_b 
    % }
    % 这个时候我们可以认为构建了一个2-form:
    % \eq{
    %     R^a{}_b =  R_{\mu\nu \rho}{}^{\eta} e^a_\eta e^\rho_b dx^\mu \wedge dx^\nu
    % }这是一个用a,b两个指标构建出来的2-form。
    % 我们可以认为这是一个2-form或者说,可以直接从张量积的基础换到全反对称的2-form的基,这是因为黎曼曲率张量的前两个指标是反对称的指标。下面是一些关系的另一个notation的重写:
    % \eq{
    %     &R^{ab}=\frac{1}{2}R^{ab}{}_{\mu\nu}(x)dx^\mu\wedge dx^\nu,\\&R^{\lambda\sigma}{}_{\mu\nu}=e_a^\lambda e_b^\sigma R^{ab}{}_{\mu\nu}.
    % }
    \vspace{0.7em}
    \textbf{使用一个特殊的notation(lcb)}

    下面我们给出一个关系,成为second cartan structure equation。这个方程告诉了我们曲率和spin connection之间的关系。(就可以类比,我们之前Christolff symbol和黎曼张亮起时有一个很好的关系,那么下面这个方程其实就是这个关系的一种推广。)

    首先,我们可以从黎曼张量得到一个2-form $ (R_\mu{}^\nu)_{ab} $这个是一个用两个数 $ \mu $和$ \nu $标记的对偶矢量场,并且是全反对称的对偶矢量是2-form,我们定义为:
    \eq{
        R_{ab\mu}{}^\nu\equiv R_{abc}{}^d(e_\mu)^c(e^\nu)_d, \quad  R_\mu{}^\nu=\frac{1}{2}R_{\rho\sigma\mu}{}^\nu e^\rho\wedge e^\sigma,
   }
    其中$ (e_\mu)^c $ 是一个用$ \mu $标记的矢量,我们用c作为矢量的抽象指标。这个是正向的变换,其实反向的变换也是可行的。我们可以有:
    \eq{
        R_{\rho\sigma\mu}{}^{\nu}=R_{ab\mu}{}^{\nu}(e_{\rho})^{a}(e_{\sigma})^{b}.
    }
    
    
    Second Cartan Structure equation告诉我们下面的事实:
    \eq{
        (R_\mu{}^\nu)_{ab} =\mathrm{d}\boldsymbol{\omega}^\nu{}_\mu+\boldsymbol{\omega}^\lambda{}_\mu\wedge\boldsymbol{\omega}^\nu{}_\lambda.
    } 


    \vspace{0.7em}
\textbf{使用另一个notation(和前面section统一)}

    其中$ \boldsymbol{\omega}_\mu^\nu $ 就是我们上面的定义的对偶矢量$ \omega^c{}_b $应该乘以一个-1 。只是这里我们使用了一些不太一样的notation。我们一般会把$ \mu $指标升上去,也就是线性组合一波。 $ \omega^{ab} = \eta^{ac} \omega^b{}_c $,这样升上去可以证明这个1-form的两个指标是对称的。或者说相应的对于黎曼张量对应定义,但同时因为符号问题,可能还需要一个指标对称性的变换。这个notation下面定义是:
    \eq{
&R^{ab}=\frac{1}{2}R_{\mu\nu}^{ab}(x)dx^\mu\wedge dx^\nu,\\&R^{\lambda\sigma}{}_{\mu\nu}=e_a^\lambda e_b^\sigma R^{ab}{}_{\mu\nu}.
    }
    用的是升上去一个指标的黎曼张量定义的!!!
    
    我们改写上面的公式有:
    \eq{
        d\omega^{ab}+\omega^a{}_c\wedge\omega^{cb}=R^{ab},
    }
    或者说用分量:
    \eq{
        (R^{ab})_{\mu\nu}=\partial_\mu\omega_\nu^{ab}-\partial_\nu\omega_\mu^{ab}+\omega_\mu^{ac}\omega_{\nu c}^b-\omega_\nu^{ac}\omega_{\mu c}^b
    }
\imp{second Cartan structure equation}{
    所以就是说公式是:
    \eq{
        d\omega^{ab}+\omega^a{}_c\wedge\omega^{cb}=R^{ab},
    }
}
    \textbf{【上方的讨论是有问题的,因为了lcb书里的$ \omega^\mu{}_\nu $的定义其实多了个-1,但是,这个符号通过对于黎曼曲率张量的指标变换,消去了呃呃呃呃呃所以在反正必然是错了 】}
}

\subsection{Einstein Hilbert Action}
对于E-H action(我们暂时并不考虑边界项)
\eq{
    S_{\mathrm{EH}}[g]\equiv\frac{1}{16\pi G}\int_{\mathcal{M}}d^3x\sqrt{-g}\left(R-2\Lambda\right)+B,
}
我们可以用上面的formalism进行改写,变成:
\eq{
    S_{\mathrm{EH}}[e,\omega]=\frac{1}{16\pi G}\int_{\mathcal{M}}\epsilon_{abc}\left(e^a\wedge R^{bc}[\omega]-\frac{\Lambda}{3}e^a\wedge e^b\wedge e^c\right).
}
下面我们进行证明:
\eq{
    d^3x\sqrt{-g}&=edx^0dx^1dx^2=\frac{1}{3!}e\epsilon_{\mu\nu\rho}dx^\mu\wedge dx^\nu\wedge dx^\rho\\&=\frac{1}{3!}\epsilon_{abc}e^a\wedge e^b\wedge e^c;
}
以及
\eq{
    \epsilon_{abc}e^a\wedge R^{bc} &=\frac{1}{2}e\epsilon_{\mu\alpha\beta}R_{\nu\rho}^{\alpha\beta}\epsilon^{\mu\nu\rho}d^3x\\
    &=d^{3}x\sqrt{-g}R.
}
\qed


在三维情况下我们有一个特殊的性质,就是任何一个二维的全反对称张量,我们可以使用一个一维的向量来进行描述。也就是dual notation。下面就是定义:
\eq{
    R_a\equiv\frac{1}{2}\epsilon_{abc}R^{bc}\leftrightarrow R^{ab}\equiv-\epsilon^{abc}R_c,\quad \omega_a\equiv\frac{1}{2}\epsilon_{abc}\omega^{bc}\leftrightarrow\omega^{ab}\equiv-\epsilon^{abc}\omega_c.
}
使用dual notation我们的EH action可以写成:
\eq{
    S_{\mathrm{EH}}[e,\omega]=\frac{1}{16\pi G}\int_{\mathcal{M}}\left(2e^a\wedge R_a[\omega]-\frac{\Lambda}{3}\epsilon_{abc}e^a\wedge e^b\wedge e^c\right).
}


\subsection{Chern-Simons Theory}
我们讨论一下Chern Simons theory到底是什么。
\imp{Chern-Simons Theory}{
    对于一个三维的流形上面我们可以赋予一个李群,并保证每一点赋予的李群都是一样的 G。这个群有个李代数称之为g。我们可以写出下面的action:
    \eq{
        S_{\mathrm{CS}}[A]=\frac{k}{4\pi}\int_{\mathcal{M}}\mathrm{Tr}\left[A\wedge dA+\frac{2}{3}A\wedge A\wedge A\right],
    }
    \itm{
        \pt{k 被称为这个理论的level}
        \pt{A 是一个1-form。但是这个1-form的系数$ A = A_\mu dx^\mu $中的$ A_\mu $是李代数g之中的一个元素。  }
        \pt{Tr 指的是这个李代数的一个合理的non-degenerate bilinear form结构。我们目前还没有定义,后面具体计算需要给出具体的定义。我们不妨作出这个假设,我们用李代数的基展开$ A_\mu = A^a_\mu T_a $。并且给李代数赋予一个Tr,$ d_{ab} = Tr(T_aT_b) $。那么这就相当于赋予李代数一个g的结构。当然这个可以任意赋予的。只有存在这样的结构的规范群,才能有CS理论  
        }
    }
    
}

这个理论的运动方程是:
    \eq{
        F\equiv dA+A\wedge A=0,
    }
    这个运动方程意味着:
    \eq{
        A = G^{-1} dG
    }
    也就是说这个$ A $其实是规范等价于这个李群(规范群)的单位元的。这就意味着这个理论其实并没有有意义的解,是一个topological的解。 

接下来我们给出一个重要的结论:
\imp{CS \& Einstein-Hilbert}{
    对于三维引力来说,我们的action对偶于一些有特殊的规范群的Chern-Simons理论:
    \itm{
        \pt{AdS:SO(2,2)}
        \pt{$ \Lambda = 0 $:ISO(2,1) }
        \pt{dS:SO(3,1)}
    }
}
下面我们仔细考虑AdS3的情况进行证明。首先我们需要讨论$ SO(2,2) $群是什么。
\imp{$ SO(2,2) $ Group }{
    这是一个有六个生成元(李代数元素)的李群。并且这六个生成元满足下面的对易关系:
    \eq{
        [J_a,J_b]=\epsilon_{abc}J^c,\quad [J_a,P_b]=\epsilon_{abc}P^c,\quad [P_a,P_b]=\epsilon_{abc}J^c,
    }
    其中$ a,b,c \in \{0,1,2\} $并且这些指标的升降我们可以用minkovski度规来定义。也就是说我们这个李群上的流形结构的度规赋予的是minkovski的。 

    也就是说他们的Tr的定义如下:
    \eq{
        (J_a,P_b)=\eta_{ab}\mathrm{~,~~~~~}(J_a,J_b)=0=(P_a,P_b).
    }
    并且三维的时候我们自然可以把二阶张量写成一阶的,我们会发现其实这个群就是三维的洛伦兹群:
    \eq{
        J_a\equiv\frac{1}{2}\epsilon_{abc}J^{bc}\leftrightarrow J^{ab}\equiv-\epsilon^{abc}J_c,
    }
    J是旋转,P是boost
} 
下面我们构造一个Gauge Field。通过我们之前定义的两个1-form:
\eq{
    A_\mu\equiv\frac{1}{\ell}e_\mu^aP_a+\omega_\mu^aJ_a.
}
我们根据这个定义我们可以写出一个SO(2,2)CS理论也就是:
\eq{
    \mathrm{Tr}[A\wedge dA]&\begin{aligned}&=(\frac{1}{\ell}e^aP_a+\omega^aJ_a,\frac{1}{\ell}de^bP_b+d\omega^bJ_b)\end{aligned}\\&=\frac{1}{\ell}\left(e^a\wedge d\omega^b+\omega^a\wedge de^b\right)\eta_{ab}=\frac{2}{\ell}e^a\wedge d\omega_a,
}
以及:
\eq{
    \frac{2}{3}\mathrm{Tr}[A\wedge A\wedge A]&=\frac{1}{3}\mathrm{Tr}[[A,A]\wedge A]\\&=\frac{1}{3\ell}\left(\frac{1}{\ell^2}e^a\wedge e^b\wedge e^c+3\epsilon_{abc}e^a\wedge\omega^b\wedge\omega^c\right).
}
整体上我们就得到了AdS3的Chern-Simons理论。
\eq{
    S_{\mathrm{CS}}[e,\omega]=\frac{k}{4\pi\ell}\int_{\mathcal{M}}\left.\left(2e^a\wedge R_a[\omega]+\frac{1}{3\ell^2}\epsilon_{abc}\right.e^a\wedge e^b\wedge e^c\right),
}
其中:
\eq{
    R_a=d\omega_a+\frac{1}{2}\epsilon_{abc}\omega^b\wedge\omega^c.
}
其level是:$ k = \frac{l}{4G} $ ; $ \Lambda = -\frac{1}{l^2} $ 我们需要知道在bulk考虑下似乎这个理论就是trivial的但是,考虑到边界条件,我们会发现我们有无穷维度的自由度在边界上!!并且边界条件其实并不是唯一的。边界上存在着对称性,我们称之为global symmetry或者asymptotic symmetry。正是boundary上的global symmetry应该求和导致了有着无穷的自由度。

接下来一个比较重要的事实,就是$ SO(2,2) $是一个半单李代数,也就是他可以写成单李代数的直和:
\eq{
    so(2,2)\boldsymbol{\approx}sl(2,\mathbb{R})\oplus sl(2,\mathbb{R})
} 
我们可以定义另一套生成元:
\eq{
    J_a^\pm\equiv\frac{1}{2}(J_a\pm P_a),
}
这些生成元满足下面的代数关系:
\eq{
    [J_{a}^{+},J_{b}^{+}]=\epsilon_{abc}J^{+c},\quad[J_{a}^{-},J_{b}^{-}]=\epsilon_{abc}J^{-c},\quad[J_{a}^{+},J_{b}^{-}]=0.
}
因此,我们可以构造两个$ SL(2,\mathbb{R}) $的Chern-Simons connection:
\eq{
    A=(e^a/\ell+\omega^a)T_a\mathrm{~,~~~~~~}\bar{A}=(e^a/\ell-\omega^a)T_a,
} 
我们的action和运动方程是如下的:
\pict{2025-03-10-22-57-32.png}{1}

最后,我们很显然有结论,就是经典的Chern-Simons理论给出了AdS3的作用量!!

\newpage
\section{基础的复几何的概念}
\subsection{黎曼曲面基础}
对于二维曲面,我们指的是一个二维的流形。我们通常认为这个流形有下面的说法:
\itm{
    \pt{1. 曲面是closed当我们的曲面没有边界}
    \pt{2. 曲面是finite type如果我们可以通过把close的曲面移除有限个点或者open disk得到的}
    \pt{3. 我们默认考虑有定向的曲面}
}
\defi{
    connected sum

    我们可以对于两个有定向的曲面定义connected sum $ S = S_1 \# S_2 $。我们才用下面的步骤:分别从两个曲面上选取一个close disk,数学表达就是:$ D_1  \subset S_1 $ 和 $ D_2 \subset S_2 $。
            \eq{
                \varphi_i:\left\{\left.(x,y)\in\mathbb{R}^2\right.:\left.x^2+y^2\leq1\right.\right\}\to D_i,\quad i=1,2,
            }
    然后我们可以定义:
    \eq{
        S_1\#S_2=\left(S_1\smallsetminus\mathring{D}_1\sqcup S_2\smallsetminus\mathring{D}_2\right)/\thicksim
    }
    这个模掉的等价类是:
    \eq{
        \varphi_1(x,y)\thicksim\varphi_2(x,y)\quad\text{for all }(x,y)\in\mathbb{R}^2\mathrm{~with~}x^2+y^2=1.
    }
    图像就是:
    \pict{2025-03-11-14-46-08.png}{0.7}
}
给出了粘贴的严格定义之后,我们会给出戏amain的一个定义,就是怎么分类所有close 2-surface:
\thm{
    二维close流形分类定理

    所有的2-d,close,有定向的流形都微分同胚于2-sphere和有限个torus的connected sum。
}
为此我们可以构建一些数字来character一个finite type 曲面(g,b,n):
\defi{
    Signature of surface

    对于一个二维曲面,我们可以用三个数来描述,被称为signature。
    \itm{
    \pt{g是connected sum之中使用的tori的数量,被称为genus}
    \pt{b是finite type流形移除的disk的数量,被称为number of boundary components}
    \pt{n是finite type流形上移除的点的个数,被称为punctures}
}
任何一个流形我们可以写成$ \Sigma_{g,b,n} $或者$ \Sigma_{g} = \Sigma_{g,0,0} $  
}
下面我们介绍一个二维流形上重要的拓扑不变量Euler characteristic。

我们的二维流形可以进行三角刨分,我们可以把它分成三部分: $ \mT = (V,E,F) $。
\itm{
    \pt{也就是在流形上选择有限的点$ V = \{v_1, \ldots ,v_k\} $}
    \pt{并且给出有限个边$ E = \{e_1, \ldots,e_l \} $保证连接两个点。}
    \pt{剩下的面$ F $,必须有三个edge作为边界。 }
}
\rmk{
    我们需要注意triangulation并不是代数拓扑里面的simplical complex的概念。比如,下面的图就不是一个simplical complex(具体看看代数拓扑的书)
    \pict{2025-03-11-20-12-32.png}{0.88}
}
我们根据triangulation可以为流形赋予一个数字,我们可以定义这个Euler Characteristic。
\defi{
    Euler characteristic

    对于一个close manifold我们给出的定义是:(注意我们的这个计算只能)
    \eq{
        \chi(S)=|V|-|E|+|F|.
    }
}
根据代数拓扑里面的singular homology,我们可以发现这个数是同伦不变的。为此我们可以给出下面的定理:
\thm{
    Euler Characteristic的具体数
    \itm{
        \pt{对于一个close的surface来说:$ \chi(S) = 2-2g $ }
        \pt{对于一个open的surface来说:$ \chi(\Sigma_{g,b,n})=2-2g-b-n. $ }
    }
}
注意,我们本身的triangulation并不能够在一个有puncture的曲面上定义。但是我们可以修订定义,认为puncture就是一个vertex,那么就可以给出上面的第二个式子。

\vspace{0.7em}
\textbf{黎曼曲面的概念}

下面我们定义一个很重要的概念,也就是黎曼曲面:
\imp{黎曼曲面}{
    Riemann surface X 是一个连续的Hausdorff的拓扑空间X。并且这个空间上面需要有两个结构:open cover $ \{U_\alpha\}_{\alpha \in A} $ ; maps $ \phi_\alpha : U_\alpha \to \mathbb{C} $这两个结构需要满足下面的关系:
    \itm{
        \pt{$ \phi_\alpha(U_\alpha) $ 是一个开集,并且$ \phi_\alpha $是一个homeo(也就是连续的双射) }
        \pt{对于$ \alpha , \beta \in A $ 我们对于$ U_\alpha \cap U_\beta \neq \emptyset $ 我们下面的映射是全纯的:
        \eq{
            \varphi_\alpha\circ(\varphi_\beta)^{-1}:\varphi_\beta(U_\alpha\cap U_\beta)\to\varphi_\alpha(U_\alpha\cap U_\beta)
        }
        }
    }  
    我们的$ ((U_\alpha, \phi_\alpha))_{\alpha\in A} $ 被称为atlas。
}
对于这个群面我们有两个性质,是可以通过定义证明出来的:
\itm{
    \pt{Riemann surface是second countable}
    \pt{Riemann surface是自动是有一个定向的}
}
下面我们给出一些黎曼曲面的例子,$ \mathbb{C} $是最经典的例子,而如果我们算上无穷远点我们称之为projective line $ \mathbb{P}^1 $或者有的时候我们使用$ \hat{\mathbb{C}} $进行描述, 也是,被称为Riemann sphere。所有Riemann Sphere的连通开集都是黎曼曲面。
\defi{
    Domain

    也就是Riemann Sphere的连通开集!!所有的domain都是黎曼曲面,从$ \mathbb{P}^1 $里面继承了黎曼曲面的结构!!! 
}

\vspace{0.7em}
\textbf{黎曼曲面的自同态}

首先我们定义什么是全纯的holomorphic的映射:
\defi{
    holomorphic map

    首先我们与两个黎曼曲面,$ X= \{(U_\alpha,\varphi_\alpha)\}_{\alpha\in A} $ ,$ Y= \{(V_\beta,\psi_\beta)\}_{\beta\in B} $ 。我们有一个两个曲面之间的映射:$ f:X \to Y $。 如果是全纯的需要满足下面的函数是一个全纯函数:
    \eq{
        \psi_\beta\circ f\circ\varphi_\alpha^{-1}:\varphi_\alpha(U_\alpha\cap f^{-1}(V_\beta))\to\psi_\beta(f(U_\alpha)\cap V_\beta)
    }
    \itm{
        \pt{如果f是一个bijection,那么我们称之为是一个biholomorphism或者conformal}
        \pt{$ \text{AuT}(X) $表示X的自同态群,也就是所有的biholo构成的群 }
    }
}
一个重要的例子就是Riemann Sphere这个面的autholo群是:
\eq{
    \mathrm{Aut}(\mathbb{P}^1(\mathbb{C}))=\mathrm{PGL}(2,\mathbb{C})=\mathrm{GL}(2,\mathbb{C})/\left\{\begin{pmatrix}\lambda&0\\0&\lambda\end{pmatrix}:\lambda\neq0\right\}.
}
同时这个群也同构于:
\eq{
    \mathrm{PGL}(2,\mathbb{C})\simeq\mathrm{PSL}(2,\mathbb{C})=\left\{\left.\begin{pmatrix}a&b\\c&d\end{pmatrix}\right.:a,b,c,d\in\mathbb{C},\quad ad-bc=1\right\}/\left\{\pm\begin{pmatrix}1&0\\0&1\end{pmatrix}\right\}
}

\vspace{0.7em}
\textbf{黎曼群面的Quotient}

\imp{Quotient of Riemann Surface}{
    \pict{2025-03-12-12-06-03.png}{1}
    这个定理告诉我们,如果一个Domain进行quotient。只有对freely,properly discontinuously的子群的quotient才是一个Riemann Surface。
}    
这个定理告诉我们什么呢?就是给定一个domain,我们可以通过quotient这个domain得到一个黎曼曲面。一个很重要的例子就是构造Torus。

对于$ \mathbb{C} $作为Riemann Sphere的一个Domain。我们可以模去下面的矩阵generate的群:
\eq{
    g_1:=\left[{\begin{array}{cc}1&1\\0&1\end{array}}\right],g_\tau:=\left[{\begin{array}{cc}1&\tau\\0&1\end{array}}\right]\in\mathrm{PSL}(2,\mathbb{C}),
}
并且我们定义这个群作用在二维平面上面是这样定义的:
\eq{
    g_1(z)=z+1\quad\mathrm{and}\quad g_\tau(z)=z+\tau
}
我们可以证明这个群满足上面很好的性质所以,Quotient的结果依旧是一个黎曼曲面。这个黎曼曲面我们记作: $ \mathbb{C}/\Lambda_\tau $ 。然后我们会发现这个流形其实就是一个torus。我们可以构建下面的homeomorphism:$ \mathbb{C}/\Lambda_\tau\to\mathbb{S}^1\times\mathbb{S}^1 $ 
\eq{
    \large[x+y\tau]\mapsto(e^{2\pi ix},e^{2\pi iy})
}

我们还可以构建Hyperbolic Surface的Quotient出来的黎曼流形:

\defi{
    Hyperbolic Surface
    \eq{
        \mathbb{H}^2=\{z\in\mathbb{C}:\mathrm{Im}(z)>0\}
    }
    也就是上半复平面。这个平面的$ AuT(\mathbb{H}^2)  = PSL(2,\mathbb{R})$。对于这个群我们也可以构建出一堆quotient出来的子黎曼曲面。 
}
\rmk{
    $ PSL(2,\mathbb{R}) $的群,其实是保下面的这个度规的群:
    \eq{
        ds^2=\frac{dx^2+dy^2}{y^2}.
    } 
    这个度规其实是曲率为-1的时空的度规。
}
下面我们给出一个特别特别强的定理:
\imp{Uniformization Theorem}{
X是一个simply connected Riemann surface。那么X必然biholo于下面的几个黎曼曲面之一:
\eq{
    \widehat{\mathbb{C}},\quad\mathbb{C}\quad or\quad\mathbb{H}^2.
}
这意味着这三个黎曼曲面我们选任意一个进行quotient可以得到所有的黎曼曲面!!!也就是说:
\pict{2025-03-12-13-26-47.png}{1}
}
为了一定程度说明这个定理我们需要定义下面的概念就是一个拓扑空间的Universal Covering。
\pict{2025-03-12-13-39-17.png}{1}

\imp{解读上面的Uniformization Thm}{
    \textbf{对于一个黎曼曲面来说,我们有定理保证可以构造一个Universal Covering,并且这个covering是一个simply connected的Riemann surface。}

    所以根据UT,我们知道,任意黎曼曲面的Universal covering必然biholo于下面三个黎曼曲面之一:
    \eq{
        \widehat{\mathbb{C}},\quad\mathbb{C}\quad or\quad\mathbb{H}^2.
    }
    也正因此,所有的黎曼曲面都可以通过被Universal covering 进行quotient。也就是被那三种黎曼曲面quotient得到。并且我们的Universal Covering的构造保证:
    \eq{
        \widetilde{X}/\pi_1(X)=X.
    }
    我们的quotient选择的群就是这个黎曼曲面的基本群。所以我们,只要说这个黎曼曲面的基本群是某一个Domain的aut群的free,properly discontinuously的子群那么就说明这个曲面是哪一个Domain Quotient出来的。
}

那么为了研究怎么进行Quotient,下面是这三个流形的aut群!
\itm{
    \pt{$ \mathrm{Aut}(\widehat{\mathbb{C}})=\mathrm{PSL}(2,\mathbb{C}) $ }
    \pt{$ \mathrm{Aut}(\mathbb{C})=\{\varphi:z\mapsto az+b:a\in\mathbb{C}\setminus\{0\},b\in\mathbb{C}\}\simeq\mathbb{C}\rtimes\mathbb{C}^*, $ }
    \pt{$ \mathrm{Aut}(\mathbb{H}^2)=\mathrm{PSL}(2,\mathbb{R}) $ }
}

根据这个定理我们可以把所有黎曼曲面分成三类,那么现在,我们试试能不能再仔细的分类每一类里面的黎曼曲面。

\imp{三个基本的Riemann Surface能quotient出个啥?}{
    \itm{
    \pt{如果一个黎曼曲面X的Universal Covering biholo于$\widehat{\mathbb{C}}$,那么这个黎曼曲面自己biholo于$\widehat{\mathbb{C}}$ }
    \pt{如果一个黎曼曲面X的Universal Covering biholo于$ \mathbb{C} $,那么这个黎曼曲面自己biholo于下面三种情况之一:
    \eq{
        \mathbb{C}, \quad \mathbb{C}/\{0\}, \quad \mathbb{C}/\left\langle\begin{bmatrix}1&\lambda\\0&1\end{bmatrix},\begin{bmatrix}1&\mu\\0&1\end{bmatrix}\right\rangle
    }
    }
    \pt{如果一个黎曼曲面X的Universal Covering biholo于$ \mathbb{H}^2 $,那么情况就会十分复杂,我们需要仔细研究$ Aut(\mathbb{H}^2) $的FP子群的结构从而得到,能够被quotient出来的黎曼曲面 }
    }
}
为了研究什么黎曼曲面可以被hyperbolic surface quotient出来,我们下面讨论$ Aut(\mathbb{H}^2) $ 的子群结构。
\thm{
    $ Aut(\mathbb{H}^2) $ 的特殊子群

    如果一个群$ G \subset PSL(2,\mathbb{R}) $并且act properly on $ \mathbb{H}^2 $并且是Abelian的。那么这个群只有两种情况:
    \itm{
        \pt{$ G \sim \mathbb{Z} $ }
        \pt{G是一个有限群,并且rank 1}
    }
}



\subsection{黎曼曲面的流形结构}
\subsubsection{黎曼曲面的曲率结构}

由于黎曼曲面是一个流形,其实其上有一个很自然的metric结构的。对于$ P^1 $来说就是恒定的 1 curvature的结构;对于$ \mathbb{C} $来说就是 0 curvature;对于$ \mathbb{H} $来说就是 -1 curvature。  我们其实会发现,黎曼曲面的结构,会自动赋予所有的黎曼曲面一个等curvature的度规。我们可以通过一个定理看出来:
\thm{
    Killing-Hopf 定理

    任意一个有常曲率属于$ \{1,0,-1\} $ 的黎曼曲面可以通过quotient一个有黎曼曲面结构的广义黎曼流形的orientation preserving isometry group的一个FP子群得到,并且quotient自下面三个有黎曼曲面结构的二维广义黎曼流形:
    \eq{
        S^2 \text{with round metric} \quad \mathbb{R} \text{with Euclidean metric} \quad \mathbb{H}^2 \text{with hyperbolic metric} 
    }
}
这个定理出发我们其实可以还原上面三种基本的Rienmann surface能quotient出的东西。我们先定义有黎曼曲面结构的广义黎曼流形的orientation preserving isometry group:
\eq{
    \mathrm{Isom}^+(M)=\left\{\varphi:M\to M:\varphi\text{ is an orientation preserving isometry }\right\}.
}
对于上面定理提到的三种流形来说:
\itm{
    \pt{$ \mathrm{Isom}^+(\mathbb{S}^2)=\mathrm{SO}(2,\mathbb{R}) $这个群的FP子群只有trivial的,所以quotient出来的黎曼曲面都同构于其自己 }
    \pt{$ \mathrm{Isom}^+(\mathbb{R}^2)=\mathrm{SO}(2,\mathbb{R})\ltimes\mathbb{R}^2 $这个群的FP子群,有三种,一个是trivial的,另外两个正好是圆柱($ \mathbb{C}/{0} $ )和torus上面的基本群,所以quotient出来就是我们之前讨论的三种黎曼曲面 }
    \pt{$ \mathrm{Isom}^+(\mathbb{H}^2)=\mathrm{PSL}(2,\mathbb{R}). $我们上面已经讨论了,这个的FP子群巨多无比,所以就quotient出来各种各样子的曲面 }
}
所以我们会发现,这些quotient出来的黎曼曲面都有一个\{0,1,-1\}的曲率结构。

\imp{所有黎曼曲面的度规结构}{
    根据上面的定理我们其实可以构造一个一一映射,就是从所有没有边界的黎曼曲面的空间(quotient掉biholo)到曲率为\{0,-1,1\}的二维广义黎曼流形(quotient掉isometry):
    % \eq{
    %    \{ \text{所有的黎曼曲面on } \Sigma\}/ \sim  \longleftrightarrow \{\text{所有的二维广义里面流行保证metric是 \{0,-1,1\} on } \Sigma \}/ \sim
    % }
    \pict{2025-03-12-20-58-28.png}{1}
    也就是说所有的黎曼曲面都可以赋予一个恒定曲率的度规结构!!!!
}

\rmk{
    这一定程度的意味着一件事情就是有两个东西是一模一样的:
    \itm{
        \pt{黎曼曲面和\{0,-1,1\}度规的二维广义黎曼流形是对应的}
        \pt{其上的biholo一一对应于广义黎曼流形的isometry}
    }
}
并且对于黎曼曲面来说,欧拉指标和曲率满足关系:
\eq{
    \kappa\cdot\mathrm{area}(X)=2\pi\mathrm{~}\chi(X)
}
很自然的面积我们也可以求,通过欧拉指标,对于$ \Sigma_{g,b,n} $来说 :
\eq{
    \mathrm{area}(X)=2\pi(2g+n+b-2).
}

\subsubsection{黎曼曲面的共形结构}

我们任意黎曼曲面都可以赋予一个常曲率的度规结构。同样的,我们的任意黎曼面其实还有一个共形结构。首先我们定义什么是共性等价的。
\defi{
    conformally equivalent

    我们说同一个流形的两个度规是共形等价的,其实是说,对于其流形上存在一个映射:$ \rho: X \to \mathbb{R}_+ $使得两个度规之间满足:
    \eq{
        ds_1^2=\rho\cdot ds_2^2.
    }
    满足这个等价关系的等价类我们称之为共形等价类。
}
下面我们可以给出一个类似的定理(我也懒得做出说明了呃呃呃呃呃)
\pict{2025-03-12-18-33-29.png}{1}

\subsection{Torus上面的Teichmuller Space}
下面我们通过对黎曼曲面的理解引申出两个衍生的空间:Moduli Space和Teichmuller Space。首先我们可以通过一些具体例子入手进行研究。由于Riemann Sphere上面的Moduli Space和Teichmuller Space都是Trivial的。所以我们不妨先研究一个更有意思一点的,就是Torus。Torus的定义就是$ \mathbb{C} $ quotient去两个变量给出的矩阵出来的黎曼曲面。 

Torus可以如下构造biholo等价类。$ R_\tau:=\mathbb{C}/\Lambda_\tau, $ 其中$ \tau \in \mathbb{H}^2 $并且我们有:
\eq{
    \Lambda_\tau=\left\langle\begin{bmatrix}1&1\\0&1\end{bmatrix},\begin{bmatrix}1&\tau\\0&1\end{bmatrix}\right\rangle.
} 
所有的torus可以biholo于用$ \tau $标记的。但是注意\textbf{同样的一个torus可以biholo于不止一个$ \tau $标记的torus }所以我们应该研究什么样子的$ \tau $才是真正的对应的torus的biholo等价类!!  
\thm{
    Torus的Biholo等价类

    对于两族Torus $ R_\tau $和$ R_{\tau'} $他们之间是biholo当且仅当:
    \eq{
        \large\tau^{\prime}=\frac{a\tau+b}{c\tau+d}
    }  
    其中满足:$ a,b,c,d\in\mathbb{Z}\mathrm{~with~}ad-bc=1. $ 
}
这样看上去我们所有Torus的biholo等价类都可以用下面这个空间进行标定:
\eq{
    \mathcal{M}_1=\mathbb{H}^2 / \mathrm{SL}(2,\mathbb{Z})=\mathbb{H}^2/\mathrm{PSL}(2,\mathbb{Z})
}
有一个P是因为如果把四个整数都进行变号其实结果是不变的。首先讨论这个quotient后的面是不是黎曼曲面,由于我们quotient的群$ PSL(2,\mathbb{Z}) $其实并不是freely的,所以quotient之后的结果并不是一个黎曼曲面。但是我们可以讨论就是这个quotient的\textbf{fundamental domain}。也就是说,对于所有的$ \tau \in \mathbb{H}^2 $,存在$ g \in PSL(2,\mathbb{Z}) $使得$ g \tau \in \mF $    我们可以用下面的公式和图片表示:
\eq{
    \mathcal{F}=\begin{Bmatrix}z\in\mathbb{H}^2:|z|\geq1\mathrm{~and~}-\frac{1}{2}\leq\mathrm{Re}(z)\leq\frac{1}{2}\end{Bmatrix}.
}
在上半平面里面画出来就是:
\pict{2025-03-12-19-15-32.png}{1}
下面我们仔细分析这个灰色的区域每一个部分作用上$ p \in PSL(2,\mathbb{Z}) $之后的结果! 分析之后我们会发现,我们有基本的操作,就是左右平移1;或者把中间的圆里面的点map到圆外面的!!!所以灰色的空间其实左右的边界是站起来的下面的圆左右其实也是“粘起来的”。我们把这些边界粘起来其实这个流形应该长这个样子的;
\pict{2025-03-12-19-25-43.png}{0.8}
这个也就是$ M_1 $的形状!!! 这个形状的名字其实是hyperbolic orbifold(像是两个流形但是把边界完全粘贴在一起一样所以我们叫orbifold)。下面我们定义torus上面的两个衍生的空间:
\itm{
    \pt{Moduli Space:也就是$ M_1 = \mathbb{H}^2/ PSL(2,\mathbb{Z}) $ }
    \pt{Teichmuller Space:也就是$ T_1 = \mathbb{H}^2 $ }
}
下面我们试图给出更加general的定义。下面我们会发现有两种定义Torus上面的Teichmuller空间的方法。一个是从定义marking和marking等价的角度;另一个是从定义微分同胚的角度。

\subsubsection{从Marking的角度定义Teichmuller 空间}
我们定义什么是一个Torus的marking。也就是Torus的某一个点上面的基本群的两个generator。我们记作:
\eq{
    [A_\tau],[B_\tau]\in\pi_1(R_\tau,p_0).
}
我们发现可以给一个Torus上面赋予一个自然的Marking结构(也就是1和$ \tau $这两条线 )(就是选取一个特殊的点,给出这个点上面的基本群的生成元):
\pict{2025-03-12-20-47-21.png}{0.9}
所以我们可以定义Teichmuller空间:
\defi{
    从Marking角度定义Teichmuller space

    我们认为两个有marking结构的Torus是equivalent的。如果存在一个biholo $ f : R \to R' $使得这个biholo induce出来的自然marking结构是同构的:
    \eq{
        h_*(\Sigma)\simeq\Sigma^{\prime}.
    }
    我们定义
    \pict{2025-03-12-20-55-37.png}{1}
}
\rmk{
    这个定义的方式的本质其实就是选择一种方式,让我们构造Riemann曲面的等价类的时候,并不考虑不同$ \tau $之间的同构的关系。通过引入一种特殊的数学结构,保证不同$ \tau $的选择之中没有equivalent了!!!(也就是不能再用biholo了!!)  
}

\subsubsection{从微分同胚的角度定义Teichmuller 空间}
首先我们选用一个固定的唯一的(但是比较任取的)surface $ S $ 并且保证这个曲面微分同胚于$ \mathbb{T}^2 $。也就是选择一个universal的独特的torus。下面我们定义,赋予“微分同胚”结构的等价:

\defi{
    微分同胚结构等价关系

    我们令$ R $和$ R' $是两个torus。这个时候定义一个orientation preserving diffeomorphism
    \eq{
        f:S\to R\quad\mathrm{and}\quad f^\prime:S\to R^\prime
    }  
    这样我们就可以赋予一个黎曼曲面一个微分同胚结构,我们记作$ (R,f) $和$ (R',f') $。
    
    如果这两个黎曼曲面等价,就是说存在一个biholo $ h:R \to R' $保证:
    \eq{
        (f^{\prime})^{-1}\circ h\circ f:S\to S
    } 
    这个函数是homotopy于identity的(而不是基本群里面其他的,像是转一圈回去那种。)
}
下面,我们给出这个微分同胚等价和赋予marking结构等价之间的关系。如果我们赋予$ S $这个universal的torus一个marking的结构,我们可以证明我们的$ f $微分同胚可以赋予流形$ (R,f) $一个marking的结构。这个样子,赋予微分同胚结构和赋予marking结构其实是可以一一映射的。并且marking等价和微分同胚等价是一毛一样的。
\pict{2025-03-12-21-20-22.png}{1}

\subsection{Teichmuller theory的基本定义}
我们最后一般选择第二种方法定义Teichmuller Space。我们严格的写下来:
\imp{定义Teichmuller Space}{
    \pict{2025-03-12-21-28-51.png}{1}
}
当然对于任意的finite type的黎曼曲面我们其实也可以找到一个等价的第一种定义。上面我们仅仅是对于finite type的曲面进行定义。我们下面把这个定义推广到有mark point 的puncture的情况。
\imp{定义有标记点的Teichmuller Space}{
    \pict{2025-03-12-21-47-28.png}{1}
}

\rmk{
    我想起来我们在画conformal block的时候我们一般使用流形上面embed一堆线。在想这些线其实可以理解成为Teichmuller空间上给出来的微分同胚$ f:S \to R $的结构。因为我们关心的是这个微分同胚在基本群里面的地位,所以我们只需要画一条线就好。 
}

\que{关于braiding和mapping class group的关系}{
    似乎mapping class group只给出了一个dehn twist和F move的generation但是并没有给出braiding的产出呀!!!!
}
下面我们给出一个Teichmuller theory里面的一个很重要的基本概念。也就是mapping class group。我们下面给出这个群的定义。
\imp{Mapping Class Group}{
    对于一个compact finite type的曲面$ S_0 $我们有其中的一个有限集合$ \Sigma \in S_0 $。我们可以通过进行quotient给出一个更小一点的空间$ S $。我们下面定义 $ S $上面的Mapping class group:
    \eq{
        \mathrm{MCG}(S)=\mathrm{Diff}^+(S,\partial S,\Sigma)/\mathrm{Diff}_0^+(S,\partial S,\Sigma)
    }    
    其中我们涉及两个微分同胚群,我们如下定义。
\pict{2025-03-13-12-48-48.png}{1}
注意,我们的Diff不仅仅是orientation preserving的并且还需要是保证映射先后$ \Sigma $上面的点必须映射回自己的。 
}
根据这个定义我们可以通过我们已经构造的Teichmuller 空间给出moduli space的定义。
\defi{
    Moduli Space

    我们定义这个空间就是Teichmuller空间对于mapping class group取一个模!
    \eq{
        \mathcal{M}(S)=\mathcal{T}(S)/\operatorname{MCG}(S).
    }
    一般是用这个记号:
    \eq{
        \mathcal{M}(\Sigma_{g,n})=\mathcal{M}_{g,n}\quad\mathrm{~and~}\quad\mathcal{M}(\Sigma_g)=\mathcal{M}_g.
    }
}
给出了定义之后我们下面给出一些黎曼曲面给出的Teichmuller空间,mapping class group还有moduli空间的例子。



\subsection{Teichmuller theory}
我希望继续讲一讲Teichmuller理论是什么?


\newpage
\section{Mathematical Quantization}
有一个比较清楚的给物理人的文章可以fol一下











\newpage
\section{Quantization of Bosonic String}
我们相当于在一个二维的曲面上面进行路径积分从而量子化一个弦理论。Naively我们的量子化可以写成:
\eq{
    \int[dX\mathrm{~dg}]\mathrm{~exp}(-S)\mathrm{~.}
}
其中我们的$ S $可以写成:
\eq{
    S=S_X+\lambda\chi_1,
}
具体地说就是:
\eq{
    S_X & =\frac{1}{4\pi\alpha^{\prime}}\int_Md^2\sigma\mathbf{g}^{1/2}\mathbf{g}^{ab}\partial_aX^\mu\partial_bX_\mu,\\
    \lambda_\chi &=\frac{1}{4\pi}\int_Md^2\sigma\mathrm{~g}^{1/2}R+\frac{1}{2\pi}\int_{\partial M}dsk.
}
但是我们并不可以这样进行路径积分因为这会导致overcounting,因为我们的度规和场在Diff和Weyl transformation之下其实是不变的,我们很可能把同样的一个场算了两次。真正的配分函数是:
\eq{
    \int\frac{[dX\mathrm{~}d\mathrm{g}]}{V_{\mathrm{diff}\times\mathrm{Weyl}}}\exp(-S)\equiv Z.
}
我们首先一个问题是我们的$ g_{\mu\nu} $其实并没有任何的自由度。在二维的时候,他只有三个分量而如果确定一个gauge的话,那么刚好就是三个约束方程(两个空间坐标变换的,也就是diff;一个Weyl transformation的)我们希望fix我们的metric在一个fiducial metric不妨取下面的两种gauge之一:
\itm{
    \pt{如果考虑三种变换,可以使用unit gauge:$ \hat{g}_{ab}(\sigma)=\delta_{ab}\mathrm{~.} $ }
    \pt{如果仅仅考虑diffeo,那么就用conformal gauge:$ \hat{g}_{ab}(\sigma)=\exp[2\omega(\sigma)]\delta_{ab}. $ }
}
我们使用FP量子化的技巧可以给出结论:
\eq{
    Z\left[\hat{g}\right]=\int[dX]\Delta_{\mathrm{FP}}(\hat{g})\exp(-S[X,\hat{g}]).
}
其中我们通过det和配分函数的关系这个FP trick可以得到:
\eq{
    \Delta_{\mathrm{FP}}(\hat{g})=\int[dbdc]\mathrm{~exp}(-S_{\mathrm{g}})\mathrm{~,}
}
其中:
\eq{
    S_\mathrm{g}=\frac{1}{2\pi}\int d^2\sigma\hat{g}^{1/2}b_{ab}\hat{\nabla}^ac^b=\frac{1}{2\pi}\int d^2\sigma\hat{g}^{1/2}b_{ab}(\hat{P}_1c)^{ab}.
}
也就是相当于需要在理论里面加入一个bc ghost的配分函数才能合理的在二维平面上进行量子化!!并不考虑弦是一个曲面的问题。










\newpage
\section{速通Liouville theory}
普通的liouville theory我们称之为spacelike liouville theory。我们下面进行定义。
\subsection{Spacelike Liouville}

这是一个理论,对于这个理论有下面的要求:
\imp{Liouville theory定义}{
    首先说明convention,我们一般这样子parametrize:
    \eq{
        c=1+6Q^2,\quad Q=b+b^{-1},\quad\Delta=\alpha(Q-\alpha),\quad\alpha=\frac{Q}{2}+iP.
    }
    表示c,可以使用Q和b;表示$ \Delta $可以使用$ \alpha $和P。 这个理论有下面的约束:
    \itm{
        \pt{$ c >1 $,但是一般我们考虑$ c \geq 25 $ 的情况 }
        \pt{$ \Delta \geq \frac{Q^2}{4}  = \frac{c-1}{24}$也就是说$ P\in \mathbb{R} $  }
    }
}

对于这个样子的一个理论我们可以进行Bootstrap出所有的三点函数和两点函数:
\imp{3/2-point functions}{
    对于三点函数我们有:
    \eq{
        \langle V_{P_1}(0)V_{P_2}(1)V_{P_3}(\infty)\rangle=C_b(P_1,P_2,P_3)\equiv\frac{\Gamma_b(2Q)\Gamma_b(\frac{Q}{2}\pm iP_1\pm iP_2\pm iP_3)}{\sqrt2\Gamma_b(Q)^3\prod_{k=1}^3\Gamma_b(Q\pm2iP_k)}.
    }
    从这个公式中我们会发现,把P反号方程并不会改变,所以我们认为:
    \eq{
        V_p = V_{-P}
    }
    对于两点函数我们就是让第三个场$ \Delta_3 = 0 $,也就是说$ P_3 = \frac{iQ}{2} $  有:
    \eq{
        \langle V_{P_1}(0)V_{P_2}(1)\rangle=C_b(P_1,P_2,1)=\frac{1}{\rho_0^{(b)}(P_1)}(\delta(P_1-P_2)+\delta(P_1+P_2)).
    }
    其中:
    \eq{
        \rho_0^{(b)}(P)=4\sqrt{2}\sinh(2\pi bP)\sinh(2\pi b^{-1}P)\mathrm{~.}
    }
}
这些解都是很universal的数在2D CFT之中。因为他们其实是2D CFT的一组crossing kernel!!!!
\itm{
    \pt{$ \rho_0^{(b)} $其实是modular crossing kernel for torus vaccum character }
    \pt{$ C_b $其实是4点函数的crossing kernel之中的一个!!特殊的crossing kernel }
}
\pict{2025-03-07-15-29-05.png}{0.7}

那么我们接下来可以发现用这些数据可以算出liouville CFT的所有correlation function:
\imp{关联函数}{
    对于任意黎曼面上面的关联函数我们可以写成:
    \eq{
        \langle V_{P_1}\cdots V_{P_n}\rangle_g=\int_{\mathbb{R}_{\geq0}}\left(\prod_a\mathrm{d}P_a\rho_0^{(b)}(P_a)\right)\left(\prod_{(j,k,l)}C_b(P_j,P_k,P_l)\right)|\mathcal{F}_{g,n}^{(b)}(\mathbf{P}^{\mathrm{ext}};\mathbf{P}|\mathbf{m})|^2.
    }
    其中:
    \itm{
        \pt{$ \mathcal{F}_{g,n}^{(b)}(\mathbf{P}^{\mathrm{ext}};\mathbf{P}|\mathbf{m}) $是有$  \mathbf{P}^{\mathrm{ext}}$ 个external point;以及$ 3g-3+n $ 个internal point $ \mathbf{P} $ 的 }
    }
}


