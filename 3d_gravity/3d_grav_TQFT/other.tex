这里我们需要仔细阅读一下这个文章的重要章节。可以学习两个东西:
\itm{
    \pt{为什么AdS3的相空间是两个Teichmuller空间}
    \pt{怎么在基础的这两个Teichmuller空间上进行量子化。}
}

\section{理论总体陈述}
我们会发现量子化AdS3我们可以使用一种"constrain first"的思路进行操作。也就是我们首先找到一个non-dynamical equation的解空间,然后把这个量子化成Hilbert空间。

\imp{两种量子化的思路}{
    对于量子化我们一般有两种思路:
    \itm{
        \pt{directly quantize first}
        \pt{constrain first quantization}
    }
    之前Verlinde已经通过第一种方式考虑Chern-Simons理论完成了对于AdS3的量子化,并且认为波函数就是conformal block。那么这篇文章我们主要想说明的是,第二种思路依旧可以得出一样的结论。

    \vspace{0.7em}
    对于AdS3来说,我们这个"constrain first"的操作结果就是,找到2nd order Einstein Gravity的解,或者1st order的CS理论的约束方程量子化;这样的量子化结果和对于某个Chern-Simons理论进行quantize first的量子化的结果是一样的。
}

\section{经典理论的复习}
\section{CS formulism}
