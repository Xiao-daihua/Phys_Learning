这里我想主要fol Verlinde的文章捏!!我希望给出的思路就是量子化Teichmuller空间然后把量子化之后的结果和CFT的理论对应上!!

\subsection{如何量子化Teichmuller空间}
Teichmuller空间一个定义可以理解为一个度规的空间:
\eq{
    \frac{\text{metrics on }\Sigma}{\mathrm{Weyl}(\Sigma)\times\mathrm{Diff}_0(\Sigma)}
}
也就是所有的可能的度规矩阵,但是模掉identical sector的线性变换和weyl transformation。对于度规来说,我们可以使用zweibein的公理体系来进行描述:
\eq{
    \mathrm{d}s^2=e^+\otimes e^-.
}
这里我们使用Lightcone coordinate。对于正常的zweibein就是$ e^+ = e^0+e^1 $ 以及$ e^+ = e^0-e^1 $ 。对于一般的这样的理论我们可以有一个$ U(1) $的规范场$ \omega $ (因为二维的洛伦兹变换其实就是U(1)的,我们使用这个是因为我们希望有一个方法求曲率)。我们可以使用cartan,发现我们的理论满足这样子的关系:
\eq{
    \begin{array}{c}\mathcal{D}^{+}e^{-}\equiv\mathsf{d}e^{-}+\omega\wedge e^{-}=0,\\\\\mathcal{D}^{-}e^{+}\equiv\mathsf{d}e^{+}-\omega\wedge e^{+}=0.\end{array}
}
然后其中curvature的形式是:
\eq{
    R\equiv\operatorname{d}\omega=\Lambda e^+\wedge e^-,
}
这个式子让我们意识到如果曲率R是一个固定的量,其实我们确定的是一个weyl transformation。下面我们就可以通过zweibein的理论定义一个teichmuller空间:
\imp{Teichmuller space from zweibein}{
    空间可以定义为:
    \eq{
        \mathscr{T}=\mathscr{E}_{\mathrm{cc}}/\mathscr{G},\quad\mathscr{S}=\mathrm{Diff}_0\times\mathrm{LL},
    }    
    其中$ \mathscr{E}_{\mathrm{cc}} $是curvature恒定的zweibein;$ \mathrm{Diff}_0 $  是小的identity component of 2D diffeo group;$ \mathrm{LL} $ local lorentz transformation,也就是消除,选择一个zweibein的自由度!!
}
定义了空间之后我们需要找到一个symplectic form on $ \mathscr{T} $ 

