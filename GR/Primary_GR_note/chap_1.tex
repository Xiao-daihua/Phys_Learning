这部分是我最初学习几何的笔记,很多个人见解也不是特别清晰的呀呀呀!

\imp{广义相对论总论}{
  我们希望构建一个理论,这个理论与坐标变换没有任何关系。包括我们取一个加速的坐标系,我们的理论也不会变换。这样的理论我们之前的理论并没有做到过。因为,我们之前所有的理论并不对任意坐标变换都适用。
  \itm{
    \pt{牛顿力学来说,我们所有的物理量使用的是$ SO(3)\times \text{三维平移群} $的表示,也就是矢量。【但是实际上这些都很有问题,因为并不是所有我们牛顿力学谈及的矢量都是群表示,可能是更奇怪的数学结构】所以,我们的理论仅仅对于这个群给出的微分同胚不变。但是对于其他微分同胚会失效。

比如加速运动的时候需要手动加入惯性力。

    同时,我们会发现一些物理量完全没法用这个表示描述,比如电磁场,他根本不在这个表示空间里面,所以电磁学在牛顿力学体系下完全不能描述。}
    \pt{狭义相对论来说,我们的理论对于poncare群有着协变。但是,一旦涉及加速运动,那么理论依旧不能解释。}
  }
  所以我们的目的就是构建一个理论,这个理论与任意流形上的微分同胚没有关系。这样的理论对于所有的坐标变换,不论是不是加速的都是协变的,那么这就符合我们对于世界上所有物体的认知。这样的理论就是广义相对论。

  这样的理论可以用来描述引力,我们称为广义相对论。
}

25年10月的liuy回来看感觉很多内容都是不是特别妥当的!只是曾经阶段自己的比较简单的理解而已。

\section{流形上的张量场} 
\imp{张量是什么?}{
    张量的定义是:\textbf{协变的量!}

    本质是,对于任意的微分同胚(或者说坐标变换)这个量不会发生任何变换。

    而广义相对论里面我们要探讨的是最一般的张量。就是对于任意微分同胚,都不会发生变换的量。
}
\subsection{流形与流形上面的坐标}

流形的构造和具体的定义并不重要,我们需要知道的比较重要的是流形上面需要有这样的几个基本结构:
\itm{
    \pt{流形上面可以构建\textbf{坐标系}$ \mathbb{R}^n \to M $ ,坐标系的维度和流形的维度一样}
    \pt{流形之间存在一一映射并且来回都是$ C^\infty $的我们称之为\textbf{微分同胚}如果能找到两个流形之间存在这样的映射那么就认为这两个流形互为微分同胚。 }
    \pt{流形上自然的可以定义\textbf{标量场}其实就是$ M \to \mathbb{R} $ 的映射。}
}

\subsection{固定一个点上的特定张量}
目前我的一个基本的理解就是:广相之中定义的张量其实就是一个在流形上面某一个点定义的量,一切讨论都不包括“自变量”或者“场函数”,这个量真的就是一个矢量空间的量,而并不是一个变量或者函数!

但问题是,我们研究的广义相对论定义还是存在矢量场的,什么样的矢量放在一起组成的矢量场是什么样子的还是需要讨论的,但是这个讨论就不在这个subsection的范畴里面。

\textbf{这里我们需要澄清很多很重要的概念!也就是什么样子的量是张量,或者说,我们希望构建出来满足什么特质的量:}
~\\

1. \textbf{随着坐标系变化不变,我们限制在张量是写在一个tensor product构成的坐标系之中。}而不是坐标基矢量神奇组合(例如:wedge product的基矢量$(e^{\mu_1...\mu_n})_{a_1...a_n} = (d x^{\mu_1})_{a_1} \wedge ...\wedge (d x^{\mu_n})_{a_n}$,这种情况下只有全反对称张量能够分量不发生变化,其他的张量写在这个基之下分量会和一般张量积的分量不一样,所以不好讨论。)
也就是说,只有写成
\begin{equation}
    T = T^{\mu_1...\mu_n}{}_{\nu_1...\nu_n} (dx^{\nu_1})...(dx^{\nu_n}) (\partial_{\mu_1})...(\partial_{\mu_n})
\end{equation}

这样的形式我们才能够合理的讨论张量。因为,只有使用直接张量积的形式我们才能够很直接的定义什么是协变,因为只有在张量积的坐标架下面所有的基变换矩阵是乘起来的!

\rmk{
    同时这样的定义的结果也是,由于张量积是不能换顺序的!所以,所以我们的上下指标的框框次序是固定的。
    \eq{
        T^{\Box... \Box}{}_{\Box... \Box}
    }
    但是,框框里面可以填充不同的数。也就是\seq{\mu}之类的。他们都是具体的数。这个顺序的操作,我们也可以通过抽象指标来进行描述。
}
~\\

2.\textbf{我们为了表述张量是什么,我们需要明确张量空间的基之间的关系},我们定义张量空间的基满足下面的关系。以及!具体张量的基是什么并不重要,重要的是,基之间是无关的!:
\defi{\hdt{张量空间的基}

坐标系tensor product生成的基满足下方定义的基的变换的tensor product
写成的样子是:
\begin{equation}
    d x^{\mu} = \frac{\partial x^\mu(x^{\mu'})}{\partial x^{\mu'}} \bigg |_p d x^{\mu'} 
    \quad
    \partial_\mu = \frac{\partial x^{\mu'}(x^\mu)}{\partial x^{\mu}} \bigg |_p \partial_{\mu'}
\end{equation}
} 

根据我们定义,我们分别在两个坐标标架里面展开我们的某个张量,出于简单我们就写一个(1,1)rank张量:
\begin{equation}
    T = T^\mu{}_\nu (\partial_\mu)(d x^\nu) = T^{\mu'}{}_{\nu'} (\partial_{\mu'})(d x^{\nu'})
\end{equation}
怎么认定这个量是一个张量呢?
下面我们写一个抽象的张量的定义:
\defi{\hdt{抽象定义的张量}

    某个量在两个坐标系下\textbf{对于tensor product的基矢量进行展开}之后的分量形式之间是否满足和基矢量同样的变换形式
也就是说这个量在两个坐标之下展开分量正好满足:
\begin{equation}
    T^{\mu}_{\nu} = T^{\mu'}_{\nu'} \frac{\partial x^{\nu'}}{\partial x^\nu} \frac{\partial x^{\mu}}{\partial x^{\mu'}}
\end{equation}
}

这里我们使用一个简单的例子来说明,我们看看全反对称算子是不是张量,首先我们定义:全反对称算子在所有的坐标基下面都满足全反对称且取长度为1.
显然我们发现:因为左边是一个算符,右面是一个矩阵的行列式乘以这个算符。
\eq{
     \tilde{\epsilon}_{\mu_1...\mu_n} \neq \tilde{\epsilon}_{\mu_1...\mu_n} \frac{\partial x^{\mu_1}(x^{\mu_1'})}{\partial x^{\mu_1'}} \bigg |_p ... \frac{\partial x^{\mu_n}(x^{\mu_n'})}{\partial x^{\mu_n'}} \bigg |_p
}
所以全反对称算符在不同的基下面展开并不满足张量的变换关系。所以并不是张量。

当然我们判断一个量是不是张量必然提前知道这个量在不同基下展开的分量的定义,否则,我们也不能判断这是不是张量。

~\\


3. \textbf{在我看来讨论坐标系的基,也就是$\partial_\mu$和$d x^\mu$是不是张量是没有任何意义的。}

因为,张量定义必然涉及坐标系基的变化。而基的变化之后,基已经不再是坐标系的基了。
所以,我认为这个是不能讨论的。但有的时候我们会说,体元是一个张量密度,我认为这个说法只能说是形式化的定义。通过形式化的定义保证和欧几里得空间的
微积分的定义相符合即可,而不要讨论,体元是不是张量!

当然,对于这个问题还有一种理解就是,认为体元是一个对于levi-civita算符作为分量的张量,显然levi-civita算符是一个张量密度,所以体元可以理解为张量密度!
但是单纯讨论$dx^\mu$或者$\partial_\mu$是不是张量其实是没有意义的!
~\\

4. \textbf{广义相对论里面讨论的张量和量子场论里面讨论的张量的关系是什么}
\itm{
\pt{
    \hdt{主动和被动的观点:}

    广义相对论里面认为单个点上的张量协变是,对于“坐标基”的变化下,分量满足一定特殊的变化。我们认为广相的坐标基在变换,但是场论之中我们讨论完全不涉及坐标基,基并没有变换而是场在变换。
    
    看起来上面两种描述并没有任何关系。这是因为单个点上的张量的定义只能通过被动观点进行描述。但是一旦进行主动观点的描述,我们会发现两者都是一样的,只不过就是讨论一个流形上面的场,在流形发生微分同胚变换的时候的场的变换。

    只不过GR我们考虑的是全体微分同胚群,从场论的视角这一般就是一个gauge group。但是,QFT里面我们一般讨论的是global transformation group。
    
}
\pt{
    \hdt{对于任意坐标基变化协变:}

    我们简陋的会发现广义相对论的理论认为是对于“任何坐标变换”(我们后面会知道这其实就是等价于主动观点的任意微分同胚)我们都是协变的;但是量子场论我们会规定一个群G,我们认为张量是对于这个群G协变的量(也就是这个群给出的微分同胚),并且我们可以定义流行上面的张量场$T^{\mu}(x^\mu)$如果这是个张量场,那么应该满足对于群G的协变条件:
\begin{equation}
    T^{\mu'}(x^{\mu'}) = R(g)^{\mu'}_\mu T^{\mu}(x^\mu) 
\end{equation}

其中R是我们使用的群变换的群元素的表示!!这个概念是超级自然的。也就相当于,当坐标被一个变换变到另一个坐标点,函数被变换变到另一个函数。
新坐标点上的新函数形状应该跟旧坐标点上的旧函数形状“一样”。
一个图片来解释就是:
\begin{figure}[H]
    \includegraphics[width=8cm]{6311720842512_.pic.jpg}
    \centering
\end{figure}
}
\pt{
    \hdt{声明:}我们到现在讨论的都是某一个点上的,不涉及场!!因此我们不能用主动观点进行讨论。但是之后我们理解流形上的张量场之后就能够完全理解二者关系。
}
}




\begin{leftbar}
    \textbf{注释:}

    我们在上文一直用到了坐标变换的矩阵$\frac{\partial x^\mu}{\partial x^\nu}$其实这个是一个缩写。这不是一个矢量场或者什么意义上的东西。他就是一个普通到不能再普通的矩阵!
    一个问题就是怎么计算这个矩阵,我认为我们之前都是在进行抽象的讨论,如果上来直接用定义给出计算的方法,我觉得我会让人困惑。所以在此存疑,我们下文介绍具体的计算。

    其实我们的所有的讨论都是,一个矢量,乘一个矩阵,变成另一个矢量的讨论,全都是concrete的“数”的计算,而不涉及“场”。但是下一节我希望能够从
    “场”的角度重新澄清这个事实!
\end{leftbar}

\line

上方我们讲述的张量都是抽象定义的数学结构,其中很多很多定义和计算都显得很奇怪,以为我们只是“定义了要这么算”。我们并没有给定一个具体的张量的形式,给定一个具体的体系,告诉你很自然的满足这些数学结构并且能够进行concrete的计算。


但是,在微分几何里面我们可以通过一些流形上面的参数化明确的给出一个具体的张量的定义。并且这样的参数化
的定义自然满足所有张量的性质。而且更重要的是,这样的体系可以做具体的计算

我最开始不提及这个具体的定义,但是一直在使用相关的符号,就像是\seq{\partial_\mu}作为基底。是因为我认为具体的定义的张量有太多太多的性质
但是大家会因此忽略掉张量最本质的性质,也就是在坐标架改变的时候的协变性!


接下来我们通过流形定义其上面的张量:
\itm{
\pt{
   \hdt{坐标系:}我们定义坐标系是\seq{ \mathbb{R}^n}到n-dim流形的一个映射\seq{\psi},使得流形上面的点有:\seq{p = \psi(x^1,...,x^n)}
    很显然我们可以有很多个坐标系,那么他们之间的关系我也可以通过一个函数进行表达:
    \lqt{
        x^{1\prime} = x^{1\prime}(x^1,...,x^n)\\
        ...\\
        x^{n\prime} = x^{n\prime}(x^1,...,x^n)
    }
   }
\pt{
    \hdt{标量场:}我们赋予流形上每一个点一个数,我们称之为标量场,计作\seq{f:M\rightarrow \mathbb{R}}
}
\pt{
    \hdt{结合标量场和坐标系:}我们可以定义一个函数\seq{f: \mathbb{R}^3 \rightarrow  \mathbb{R}} 可以写成\seq{f(x^1,...,x^n) = k}
}
}
\defi{
    \hdt{矢量}
    
    我们定义我认为是从所有流形上的标量场生活的空间到实数的映射\seq{v: \mF_M \rightarrow  \mathbb{R}}。
    
    给定一个\seq{v}我们可以给出
    一个映射\seq{v(f) = m \in  \mathbb{R}}
}
\itm{
    \pt{
        \hdt{矢量空间的维度:}n-dim 流形上面的一个点的矢量场,维数是\seq{dim(V) = n}
    }
    \pt{
        \hdt{基矢量:}
        如果在流形我们考虑的点的附近引入一个坐标系\seq{\psi(x^1,...,x^n)}我们改写任意标量场是这样的:
        \seq{f(x^1,...,x^n)}(也就是结合标量场和坐标系)
    
        接下来我们定义这个\textbf{【坐标系下,某个点p上的】}基矢量为\seq{\partial_\mu}
        \eq{
            \partial_\mu (f(x^1,...,x^n)) = \frac{\partial f(x^1,...x^n)}{\partial x^\mu} \bigg |_p
        }
        显然我们认为一切矢量都可以用依赖于坐标系的基矢量展开:
        \eq{
            v = v^\mu \partial_\mu
        }
        其中我们认为\seq{v^\mu}是矢量在某个坐标系下的分量。
        这里我们虽然根据坐标基矢量的定义很容易我们可以知道基矢量的变换矩阵正好是:
        \eq{
            \partial_\mu = \frac{\partial x^{\mu'}(x^\mu)}{\partial x^{\mu}} \bigg |_p \partial_{\mu'}
        }
        所以我们认为这样具体的求导操作定义的基矢量是一个合理的基矢量的“表示”,
        其中使用了之前提及的坐标系变换的函数在p点的导数:
        \lqt{
        x^{1\prime} = x^{1\prime}(x^1,...,x^n)\\
        ...\\
        x^{n\prime} = x^{n\prime}(x^1,...,x^n)
    }
    进一步我们根据\seq{v}矢量的定义(Definition)里面的内容完全不涉及坐标系,我们可以知道,在坐标系变换前后\seq{v}是完全一样的东西,根据这个我们可以推导出
    结论:
    \eq{
        v^{\mu\prime} = \frac{\partial x^{\mu\prime}(x^\nu)}{\partial x^\nu} \bigg |_p v^\nu
    }
    我们会发现这个正好满足前文对于张量的定义。所以我们说这样根据映射定义的矢量是一个合理的矢量的“表示”;

    \at{
        当我们有一个坐标系也就是\seq{\mR^n \to M}的映射的时候我们会自然而然的完成两个任务:

        1. 用一组坐标标定流形上的点

        2. 为流形上每一个点的张量空间赋予一组基
    }
    }
\pt{
    \hdt{切矢量:}
    \thm{
        流形上的点p的矢量空间的【每一个矢量都可以用一种特殊的形式表示】,也就是写成“切矢量”。
        也就是存在一个函数:\seq{C(t) = \{ x^1(t),...,x^n(t)\}}
        \eq{
            v = \partial_t = \frac{\partial C^\mu(t)}{\partial t}\bigg |_p \partial_\mu
        }
        或者用我很爱的分量写法就是:
        \eq{
            V^\mu = \frac{\partial x^\mu(t)}{\partial t}\bigg |_p
        }
    }
    这个其实告诉我们一个重要的事实:\textbf{任何一个矢量都可以generate出来一堆曲线,这些曲线是一个微分方程的解;并且一个曲线也可以generate一个矢量。}
}
\pt{
    \hdt{
        对偶矢量:
    }
    很遗憾的是,对偶矢量我们不能够通过一个直观的求导来定义,但是目前有了矢量的定义,我们可以通过直观的
    定义出矢量计算出矢量,再根据矢量计算出对偶矢量。

    但是对于一些特殊的对偶矢量,比如全反对称的对偶矢量,我们可以赋予一些意义,详情见下方对于微分形式的讨论

    我们定义为:
    \defi{\hdt{对偶矢量}

        对偶矢量是一个矢量空间到实数的映射。\seq{\omega: V \rightarrow \mR}
    }
}
\pt{
    \hdt{对偶矢量基:}
    对于对偶矢量我们怎么进行映射我们并不知道,但是为了构建这个,我们必须构建基是怎么映射到实数的。当然显然,\textbf{对于同样的一个流形,同样的切空间我们可以有很多很多很多种不同的dual vector的构造方法},为了确定一种具体的构造方法,我们需要定义metric tensor。这个就留到后面讨论!

    我们可以通过这一点基矢量定义一个对偶矢量的基\seq{dx^\mu ( \partial_\nu ) = \delta^\mu_\nu}
    同样的我们可以验证这些都满足最开始对于抽象的张量的定义。
    位了保证表述清楚我还是写成:
    \eq{
        \omega = \omega_\mu dx^\mu
    }
    }
}

\rmk{
    关于书写张量的操作我们应该进行一个说明,张量写成分量形式之后我们会发现:
    \eq{
        T^{\mu_1...\mu_n}
    }
    这里的\seq{\mu_1...\mu_n}其实都是数。真正代表每一个指标的是,指标的位置而不是\seq{\mu_1...\mu_n}这些具体的数。
    真正的代表张量指标的应该是每一个指标的前后位置,其实真正的指标应该是:
    \eq{
        T^{\Box... \Box}
    }
    我们认为这些空格的顺序是不变的,本质的原因是因为张量积是不能换顺序的!!!所以空格的顺序就代表着对应的基的张量积的前后顺序。至于上面填充的数具体的数值,我们认为是当某个顺序的基取这个数值标记的特定的基的时候的结果。
}


\subsection{整个流形上每一点构成的张量场}
顾名思义,我们只要在流形上面每一个点引入一个张量,那么我们就可以得到一个张量场。显然,张量场是一个流形上面每一个点到这个张量空间的映射\seq{M \rightarrow T}我们考虑这个映射
的很多性质。

不同于单纯的讨论“每一个点”上面的张量,我们可以通过张量场给出一套等价的张量的定义。
我们称“某一个点上的张量”的定义为“被动观点下张量的定义”;我们称“张量场意义下张量的定义”为“主动观点下张量的定义”

\rmk{我个人认为最有意义的是讨论主动观点下张量的定义,因为我们日常研究张量并不是单独的拿出一个点
上的张量进行研究,而是研究流形上面的张量场,以及流形进行变化和变形的时候张量场是怎么变形的。

特别是这个很像量子场论之中,我们把一个理论按照一个群进行变换,而当这个群作用的空间包含我们的流形空间。比如,我们对于一个四维是空的量子场论
进行变形的时候,我们将一个SO(3,1)群元素作用在这上,那么这个张量场按照什么变化呢?

我们知道,张量场的变化满足:
\eq{
    T'^\mu(x') = R^\mu{}_\nu T^\nu(x)
}
的变化关系。我们认为这个量T是一个SO(3,1)协变张量

那么值得思考的是,广相的张量场是不是满足一样的形式,广义相对论里面定义的张量又是怎么协变的呢?

我们的结论是,广相的张量确实满足同样的定义形式(并且等价于之前说的定义);广义相对论里面的张量对于流形的一切微分同胚协变!}

为了研究流形上的张量场的主动变换,我们需要定义:【在流形发生变化的时候,其上张量场应该怎样变化】。
其实这个与其作为定理,不如作为一个协变的张量场的定义。但是我们已经通过每个点上的张量定义了张量。那么我们下面需要干的是,根据已经定义的张量保证
流形进行微分同胚变换前后,其上张量的形式的前后的变化保证其在每点上是一个张量。

\itm{
    \pt{
        \hdt{微分同胚:} 我们定义\seq{\phi}是维度相同的两个流形M到N的微分同胚。

        根据\seq{\phi}这个映射我们可以完整的定义一种流形变化下其上张量场的变换方式,我们会发现其变换前后如果分量满足一定性质,等价于张量的定义。那么我们使用这个作为张量的第二种定义。

    }
    \pt{
        \hdt{pull back: }pull back \seq{\phi^*}表示如果N上有张量场怎么定义M上张量场。

        所有的N流形上面的点,任取一个点为\seq{\phi(p)}其切空间上的张量\seq{T \big |_{\phi(p)}}可以导致一个M流形上面的p点的张量:
        \eq{
            T' \big |_p = (\phi^* T) \big |_p 
        }
    }
    \pt{
        \hdt{push forward:}push forward \seq{\phi_*}表示如果M上有张量场怎么定义N上的张量场。

        所有的M流形上面的点,任取一个点为\seq{p}其切空间上的张量\seq{T \big |_{p}}可以导致一个N流形上面的\seq{\phi(p)}点的张量:
        \eq{
            T' \big |_{\phi(p)} = (\phi_* T) \big |_\phi(p) 
        }

    }
}
两个映射的具体定义可以从下面书中查到,我不想多说,因为太细节了。但重要的是我们清楚,这个函数是能够通过我们的微分同胚\seq{\phi}
唯一的确定的映射。
\pict{
    2024-07-20-13-56-45.png
}{0.9}
\pict{2024-07-20-13-57-51.png}{0.9}

\rmk{
    一般情况下我们只能定义一个(k,0)张量场的push forward或者一个(0,l)张量场的pull back
    但是,对于微分同胚我们可以定义任意张量。因为微分同胚我们可以吧逆映射的pull back / push forward作为一个定义。
}


接下来我们介绍一个神奇的定理。
\imp{协变等价定理}{
    对于赋予一个坐标系\seq{p = \psi(x^1,...,x^n)}之后的流形 M(其上某一点p的张量空间坐标为\seq{\{ x^\mu\}})
对于赋予一个坐标系\seq{p = \psi(y^1,...,y^n)}之后的流形 N(其上某一点p的张量空间坐标为\seq{\{ y^\mu\}})
对于坐标系诱导出的张量空间之中的张量对于相应基的分量来说我们有:

~\\
    存在一组M流形上的坐标系我们定义为\seq{p = \psi'(x'^1,...,x'^n)}这组坐标下M上点p的分量满足:
    \eq{
        x'^{ \mu}(p) = y^\mu (\phi(p)) 
    }
    这个时候我们会发现这个微分同胚和N流形上的坐标系诱导出的M流形上坐标变换下,M流形上p点的张量在新坐标下的分量与N流形上微分同胚诱导出的张量在y坐标下的分量满足下面关系:
    \eq{
        (\phi_* T)^{\mu_1...\mu_k}{}_{\nu_1...\nu_k} \bigg|_{\phi(p)} = T^{\mu_1'...\mu_k'}{}_{\nu_1'...\nu_k'} \bigg |_p = (matrix \ x \to x')T^{\mu_1...\mu_k}{}_{\nu_1...\nu_k} \big|_p
    }
}


这个定理的存在让我们可以换一种方式翻译张量的定义之中:“坐标系变化下分量按照坐标系变化的矩阵变化”为“流形进行微分同胚变化时,被push forward之后在新的坐标系下分量按照一种特殊坐标系变化矩阵变化”
这里我们会发现微分同胚可以理解为定义了一种特殊的坐标系变化!如果满足张量的定义,那么微分同胚前后张量分量必须按照一定规则变化!






\subsection{流形上张量场的内积结构}
对于我们的流形的矢量空间,我们其实可以引入内积结构的。这样子我们可以定义矢量的“长度”的概念,这就可以更好的描述我们的物理世界,毕竟我们世界上很自然的是有长度的概念的。为了定义长度的概念我们需要引入一个很重要的张量场(后面我们会意识到,这个张量场描述了流行很重要的信息)——\textbf{metric tensor field}。我们给出定义:
\defi{
    metric tensor field

    g 是流形上面的一个对称的,non-degenerate的(0,2)的张量场。
    \itm{
        \pt{对称:指$ g(u,v) = g(v,u) $也就是作用在(1,0)矢量上面不分先后 }
        \pt{non-degenerate:指$ g(u,v) = 0, \forall u \in V \to v = 0 $ }
    }
}
\rmk{
    non-degenerate的条件等价于对于任意基进行展开展开系数的矩阵行列式不为0:
    $$
        det(g_{\mu\nu}) \neq 0
    $$ 
}
metric作为(0,2)Tensor构造的定义是把两个矢量映射成为一个数字。这类似于我们的内积。
\imp{流形上切空间的内积}{
    我们使用metric定义流行上的切空间的内积。因为我们之前定义dual vector其实是通过对vector到实数映射定义的,那么我们不妨通过这个定义很自然的引入内积结构。

    \itm{
        \pt{矢量的长度我们定义为:$ \left\vert v \right\vert \overset{\underset{\mathrm{def}}{}}{=}\sqrt[root]{\left\vert g(v,v) \right\vert }  $ }
        \pt{矢量正交指的是:$ g(v,u) = 0 $ }
        \pt{一组基是正交归一指的是:$ g(e_\mu,e_\mu) = \pm 1, \forall \mu $ }   
    }
}
有了上面的定义,我们有下方的一个可以证明的数学定理:
\thm{
    local flat theorem

    任何定义metric的矢量空间,永远可以找到一个基使得这组基是正交归一的。

    翻译成物理的话,就是永远存在一个坐标系,使得对应的切空间的基是正交归一的,$ g_{\mu\nu} = \delta_{\mu\nu} \pm 1 $ 
}

根据这个定理我们就可以定义这个metric的一些名词。首先,我们使用正交归一的基之后我们可以通过对角元素的$ {+1,-1,0} $的选取对于度规进行分类,分成:
\itm{
    \pt{正定的:所有元素都是+1}
    \pt{负定的:所有元素都是-1}
    \pt{Lorenzian:只有一个元素是-1的}
} 
metric的对角元的和我们称之为号差!!
\defi{
    xxlike vector

    \itm{
        \pt{$ g(v,v) >0  $ 我们称之为 spacelike vector}
        \pt{$ g(v,v) <0  $我们称之为timelike vector}
        \pt{$ g(v,v) =0  $我们称之为null vector}
    }
}

对于同样的一个流形我们可以赋予很多种不一样的度规结构。(就像是对于同样的一个流形我们可以赋予很多不一样的dual vector的具体构造)(当然,赋予一个metric也意味着我们具体的定义了dual vector的构造)。一个被赋予了具体的度规结构的流形我们称之为\textbf{广义黎曼空间}:(M,g)。每一个广义黎曼空间我们都可以很具体的构造出来,vector和dual vector并且内积是有具体的定义的。从此开始我们才能具体的计算流形上面的数值!

显然同样的一个流形我们可以赋予不一样的metric。比如:
\itm{
    \pt{$ \mathbb{R}^n $空间,我们可以对于正交归一基(显然我们还有其他基,比如,对于球坐标搞出来的基)赋予全是+1的度规,这个广义黎曼空间我们称之为“欧氏空间”}
    \pt{$ \mathbb{R}^n $空间,我们可以对正交归一基赋予\{-1,1,1,1,\dots\}的度规,这个广义黎曼空间我们称之为“Minkovski 空间”}
}

显然,选取不一样的基,我们的g张量的分量会按照张量的变换矩阵进行变换。但是,我们会发现一些特殊的坐标变换,让g张量的分量并不发生变换。这些变换构成了一些群!比如,Minkovski空间之中,这些变换构成$ SO(3,1) $群!! 我们会认为Minkovski空间是$ SO(3,1) $不变的!! 

\imp{里程碑}{
    这里我们完成了“广义黎曼空间”的全部基础结构的构造。我们的广义相对论完全就是在这样的一个空间上面进行讨论的!!!

    后面我们进一步讨论,这个空间上能够赋予什么更加复杂的结构!
}

\subsection{抽象指标记号}

后面的章节之中。为了记号方便(虽然我个人表示怀疑)我们会使用一个抽象指标记号来书写张量。

这个时候。$\omega_{a_1...a_p}$代表的并不是某一个张量的分量,而代表的就是一个(0,p)rank的张量
用某一个确定的坐标架展开应该写成:
\begin{equation}
    \omega_{a_1...a_p} = \omega_{\mu_1...\mu_p} (d x^{\mu_1})_{a_1}...(d x^{\mu_p})_{a_p}
\end{equation}
同样的这里面使用的$(d x^{\mu_i})_{a_i}$表示的也是某个张量空间的基,是一个有(0,1)rank的张量。而不是什么分量,更不是(1,0)rank 张量的分量!!!

我们使用抽象指标的原因是因为,张量积不能换顺序的要求太奇怪难以写了。我们希望有一个能随便换书写顺序的东西当成基础,但是,同时也保证有一个东西label这个“顺序”。

这里抽象指标就起到了这个label顺序的作用。我们认为:
\eq{
    \omega_a \mu_b = \mu_b \omega_a
}
虽然是张量积但是,如果抽象指标和张量符号的顺序一起交换,那么这个张量积是不变的。\hdt{但是,如果只有抽象指标的顺序交换或者只有张量符号的顺序交换,那么我们认为是张量积发生了交换。这样子我们建立了一个抽象指标表达的顺序体系。}

\imp{一句话说明白抽象指标}{
  一般的张量指标是不能换顺序的,因为我们指标都是框框!对应着1,2,3,4...这些标号!!但是抽象指标我们并不再是使用数字进行框框的标号了!!我们改用字母进行标号!!并且我们认为字母的顺序是可以交换的!!
}


下面一个问题是,普通的张量定义是怎么和抽象指标的定义相对应的!结论是这样的:

\thm{
    抽象指标和普通张量对应

    如果等式两边抽象指标的顺序是一样的,那么可以直接消除所有的抽象指标记号了!
}

对于一般情况来说,由于抽象指标和张量记号同时交换并没有任何意义,所以我们一般只交换一个但是至于交换哪个我们看情况计算。很多时候为了些全反对张量好写,我们一般选择交换抽象指标的顺序。因为抽象指标的顺序代表着张量积的顺序。

比如,下面我们就可以定义两个完全不一样的张量:
\begin{equation}
    \omega_{a_1 a_2 ...a_p} = \omega_{\mu_1...\mu_p} (d x^{\mu_1})_{a_1} (d x^{\mu_2})_{a_2}...(d x^{\mu_p})_{a_p}
\end{equation}
以及
\eq{
    \omega_{a_2 a_1 ...a_p} &= \omega_{\mu_1...\mu_p} (d x^{\mu_1})_{a_2} (d x^{\mu_2})_{a_1}...(d x^{\mu_p})_{a_p}\\
    &=  \omega_{\mu_1...\mu_p} (d x^{\mu_2})_{a_1} (d x^{\mu_1})_{a_2}...(d x^{\mu_p})_{a_p}\\
    & = \omega_{\mu_2 \mu_1...\mu_p} (d x^{\mu_1})_{a_1} (d x^{\mu_2})_{a_2}...(d x^{\mu_p})_{a_p}
}
最后一行我们一般不这么写,因为这个很confusing。但是我们知道这是对的。并且这样写出来我们会发现一个很明显的东西。我们写成下面的定理:
\thm{
抽象指标交换

    对于一个张量进行抽象指标的交换。变换前后对于同样一组张量按照同样的顺序张量积之后的结果,就是相当于把分量的对应的两个指标位置进行交换。
    \eq{\omega_{a_1 a_2 ...a_p} = \omega_{\mu_1 \mu_2...\mu_p} (d x^{\mu_1})_{a_1} (d x^{\mu_2})_{a_2}...(d x^{\mu_p})_{a_p}}
    \eq{
        \omega_{a_2 a_1 ...a_p} = \omega_{\mu_2 \mu_1...\mu_p} (d x^{\mu_1})_{a_1} (d x^{\mu_2})_{a_2}...(d x^{\mu_p})_{a_p}
    }
}
\imp{呃呃呃}{
  本页书写的东西比较抽象,不建议记住。没有任何意义。我们不要对比任何抽象指标顺序不同的张量就可以了!!!永远记住Theorem 3捏捏
}

\subsection{SO(3,1)群的特殊讨论}
我们会讨论这个是因为,我们在狭义相对论里面,仅仅讨论$ SO(3,1) $外加平移微分同胚协变的张量,而并不讨论对于任意变换协变的张量!!我们的4-vector都是这样的张量!!! 


\subsection{流形上张量场具体计算}
首先我们需要明确我们一般提到的矩阵是什么?我们认为矩阵代表着一个(1,1) rank张量!
因此矩阵可以直接进行很多操作,包括:
\itm{
    \pt{矩阵直接进行对角项的求和就可以得到矩阵的迹}
}


\newpage
\section{流形上微积分}
对于任意流形上面定义微积分是一件很困难的事情。并不是每一个流形都可以定义矢量场的微积分。因此,人们发明了微分形式的数学语言来描述流行上面的场。
并且正因为这样的数学语言的出现我们才能够很顺利的在奇奇怪怪的流形上面定义微积分。

我们可以研究我们熟悉的三维的微积分,我们可以定义矢量的积分:
\begin{equation}
    \int_l \vec{A} \cdot d \vec{l} \quad \int_S \vec{A} \cdot d \vec{S}
\end{equation}
这两种二型微积分都是表示的是矢量场在三维空间中的积分,这两种积分存在一些问题:
\begin{itemize}
    \item 都是定义在embed在三维空间之中的两种流形上面的,但是对于一般的流形我们不一定可以embed一个更高维度的空间
    \item 这两个流行都有定向,虽然在我们的积分写法之中这个定向表达的并不是很显然。但是推广到更高维度的流形就很奇怪
\end{itemize}
因此我们为了定义一般流形上的微积分就不再使用这样的二型矢量微积分的语言,因为这样的语言过于复杂。此时我们选用微分形式的语言,并且我们定义只包含标量积分的推广(也就是说p-dim流形上面只能定义p-form的微积分)这样子我们再也不用在奇怪的流形上
“特殊定义矢量微积分”同时还能很顺利的“计算矢量微积分”。

\subsection{微分形式的微分语言}
首先,我们定义微分形式
\defi{
    n-dim流形上面的一组全反对称的(0,p)张量被称为p-form
    \begin{equation}
        \omega_{a_1,a_2...a_p} =  \omega_{[a_1,a_2...a_p]}
    \end{equation}
    如果使用一般的指标书写:
    \begin{equation}
        \omega_{\mu_1,\mu_2...\mu_p} dx^{\mu_1}\cdots dx^{\mu_p}=  \omega_{[\mu_1,\mu_2...\mu_p]}dx^{\mu_1}\cdots dx^{\mu_p}
    \end{equation}
    }


\begin{leftbar}
    \textbf{注释:}
    
    关于什么是全反对称,我们注意,对于一般的指标记号来说张量指标的前后顺序是固定的,从左数第几个指标代表着固定的有物理意义的指标!不固定的东西是上面使用的字母,字母代表着分量,代表着这个指标和哪个指标进行缩并。所以字母并不代表着指标!!真正的指标应该是“框框”。而使用抽象指标的记号,我们可以很显示的表达“框框”是什么!

    反对称的定义是交换字母,也就是交换指标上面的取值,而不是交换指标!也就是说交换取值之后得到的张量输出正好相差-1
\end{leftbar}


由于是一个张量,所以他满足所有张量应该有的性质,再由于反对称的性质,某个流形上面的点x上的p-form生活的空间必然是(0,p)张量生活的空间上面的子空间,这个子空间的维度是。
\begin{equation}
    dim(\Lambda_x^p) = \frac{n!}{p!(n-p)!}
\end{equation}


我们注意,我们为什么要定义微分形式,是因为p-form是可以定义在p-dim流形上面积分的张量!!这个定义。
那么事关微积分首先我们关心的问题是,怎么对form进行:
\begin{itemize}
    \item {\textbf{微分:}
    \begin{equation}
        d \omega_{a_1...a_p} = (p+1) \partial_{[ \mu} \omega_{a_1...a_p ]}
    \end{equation}
    或者使用普通指标的记号:
    \eq{
        (\mathrm{d}A)_{\mu_1\cdots\mu_{p+1}}=(p+1)\partial_{[\mu_1}A_{\mu_2\cdots\mu_{p+1}}].
    }
    注意这里我们用的偏导数算符我们并没有定义是什么,可以根据我们的使用的语境进行定义。
    以及根据反对称的性质,显然我们有:
    \begin{equation}
        d d = 0
    \end{equation}
    }
    \item {\textbf{张量积:}
    关于我们怎么把一个n-form和m-form进行张量积再进行变换成为一个m+n-form。很简单只需要反对称化即可。
    \begin{equation}
        (\omega \wedge \mu)_{a_a...a_m,b_1...b_n} = \frac{(m+n)!}{m! n!}\omega_{[ a_1...a_m}\mu_{b_1...b_n]}
    \end{equation}
    或者使用普通指标的记号书写:
    \eq{
        (A\wedge B)_{\mu_1\cdots\mu_{p+q}}=\frac{(p+q)!}{p!q!}A_{[\mu_1\cdots\mu_p}B_{\mu_{p+1}\cdots\mu_{p+q}]}. 
    }
    \rmk{
        显然我们也可以对于基矢量进行wedge操作,得到一个基矢量的线性组合。进行这个操作我们有两种方法,一个是使用普通指标进行计算,另一个是使用抽象指标进行计算。
        
        普通指标计算:
        \eq{
            dx^0 \wedge dx^1 = (dx^0 \wedge dx^1 )_{\mu\nu} dx^{\mu} dx^{\nu} = dx^0dx^1 - dx^1dx^0
        }

        抽象指标计算:
        \eq{
            (dx^0 \wedge dx^1)_{ab} = dx^0_a dx^1_b - dx^1_a dx^0_b
        }这个时候我们发现等式两边的抽象指标的顺序是完全一样的可以消除,所以就可以还原成为普通指标的计算结果。
    }




    同时很显然我们有一些性质:
    \begin{equation}
        A \wedge B = (-1)^{mn} B \wedge A
    \end{equation}
    对于p-form $\omega$还有q-form $\mu$我们可以有恒等式:
    \begin{equation}
        d(\omega \wedge \mu) = (d \omega)\wedge \mu + (-1)^p \omega \wedge (d \mu)
    \end{equation}
    }
    \item{ \textbf{展开:}选定一个坐标系$\psi$之后一个n-dim流形上某个点x上面的p-form可以展开成:
    \begin{equation}
        \omega_{a_1...a_p} = \omega_{\mu_1...\mu_p} (d x^{\mu_1})_{a_1}...(d x^{\mu_p})_{a_p}
    \end{equation}
    但是,这个坐标基并不好,因为没有通过坐标直接体现出反对称的特性,这个时候我们会用另外一组坐标基矢量进行展开,我们使用一个组全反对称基矢量进行展开:
    \begin{equation}
        \omega_{a_1...a_p} = \omega_{\mu_1...\mu_p}  (d x^{\mu_1}\wedge ... \wedge d x^{\mu_p})_{a_1 a_2 a_3...a_p}
    \end{equation}

    注意有wedge product的定义,相当是把坐标基矢量全部全反对称化!
    \at{
        对于全反对称张量我们会发现一个特质就是,全反对称(0,n)张量(也就是n-form)在这两个基之下展开的分量是一模一样的!
    }
    
    根据之前wedge product导数的性质,展开后的分量的导数我们有:
    \begin{equation}
        (d \omega)_{b a_1...a_n} = (d \omega_{\mu_1...\mu_n})_b \wedge(d x^{\mu_1}\wedge ... \wedge d x^{\mu_p})_{a_1 a_2 a_3...a_n}
    \end{equation}
    注意上面的式子等号右面第一项是一个标量场$\omega_{\mu_1...\mu_n}$的1-form,所以也是1-form!

    
    }
\end{itemize}

\subsection{微分形式定义流形上的积分}
我们推广最基础的黎曼n维实数空间上面的积分!
\defi{
    n-dim流形上面只能对于n-form进行积分,我们的选择一个n-form 称为$ a $。其积分数值为,选定一定的n-dim流形上的坐标系$ \psi $之后
    \begin{equation}
        \int_{\Omega} (a_{0...n-1}(x^\mu)) (dx^0) \wedge ... \wedge (dx^{n-1}) = \int_{\psi(\Omega) \subset \mathbb{R}^n} d^n x \  a_{0...n-1}(x^\mu) 
    \end{equation}
    等式左边是一个n形式的展开形式在n维流形上面的积分,又面是展开函数在对应的欧式空间的积分(我们已经定义好的!并且可以很简单的计算的多重积分)

}


接下来我们考虑函数在某个流形上的积分操作。由于函数是一个0-form所以我不能够直接在流形上面定义积分,而是应该把它变成一个n-form再进行积分。

\defi{
    函数(或者说标量场)在流形上的积分可以分为两步:

    step 1: 通过变换把标量场变成一个n-form
    \begin{equation}
        \begin{split}
            f(x) \rightarrow f(x) \epsilon(x) = f(x) \sqrt{|g(x)|} \tilde{\epsilon}_{\mu_0...\mu_{n-1}}/n! dx^{\mu_0}\wedge...\wedge dx^{\mu_{n-1}}\\
             = f(x)\sqrt{|g(x)|}dx^{0}\wedge...\wedge dx^{n-1}
        \end{split}   
    \end{equation}

    step 2: 对于n-form进行积分
\begin{equation}
    \int_{\Omega} f(x) \sqrt{|g|} (dx^{0})\wedge...\wedge (dx^{n-1}) = \int_{R^n(not \ all)} d^n x \  f(x) \sqrt{|g(x)|} 
\end{equation}

}

所以,通常我们会混淆两个经常使用的记号,我们会认为:
\eq{
    \sqrt{|g(x)|} d^n x = \sqrt{|g(x)|} (dx^{0})\wedge...\wedge (dx^{n-1})
}
这个仅仅是一个积分微元的记号。我们写出积分微元时候同时表达了两件事。
\itm{
    \pt{
        我们已经确定了一个坐标系\seq{x^0,x^1,...,x^{n-1}}并且我们在这个确定的坐标系里面进行正常的标量积分
    }
    \pt{
        \seq{\sqrt{|g(x)|} d^n x}本身就是一个微分形式,我们可以把它看成是一个n-form。并且微分形式的内容就是\seq{\sqrt{|g(x)|} (dx^{0})\wedge...\wedge (dx^{n-1})}。但同时,他也代表着这个微分形式在流形上面积分对应的\seq{R^n}空间上面的积分的微元。
    }
}

\begin{leftbar}
    \textbf{注释:}
    
    正如前面提到的,$d x^\mu$我们不能理解是不是一个张量,这样讨论也没有意义。
    这也说明我们盲目讨论$dx^{\mu_0} \wedge ...\wedge d x^{\mu_{n-1}}$是不是张量很没有意义。
    
    但是换一种思路其实我们可以认为它是一个张量密度。 我们可以理解为是一个对于levi-civita算符作为分量的张量,其实很合理的!因为是levi-civita算符加上weight会变成
    全反对称张量,所以我们可以直接从wedge的空间变成tensor product的空间讨论它是不是张量(由于上面(12)(13)式子对于展开的描述,对于全反对称张量,分量并不发生变化)
\end{leftbar}

最后我们推广曲面上的积分,我们会发现我们是在一个n-p维的流形上对于一个p-form进行积分。因此我们需要开发一个手段
把一个p-form变成一个n-p form从而保证能在n-p维的流形上面进行积分。由此我们定义:
\defi{
    我们定义hodge dual使得一个n维流形上的p-form变成一个n-p form
    \eq{
        \begin{aligned}
            & * \omega_{a_1 \cdots a_{n-1}}:=\frac{1}{l!} \omega^{b_1 \cdots b_l} \varepsilon_{b_1 \cdots b_l a_1 \cdots a_{n-l}}, \\
            & \text { 其中 } \quad \omega^{b_1 \cdots b_l}=g^{b_1 c_1} \cdots g^{b_l c_l} \omega_{c_1 \cdots c_l} \text {. } \\
            &
        \end{aligned}
    }
}
关于hodge dual 有一些常用的结论:
\eq{{ }^{* *} \omega=(-1)^{s+l(n-l)} \omega \text {. }}
根据这样的定义我们可以进一步定义:
\defi{
    推广的积分为对hodge dual的积分!
    \eq{
        \int * \omega
    }
} 





\newpage
\section{流形上的对称性}
\subsection{不赋予度规结构的流形上李导数}
首先我们需要定义一个概念即为\textbf{单参数微分同胚群}:
\defi{
    单参数微分同胚群
    
    指的是如果存在一个$ C^\infty $的映射$ \phi: \mathbb{R} \times M \to M $  这个映射的集合被称为单参数微分同胚群,满足下面的条件:
    \itm{
        \pt{$ \phi_t : M \to M $对于$  \forall t \in \mathbb{R} $  }
        \pt{$ \phi_t\circ\phi_s=\phi_{t+s}\mathrm{,}\forall t,s\in\mathbb{R}\mathrm{~.} $ 这个定义了群的加法!!}
        \pt{对于0元我们选用identity的微分同胚}
    }  
}
下面我们根据这个定义我们会发现,任何的一个单参数微分同胚群会给出一个矢量场。我们的操作时。如果在一个流形$ M $上面赋予一个单参数微分同胚群。
\itm{
    \pt{首先我们有一个单参数微分同胚群$ \phi $ }
    \pt{考虑p点在全部为、单参数微分同胚群的群元素作用下怎么变化:对于这个流形上的某一点$ \forall p \in M $。$ \phi_t (p) : \mathbb{R} \to M $给出了一个这个流形上面的一个映射。\textbf{显然这个曲线一定过p点!!因为群必须有0元,我们一般用$ t = 0 $来标记0元 } ——我们称这个映射是单参数微分同胚群过p点的orbit}
    \pt{给出这个参数化曲线的零点定义,我们定义$ \phi_0(p)  = p$ 也就是$ t = 0 $参数化出来的那个微分同胚把p点映射到p点。 }
    \pt{我们这样子可以在p点的企鹅空间构造一个矢量。用同样的方法在流形上面每一个点的切空间的矢量也就是一个矢量场}
}
矢量场在很多时候也可以给出一个单参数微分同胚群。对于一个矢量场来说我们可以给出一个\textbf{integral curve}。也就是,我们固定流形上的一个坐标。我们可以给出一个曲线的参数化,保证其在坐标下的分量和矢量场在同样的一个坐标下的分量满足:
\eq{
    \frac{dx^\mu}{dt}=V^\mu.
}
那么我们会给出这个流形上面的一系列曲线。对于全空间我们的一个曲线其实相当于generate一个微分同胚。只需要固定一个t。那么$ \forall t , ~~ x^\mu(0) \to x^\mu(t) $ 给出了一个微分同胚(当然也可以理解为坐标变换,毕竟从流行的视角下面两者是等价的。)

\imp{李导数的概念}{
    \textbf{对于理解李导数我们需要有两个角度,一个是抽象的定义;一个是计算方法。}

    首先抽象的定义一下,李导数给出了怎么对于一个无穷小的微分同胚作用下,张量的变化率(并且这样的变化率也是一个张量。)我们定义为:
    \eq{
        \mathscr{L}_vT^{a_1\cdots a_k}{}_{b_1\cdots b_l}:=\lim_{t\to0}\frac{1}{t}(\phi_t^*T^{a_1\cdots a_k}{}_{b_1\cdots b_l}-T^{a_1\cdots a_k}{}_{b_1\cdots b_l})
    }
    \itm{
        \pt{我们注意,其中涉及张量场的减法。我们之前只能说明存在这样的运算,但是并没有给出很具体的说法。但我们一般定义减法就是\textbf{在同一个坐标系的减法}。取极限的定义也是某个坐标系下面的数的极限。}
        \pt{我们还注意到$ \phi_t^*T^{a_1\cdots a_k}{}_{b_1\cdots b_l} $ 是一个M流形上的矢量场,只是这个矢量场被微分同胚pull back到和$ T^{a_1\cdots a_k}{}_{b_1\cdots b_l} $在同样的一个点的切空间上然后做减法。 }
    }

    上面是抽象的定义,抽象的定义的好处是,让我们知道,我们其实干的事情就是把一个张量场“求导”变成另外的一个张量场。但是问题是,我们完全不知道怎么实际的计算这个量。因为一切具体计算时需要依赖于坐标系的,下面我们从坐标系和具体计算的视角重新定义李导数。

    \defi{
        我很爱的一套李导数的具体定义

        对于一个矢量场来说我们可以给出一个这个流形到他自己的单参数微分同胚群$ \phi_t :M \to M $。对于这个流形$ M $我们可以赋予一个张量场$ T^{a_1\cdots a_k}{}_{b_1\cdots b_l} $。我们考虑这个流形上面的两个点,$ p $和$ \phi_t(p) $。
        \itm{
            \pt{我们选择某一个特定的坐标系进行计算}
            \pt{这个坐标系下面我们这两个点的张量场的分量分别是:
            $ T^{\mu_1\cdots\mu_k}{}_{\nu_1\cdots\nu_l}(\phi_t(p)) $ 以及$ T^{\mu_1\cdots\mu_k}{}_{\nu_1\cdots\nu_l}(p) $ }
            \pt{我们定义一个difference between two tensor at p point:
            \eq{
                \Delta_tT^{\mu_1\cdots\mu_k}{}_{\nu_1\cdots\nu_l}(p)=\phi_t^*[T^{\mu_1\cdots\mu_k}{}_{\nu_1\cdots\nu_l}(\phi_t(p))]-T^{\mu_1\cdots\mu_k}{}_{\nu_1\cdots\nu_l}(p).
            }
            }
        }
        在上面的准备的基础上我们可以定理某一点在某坐标系下面的李导数的张量分量是:
        \eq{
            \mathcal{L}_VT^{\mu_1\cdots\mu_k}{}_{\nu_1\cdots\nu_l}=\lim_{t\to0}\left(\frac{\Delta_tT^{\mu_1\cdots\mu_k}{}_{\nu_1\cdots\nu_l}}{t}\right).
        }
    }
}

\rmk{
    我们自始自终只使用了一个流形$ M $。并不存在两个流形我们考虑的是一个流形到自己的单参数微分同胚群!!!
}

下面我们一步步讨论李导数的很多重要的性质。

\vspace{0.7em}
\textbf{李导数是线性的}
\eq{
    \mL_V(aT+bS)=a\mL_VT+b\mL_VS.
}

\textbf{李导数在张量积下满足分配律}
\eq{
    \mathcal{L}_V(T\otimes S)=(\mathcal{L}_VT)\otimes S+T\otimes(\mathcal{L}_VS),
}

\textbf{李导数作用在标量场上}
\eq{
    \mathcal{L}_Vf=V(f)=V^\mu\partial_\mu f.
}
当然对于标量场,导数和协变导数并没有任何区别捏(((我们写导数算符就好了呃呃

\vspace{0.7em}
\textbf{李导数在适配坐标系}

为了很好定义,我们可以选择一个特殊的坐标系$ x^\mu=(x^1,\ldots x^n), $ 满足下面的性质,$ V = \partial_1 $或者说用分量的语言就是$ V^\mu=(1,0,0,\ldots,0) $ 。使用满足这个性质的坐标系,我们称之为使用\textbf{适配坐标系},我们可以有结论:
\itm{
    \pt{某个矢量场的适配坐标系下,这个矢量场generate出来的单参数微分同胚群的映射在这个坐标系的坐标是:
    \eq{
        \phi_t(x^{\mu}) = y^\mu = (x^1+t,x^2,\dots,x^n)
    }
    也就相当于坐标第一个分量发生移动。有一种流形沿着矢量场$ V $的方向稍微动一个t的感觉。 
    }
    \pt{变换于一个具体的张量的时候就会是这样的:
    \eq{
        \phi_{t*}[T^{\mu_1\cdots\mu_k}{}_{\nu_1\cdots\nu_l}(\phi_t(p))]=T^{\mu_1\cdots\mu_k}{}_{\nu_1\cdots\nu_l}(x^1+t,x^2,\ldots,x^n).
    }
    }
}
在上面的讨论的基础上我们会意识到:
\eq{
    \mL_VT^{\mu_1\cdots\mu_k}{}_{\nu_1\cdots\nu_l}=\frac{\partial}{\partial x^1}T^{\mu_1\cdots\mu_k}{}_{\nu_1\cdots\nu_l},
}
对于一个矢量场来说:
\eq{
    \mathcal{L}_VU^\mu=\frac{\partial U^\mu}{\partial x^1}.
}

\textbf{李导数和对易子}

我们给出流形上面的矢量场可以定义对易子的概念。
\imp{矢量场的对易子}{
    对于两个矢量场矢量场u,v来说我们定义他们的对易子是一个矢量场。并且这个矢量场的形式定义是依赖于作用上一个标量场f的:
    \eq{
        [u,v]\left(f\right):=u\left(v\left(f\right)\right)-v\left(u\left(f\right)\right),\quad\forall f\in\mathscr{F}_M.
    }
    一个性质是,坐标系基矢量的对易子一定为0(就是把所有的标量场映射到0的映射)

    我们知道对易子是可以用协变导数算符等价的表示的。并且我们的协变导数仅仅要求是torsion free的!并不一定是metric compatible的。
    \eq{
        [u,\upsilon]^a=u^b\nabla_b\upsilon^a-\upsilon^b\nabla_bu^a\mathrm{~,}
    }

}
我们会发现矢量场的李导数可以和这些定义联系起来:
\eq{
    \mathscr{L}_vu^a=[v,u]^a,\quad\forall u^a,v^a\in\mathscr{F}(1,0),
}
或者说等价的有:
\eq{
    \mathscr{L}_vu^a=v^b\nabla_bu^a-u^b\nabla_bv^a,
}
对于对偶矢量我们同样满足:
\eq{
    \mathscr{L}_v\omega_a=\upsilon^b\nabla_b\omega_a+\omega_b\nabla_a\upsilon^b,\quad\forall\upsilon^a\in\mathscr{F}(1,0),\omega_a\in\mathscr{F}(0,1),
}

\textbf{一般张量场的李导数}
\eq{
    \mathscr{L}_vT^{a_1\cdots a_k}{}_{b_1\cdots b_l}=\upsilon^c\nabla_cT^{a_1\cdots a_k}{}_{b_1\cdots b_l}-\sum_{i=1}^kT^{a_1\cdots c\cdots a_k}{}_{b_1\cdots b_l}\nabla_c\upsilon^{a_i}+\sum_{j=1}^lT^{a_1\cdots a_k}{}_{b_1\cdots c\cdots b_l}\nabla_{b_j}\upsilon^c
}

\subsection{度规结构的流形上的李导数}

下面我们讨论当我们的流形是一个广义黎曼流形的时候。就是我们的流形被赋予了一个明确的度规结构的时候的状态。我们规定一个明确的度规张量场之后,我们可以定义一个特别的Diff叫做Isometry:
\imp{Isometry的定义和说明}{
    我们给定一个流形自己到自己的微分同胚: $ \phi: M\to M $。我们称之为Isometry,当其满足下面的性质:
    \eq{
        (\phi^*(g_{ab}|_{\phi(p)}))|_p=g_{ab}|_p.
    }
    显然我们可以有一些性质:Isometry的逆映射也是一个Isometry。

    \textbf{注意:我们的Isometry的定义是【metric张量不发生变化】。我们是从张量的视角讨论的,等式左右两边讨论的是同一个流形M上的同一点p赋予同样的坐标系的结果。}
}
我们会发现,如果我们的流形上面有这样的一组Isometry会对应出一族矢量场。也就是\textbf{Killing Vector field}。
\imp{Killing Vector field的定义}{
    对于一个广义黎曼流形$ (M, g_{ab}) $我们存在矢量场$ \xi^a $。使得其给出的单参数微分同胚群的元素是Isometry。
    
    这个定义还有两个等价的表述:
    \itm{
        \pt{对于一个矢量场,满足$ \mathscr{L}_\xi g_{ab}=0\mathrm{~.} $ }
        \pt{对于一个矢量场,满足Killing Equation: 
        \eq{
            \nabla_a\xi_b+\nabla_b\xi_a=0\text{ , 或 }\nabla_{(a}\xi_{b)}=0\text{ , 或 }\nabla_a\xi_b=\nabla_{[a}\xi_{b]}.
        }
        }
        注意我们这里面使用的导数算符都是metric competible的。
    }
}
显然后面两种定义是等价的,因为我们可以使用李导数作用于张量场的计算性质:
\eq{
    0=\mathscr{L}_\xi g_{ab}=\xi^c\nabla_cg_{ab}+g_{cb}\nabla_a\xi^c+g_{ac}\nabla_b\xi^c=\nabla_a\xi_b+\nabla_b\xi_a\mathrm{~,}
}
注意:我们这里大量使用了metric competible的条件,也就是我们的度规张量场,可以随意的移入或者移出协变导数算符。

下面给出上面的定义后,我们还可以发现killing矢量场很多很多的性质。

\vspace{0.7em}
\textbf{Killing在适配坐标系}

如果存在一个坐标系,满足$ g_{\mu\nu} $的全部分量对于这个坐标系某一个分量$ x^i $的导数为0。那么这个坐标对应的基$ \partial_i (x)$构成的矢量场就是Killing矢量场。

当然,这必须是这个坐标系,刚好就是传说中的适配坐标系才可以捏!!

\vspace{0.7em}
\textbf{Killing矢量场和测地线}

测地线可以通过一个矢量场来描述(通过解微分方程 $ V^\mu = \partial x^\mu(a) / \partial a $ 其中a是affine parameter)那么,测地线方向的协变导数是:
\eq{
    T^a\nabla_a\left(T^b\xi_b\right)=0\mathrm{~,}
}
相当于Killing矢量场和测地线求道出来的矢量场的索并在测地线方向的导数为0。可以理解为Killing矢量场可以把测地线矢量场沿着测地线方向的变化kill掉!

\vspace{0.7em}
\textbf{Killing矢量场的数量}

对于一个流形来说,Killing矢量场存在运算关系:
\itm{
    \pt{Killing矢量场的线性组合:$ \alpha\xi^a+\beta\eta^a $ 也是Killing矢量场。}
    \pt{Killing矢量场的对易子也是Killing矢量场:$ [\xi,\eta]^a $ }
}
但是独立的Killing矢量场是有限的,我们给出下面的定理:
\thm{
    独立Killing矢量场数目

    对于一个n维的流形来说,上面最多有$ \frac{n(n+1)}{2} $个独立的Killing矢量场。如果一个流形有着所有的 $ \frac{n(n+1)}{2} $个独立的Killing矢量场,我们称之为\textbf{Maximally Symmetric Space}。

    这里特别说明的是,对于独立,我们的意思是在矢量空间的意义下是独立的。一个证明方法,就是可以取内积。和其他矢量内积并不为0那么必然就是独立的。
}

\subsection{怎么求出来Killing矢量场}
显然,最简单的方法就是找一堆适配坐标系。找到适配坐标系之后适配坐标系的基就是Killing矢量场。

\vspace{0.7em}
\textbf{二维欧几里得空间}

一个很显然的例子就是二维欧几里得空间$ (\mathbb{R}^2,\delta_{ab}) $我们不难发现,如果选用笛卡尔坐标系:
\eq{
    \mathbf{d}s^2=\mathbf{d}x^2+\mathbf{d}y^2
} 
发现这个坐标系就是个适配坐标系。给出了$ \partial_x $和$ \partial_y $这两个独立的矢量场(甚至他们是正交的)。同时我们换到极坐标系:
\eq{
    \mathbf{d}s^2=\mathbf{d}r^2+r^2\mathbf{d}\varphi^2
}
我们不难发现$ \partial_\varphi $也是一个Killing矢量场。我们把这个矢量场换到笛卡尔坐标系表示:$ \left(\partial/\partial\varphi\right)^a=-y\left(\partial/\partial x\right)^a+x\left(\partial/\partial y\right)^a. $ 

\vspace{0.7em}
\textbf{四维minkovski空间}
\pict{2025-03-15-14-38-11.png}{0.90}
我们不难发现Killing矢量场在四维minkovski时空之中generate出来的微分同胚对应的坐标变换矩阵是洛伦兹变换。

\section{流形赋予子流形结构}
我们会一般讨论每一个流形
