\newpage
\section{Metric Geometry and Geodesic}

\subsection{章节内容 take home message}

上一章我们知道了类比加速参考系我们描述引力(其实是,引力影响下物体的动力学)可以固定一个线元然后使用一个特殊的度规矩阵进行描述。下面我们讨论一般度规下面,粒子的动力学应该是什么样子的。
\rmk{注意,我们这里讨论的并不是引力场本身的动力学。或者度规本身的动力学。而是一个在一个特殊度规场之中的粒子的动力学。}

\hlr{Metric的基本介绍}

我们介绍了Metric的定义以及基本性质。【不再赘述】

并且我们知道,Metric可以通过有约束的embed在一个更加高维的时空进行构造!!(比如把一个球面embed在一个三维时空之中)这个操作几个重要的发现是:
\begin{itemize}
  \item 不同的metric可以描述同样的时空,仅仅是坐标系不同。
  \item 不同的embed方式可以得到同样的时空的不同坐标系的metric;同时不同的embed方式也可能得到不同的时空metric。
  \item 有些时空metric只能embed在minkowski之中产生;而有些只能Euclidean!
\end{itemize}
\bigskip

\hlr{Geodesic Equation作为运动方程的构造}

我们使用$ \tau $作为参数化,研究有质量粒子的作用量
\begin{align}
  S_0=-m\int d\tau\mathrm{~.}
\end{align}
可以得到有质量粒子的运动方程。也就是
\begin{align}
  \Large\frac{d^2x^\mu}{d\tau^2}+\Gamma_{\nu\lambda}^\mu\frac{dx^\nu}{d\tau}\frac{dx^\lambda}{d\tau}=0. 
\end{align}
最重要的是研究了Christoffel symbol的定义以及性质。发现这个量并非tensor但是geodesic equation是一个tensor equation。我们给出Christoffel symbol的变换关系:
\begin{align}
  \Gamma_{\beta\gamma}^\alpha=\Gamma_{\nu\lambda}^\mu\frac{\partial y^\alpha}{\partial x^\mu}\frac{\partial x^\nu}{\partial y^\beta}\frac{\partial x^\lambda}{\partial y^\gamma}+\frac{\partial y^\alpha}{\partial x^\mu}\frac{\partial^2x^\mu}{\partial y^\beta\partial y^\gamma},\Gamma_{\beta\gamma}^\alpha=J_\mu^\alpha J_\beta^\nu J_\gamma^\lambda\Gamma_{\nu\lambda}^\mu+J_\mu^\alpha\partial_\beta J_\gamma^\mu.
\end{align}
以及整个geodesic equation的变换关系:
\begin{align}
  \frac{d^2y^\alpha}{d\tau^2}+\Gamma_{\beta\gamma}^\alpha\frac{dy^\beta}{d\tau}\frac{dy^\gamma}{d\tau}=\frac{\partial y^\alpha}{\partial x^\mu}\left[\frac{d^2x^\mu}{d\tau^2}+\Gamma_{\nu\lambda}^\mu\frac{dx^\nu}{d\tau}\frac{dx^\lambda}{d\tau}\right]\quad.
\end{align}

\bigskip
\hlr{计算辅助作用量}

我们定义一个新的作用量辅助计算,这个作用量【和参数化有关!】但是【只会给出标准测地线方程】:
\begin{align}
  \mathcal{L}=\frac{1}{2}g_{\alpha\beta}\frac{dx^\alpha}{d\lambda}\frac{dx^\beta}{d\lambda}\\
  S_1[x]=\int d\lambda\left.\mathcal{L}\right.
\end{align}
\begin{itemize}
  \item 这个作用量对于有质量粒子,对于所有Alfine parameter都是有意义的并且等效的「虽然我们一般使用固有时参数化」
  \item 这个作用量可以描述无质量粒子;对于无质量粒子,这个作用量对于所有参数化都是有意义的并且等效的「虽然我们一般使用和动量匹配的参数化」。并且无质量粒子,无论什么参数化,标准测地线方程都是满足的,哪怕方程换了一个参数化,我们永远可以证明「只要新的参数化还是满足null的条件」这个多出来的部分是0。
\end{itemize}
同时一个重要的结论是,我们的测地线方程能够导出这个作用量的lagrangian是一个EoM的守恒量。
\begin{align}
  \frac{d^2x^\gamma}{d\lambda^2}+\Gamma_{\alpha\beta}^\gamma\frac{dx^\alpha}{d\lambda}\frac{dx^\beta}{d\lambda}=0\quad\Rightarrow\quad\frac{d}{d\lambda}\left(g_{\alpha\beta}\frac{dx^\alpha}{d\lambda}\frac{dx^\beta}{d\lambda}\right)=0\quad.
\end{align}
这个守恒量也可以理解为这个Action是对于参数化平移不变的。也就是$ \lambda \to \lambda +a $变换下参数化不变!!「可以理解为参数化一维流形的平移不变形」


\bigskip
\hlr{两个作用量原理的关系}

数学上这两个作用量原理可以通过一个生成函数互相联系起来。这个生成函数是【与参数化无关的】!!

\bigskip
\hlr{Alfine Parameter的选择}

\begin{itemize}
  \item 测地线方程对于参数化是敏感的,只有使用Alfine parameter这个方程才是有物理意义的「也就是作为运动方程的意义」。如果用其他参数化需要加一项。
  \item 作用量$ S_0 $对于参数化无关;作用量$ S_1 $和参数化有关,并且我们只有使用
    \begin{itemize}
    \item 有质量粒子使用Alfine parameter
    \item 无质量粒子使用切矢量正好给出动量的参数化
  \end{itemize}
\end{itemize}


\bigskip
\hlr{算点例子;不同metric下面的geodesic equation}

我们会发现一些特殊形式的metric的geodesic 是有某些特殊样子的!请看讲义!

\subsection{Questions and discussions}

\question{我们既然知道测地线方程是和Reparametrization有关的,那么三种给出测地线方程的Lagrangian的写法哪些和reparametrization有关?}

上面正文已经回答\qed 


\question{我们知道测地线方程是和Parameter有关的,那么我们最开始是怎么得到这个方程的?(sec:2.7)}

我们最开始是计算一个任意参考系下面的运动方程得到的。同时我们也证明对于一般参考系系统我们使用$ S = -m \int d\tau $的action进行变分,并同时【使用$ \tau $进行参数化】得到的。无论哪种其实我们都下意识的使用了一个固定的参数化也就是$ \tau $

2.7开始就是需要讨论这个参数化如果选择另一个参数会发生什么。
\qed 


