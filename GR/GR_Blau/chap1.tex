

\section{EEP from gravity to Geometry}

\subsection{章节内容 take home message}

\hlr{Einstein 等效原理}

本章之中,我们根据Einstein Equvalence Principle 我们知道了“加速参考系”和“生活在引力之中”的物体的参考系是有着相似性的!!并且对于有引力的体系,我们可以找到一个参考系让情况locally和没有引力是一样的【这个仅仅是locally,因为弯曲时空就是不一样的「之后我们会知道会有潮汐力!或者说曲率张量的影响!我们需要选一个足够local的区间来忽略这个影响」】
\bigskip

\imp{EEP}{
一种表述是:At every space-time point in an arbitrary gravitational field it is possible  to choose a locally inertial (or freely falling) coordinate system such that,  within a sufficiently small region of this point, the laws of nature take the same form as in unaccelerated Cartesian coordinate systems in the absence of gravitation.2
}

所以我们为了研究狭义相对论意义下面的引力理论,需要先研究狭义相对论之中的加速参考系的性质【而非仅仅是洛伦兹变换相连的平直的惯性参考系的性质】

\bigskip
\hlr{狭义相对论基本概念}

具体内容请查书

我们首先复习了狭义相对论的基本概念【Minkowski时空;洛伦兹变换;能动量四矢量;自由粒子的运动;Action】;然后研究了匀加速运动在狭义相对论性质下的结果,我们发现匀加速的的物体自身的参考系是一个叫Rindler Coordinate的参考系,这个也是一个Minkowski时空的参考系。

最重要的是,由于我们认为都是Minkowski时空。我们的线元的大小应该是绝对不变的。那么这个时空下就会有一个特殊的Metric度规矩阵。我们知道,对于一个加速的参考系,我们可以通过metric来描述参考系的性质!!

\bigskip
\hlr{Minkowski时空下一般坐标变换}

我们进一步意识到,根据等效原理,引力也应该是用一样的方式【固定线元,考虑一些特殊的度规】进行描述的!!我们首先研究了general coordinate transformation在Minkowski时空下面的情况。我们发现一个很显然但是重要的事实:4加速度虽然是洛伦兹4矢量,但是对于一般的坐标变换并不是4矢量!!

我们进一步研究加速度怎么在一般坐标变换下面进行变换的,发现其按照张量变换之外还需要多加一个量也就是Christoffel symbol!!并且这个量和metric的导数是有关系的!!并且加入Chris Symbol之后加速度就变成了一个general的tensor!!!

这个时候我们的没有相互作用的运动方程就变成了:
\begin{align}
  \frac{d^2x^\mu}{d\tau^2}+\Gamma_{\nu\lambda}^\mu\frac{dx^\nu}{d\tau}\frac{dx^\lambda}{d\tau}=0.
\end{align}
这个方程叫做Geodesic Equation!

\imp{General Covariance 初探}{
  我们发现Geodesic Equation给出了一个在任意坐标变换下形式不变的方程。这就满足了General Covariance「在坐标系变换下方程形式保持不变」的要求!!!
}


\bigskip
\hlr{各种坐标系以及metric}

最后一步我们研修了一些Minkowski时空以及其他弯曲时空下面坐标系的例子。


\subsection{Questions and discussions}

\question{指标的书写规则是什么?}

\textbf{指标量的书写:}
所有的指标都是有前后顺序的,因为指标的存在意味着可能的张量积的基的存在【注意是可能的,只有张量才是一定】。由于张量积是不能改变顺序的,所以我们指标都是有前后顺序的,不论是上指标还是下指标。

\textbf{指标的求和规则:}指标求和需要满足两个规则,满足这两个规则的指标自动求和:
\begin{enumerate}
  \item 指标一个上一个下
  \item 指标字母相同
\end{enumerate}

\textbf{指标约定俗成书写记号:}
对于一些有指标的量,我们存在一些约定俗成的记号写法:
\begin{enumerate}
  \item 物理量由于其协变关系,我们会固定是上指标还是下指标;当然我们也可以通过度规张量升降指标,modify物理量改变其协变关系。我们一般约定通过度规张量升降变换前后物理量使用同一个字母表示。但是物理上,$ x^\mu, A^\mu, \partial_\mu$ 这些指标都是固定的。
  \item 对于度规张量,我们认为度规张量都是下指标的。上指标的度规张量我们定义为下指标度规张量的逆矩阵。我们不难证明度规张量的逆矩阵正好按照Jacobi的逆进行变换。
  \item Jacobi矩阵我们定义$ \tensor{J}{^\mu_a} = \displaystyle\frac{\partial x^\mu}{\partial y^a} $我们一般用不同语言的字母表示不同的参考系。对于Jacobi的逆矩阵我们一般用同样的符号进行表示,但是我们知道是逆变换的矩阵:$ J^{-1 a}{}_\mu =  \tensor{J}{^a_\mu} = \displaystyle\frac{\partial y^a}{\partial x^\mu} $并满足: $ \tensor{J}{^\mu_a}\tensor{J}{^a_\nu} = \tensor{\delta}{^\mu_\nu} $ 
  \item 我们把指标形式写成矩阵形式的时候常常遇见需要写一个转置。量的转置之后和之前是两个量,所以应该使用两个记号进行标记。转置也就是把两个基的前后张量积顺序进行交换,不改变其协变性质。也就是说,转置前后的的量满足:
    $ \tensor{Tt}{_\nu^\mu} = \tensor{T}{^\mu_\nu} $  
    
  \item 我们考虑逆矩阵和升降指标的关系。对于一个坐标变换会有【也就是变换并不改变metric】:
    \begin{equation}
      \tensor{\Lambda}{^a_\mu}\tensor{\Lambda}{^b_\nu} g_{ab} = g_{\mu\nu} 
      \label{eq:generalcoordinatetrans}
    \end{equation}
    对于这样的变换我们会发现升降指标之后的张量和逆矩阵满足关系【其实这个并不是逆矩阵,因为需要第二个指标和下一个的第一个指标求和才矩阵乘法,所以其实是逆矩阵再转制!】:
    \begin{equation}
      \Lambda^{-1 \sigma}{}_c = \Lambda_c{}^\sigma 
      \label{eq:liftandinverse}
    \end{equation}
    由于我们的Jacobi的逆矩阵定义为:$ J^{-1 a}{}_\mu =  \tensor{J}{^a_\mu} $所以也可以有一个等号。但是,我们不希望这样计算,太混淆了!!!
\end{enumerate}

\bigskip
\question{对于两个指标的量「不一定是张量」我们怎么翻译成为矩阵形式?}

指标的方程写作矩阵形式除了有的时候手动 or 输入电脑进行计算没有其他意义。【这个问题目前似乎没有意义,永远不要用矩阵来研究!】
\qed


\bigskip
\question{参考系的概念是怎么存在物理意义的??我们怎么确定一个观察者的随动参考系??}

Naively, 我们说一个参考系是belong to一个观察者的【也就是和观察者随动的】是说,这个参考系之中如果有一种运动是某个【行为很像空间】的坐标是固定【为0的】,某一个【行为很像时间】的坐标是正比于固有时间进行变化的。

但并不是每一个参考系都可以找到这样的观察者是随动的!!

\tip{GR之中广义参考系变换的物理意义}{
  在GR之中我们会进行任意广义的参考系变换。但是这个参考系变换并不一定具有物理意义。物理意义都是我们赋予的。
\bigskip

  对于Rindler Spacetime coordinate,很幸运,我们发现所有匀加速运动物体在这个参考系下都是静止的。所以我们赋予其物理意义为:「和所有匀加速运动物体随动的参考系」【note显然这个其实也不是我们经典意义上的加速参考系,因为我们牛顿力学不可能找到一个加速参考系看起来所有任意加速度加速运动的物体都是静止的。】
\bigskip

  但是我们手动赋予其这个「加速参考系」的意义。对于其具体概念我们需要回溯上面的话。这里的加速并不是某一个物体的随动参考系。而是上面「」之中描述的物理意义
\bigskip

  对于一般参考系,我们并不一定能够找到物理意义。
}

但是,对于每一个观察者。我们一定可以找到一个唯一的参考系是和其随动的!!【否则就不是物理的了】但我目前也不知道怎么写出这个参考系的坐标变换。
\qed 


\bigskip
\question{
  对于一个在某个已经知道metric的坐标系下运动的粒子,我们怎么给出这个坐标系和粒子的随动坐标系的变换关系??Rindler参考系是不是某种粒子的随动参考系??
}

我们首先知道,对于任意参考系下面一个运动轨迹的粒子,我们可以计算其随动参考系的时间。因为,我们知道$ d\tau^2 = - g_{\mu\nu}dx^\mu dx^\nu $
