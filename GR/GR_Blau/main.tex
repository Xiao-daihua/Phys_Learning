% main.tex - 科研项目学习笔记模板
% !TeX root = main.tex
%%%%%%%%%%%%%%%%%%%%%%%%%%%%%% DOCUMENT 
\documentclass[12pt]{report}

%%%%%%%%%%%%%%%%%%%%%%%%%%%%%% PACKAGES

% 中文支持(XeLaTeX 编译)
\usepackage[UTF8]{ctex}
\usepackage{xeCJKfntef} 

% \setCJKmainfont{HanziPen SC}
% \setmainfont{HanziPen SC}


% 页面设置
\usepackage[a4paper, left=20mm, right=20mm, top=15mm, bottom=15mm]{geometry}

\PassOptionsToPackage{dvipsnames,svgnames,x11names}{xcolor}
\usepackage{xcolor}


% 数学环境及符号
\usepackage{amsmath, amssymb, amsfonts, amsthm,amsopn}
\usepackage{tensor}              % 张量指标管理
\usepackage{mathtools}           % amsmath增强
\usepackage{physics}             % 物理公式快捷命令
             % Dirac符号
\usepackage{bbold}               % 数学黑体
\usepackage{dsfont}              % 另一种数字体
\usepackage[mathscr]{eucal}     % 花体字母
\usepackage{tensor}              % 张量指标管理
\usepackage{simpler-wick}       % Wick记号
\usepackage{mathrsfs}            % 另一种花体字母

% 颜色与图形相关
\usepackage{graphicx}           % 插图支持
\usepackage{float}              % 浮动体控制
\usepackage{tikz}               % 绘图库
\usetikzlibrary{math}           % tikz数学扩展
\usepackage{geometry}
% 表格与列表
\usepackage{makecell}           % 表格多行换行
\usepackage{multicol}           % 多栏排版
\usepackage{colortbl}           % 表格颜色
\usepackage{enumitem}           % 列表自定义

% 其他辅助
\usepackage{framed}             % 有边框环境
\usepackage{tcolorbox}          % 灵活盒子环境
\tcbuselibrary{breakable}       % 盒子内容分页
\usepackage{thmtools}           % 定理环境管理
\usepackage{thm-restate}        % 定理重述
\usepackage{showlabels}         % 显示标签,调试用(完成后可注释)
\usepackage[normalem]{ulem}     % 下划线、删除线
\usepackage{hyperref}           % 超链接(最后加载)
\usepackage{cleveref}           % 智能引用(紧跟hyperref)
\usepackage{soul}

% 自定义宏包
\usepackage{macros}

% 一个中文可以高亮的包
\usepackage{cjkhl}
\definecolor{lightblue}{rgb}{.8,.8,1}

%%%%%%%%%%%%%%%%%%%%%%%%%%%%%% 自定义命令
\newcommand{\tml}{Teichmüller space}
\newcommand{\hil}{Hilbert space}
\newcommand{\mtc}{Modular Tensor Category}

%%%%%%%%%%%%%%%%%%%%%%%%%%%%%% BEGINNING OF THE DOCUMENT

\begin{document}

\title{\boldmath Blau GR Reading Note \\ \large{广义相对论读书笔记}}
\author{X. D. H.}

\maketitle

\begin{abstract}
  本文是阅读Blau的广义相对论的简单学习笔记捏!!2025年秋天-冬天的阅读计划是Blau的 GR lecture note的A-E五个主要部分!我希望能够通读完成这些内容捏!
\end{abstract}

\tableofcontents
\chapter{Setion A: Physics in a gravitational field and tensor calculus}


\section{EEP from gravity to Geometry}

\subsection{章节内容 take home message}

\hlr{Einstein 等效原理}

本章之中,我们根据Einstein Equvalence Principle 我们知道了“加速参考系”和“生活在引力之中”的物体的参考系是有着相似性的!!并且对于有引力的体系,我们可以找到一个参考系让情况locally和没有引力是一样的【这个仅仅是locally,因为弯曲时空就是不一样的「之后我们会知道会有潮汐力!或者说曲率张量的影响!我们需要选一个足够local的区间来忽略这个影响」】
\bigskip

\imp{EEP}{
一种表述是:At every space-time point in an arbitrary gravitational field it is possible  to choose a locally inertial (or freely falling) coordinate system such that,  within a sufficiently small region of this point, the laws of nature take the same form as in unaccelerated Cartesian coordinate systems in the absence of gravitation.2
}

所以我们为了研究狭义相对论意义下面的引力理论,需要先研究狭义相对论之中的加速参考系的性质【而非仅仅是洛伦兹变换相连的平直的惯性参考系的性质】

\bigskip
\hlr{狭义相对论基本概念}

具体内容请查书

我们首先复习了狭义相对论的基本概念【Minkowski时空;洛伦兹变换;能动量四矢量;自由粒子的运动;Action】;然后研究了匀加速运动在狭义相对论性质下的结果,我们发现匀加速的的物体自身的参考系是一个叫Rindler Coordinate的参考系,这个也是一个Minkowski时空的参考系。

最重要的是,由于我们认为都是Minkowski时空。我们的线元的大小应该是绝对不变的。那么这个时空下就会有一个特殊的Metric度规矩阵。我们知道,对于一个加速的参考系,我们可以通过metric来描述参考系的性质!!

\bigskip
\hlr{Minkowski时空下一般坐标变换}

我们进一步意识到,根据等效原理,引力也应该是用一样的方式【固定线元,考虑一些特殊的度规】进行描述的!!我们首先研究了general coordinate transformation在Minkowski时空下面的情况。我们发现一个很显然但是重要的事实:4加速度虽然是洛伦兹4矢量,但是对于一般的坐标变换并不是4矢量!!

我们进一步研究加速度怎么在一般坐标变换下面进行变换的,发现其按照张量变换之外还需要多加一个量也就是Christoffel symbol!!并且这个量和metric的导数是有关系的!!并且加入Chris Symbol之后加速度就变成了一个general的tensor!!!

这个时候我们的没有相互作用的运动方程就变成了:
\begin{align}
  \frac{d^2x^\mu}{d\tau^2}+\Gamma_{\nu\lambda}^\mu\frac{dx^\nu}{d\tau}\frac{dx^\lambda}{d\tau}=0.
\end{align}
这个方程叫做Geodesic Equation!

\imp{General Covariance 初探}{
  我们发现Geodesic Equation给出了一个在任意坐标变换下形式不变的方程。这就满足了General Covariance「在坐标系变换下方程形式保持不变」的要求!!!
}


\bigskip
\hlr{各种坐标系以及metric}

最后一步我们研修了一些Minkowski时空以及其他弯曲时空下面坐标系的例子。


\subsection{Questions and discussions}

\question{指标的书写规则是什么?}

\textbf{指标量的书写:}
所有的指标都是有前后顺序的,因为指标的存在意味着可能的张量积的基的存在【注意是可能的,只有张量才是一定】。由于张量积是不能改变顺序的,所以我们指标都是有前后顺序的,不论是上指标还是下指标。

\textbf{指标的求和规则:}指标求和需要满足两个规则,满足这两个规则的指标自动求和:
\begin{enumerate}
  \item 指标一个上一个下
  \item 指标字母相同
\end{enumerate}

\textbf{指标约定俗成书写记号:}
对于一些有指标的量,我们存在一些约定俗成的记号写法:
\begin{enumerate}
  \item 物理量由于其协变关系,我们会固定是上指标还是下指标;当然我们也可以通过度规张量升降指标,modify物理量改变其协变关系。我们一般约定通过度规张量升降变换前后物理量使用同一个字母表示。但是物理上,$ x^\mu, A^\mu, \partial_\mu$ 这些指标都是固定的。
  \item 对于度规张量,我们认为度规张量都是下指标的。上指标的度规张量我们定义为下指标度规张量的逆矩阵。我们不难证明度规张量的逆矩阵正好按照Jacobi的逆进行变换。
  \item Jacobi矩阵我们定义$ \tensor{J}{^\mu_a} = \displaystyle\frac{\partial x^\mu}{\partial y^a} $我们一般用不同语言的字母表示不同的参考系。对于Jacobi的逆矩阵我们一般用同样的符号进行表示,但是我们知道是逆变换的矩阵:$ J^{-1 a}{}_\mu =  \tensor{J}{^a_\mu} = \displaystyle\frac{\partial y^a}{\partial x^\mu} $并满足: $ \tensor{J}{^\mu_a}\tensor{J}{^a_\nu} = \tensor{\delta}{^\mu_\nu} $ 
  \item 我们把指标形式写成矩阵形式的时候常常遇见需要写一个转置。量的转置之后和之前是两个量,所以应该使用两个记号进行标记。转置也就是把两个基的前后张量积顺序进行交换,不改变其协变性质。也就是说,转置前后的的量满足:
    $ \tensor{Tt}{_\nu^\mu} = \tensor{T}{^\mu_\nu} $  
    
  \item 我们考虑逆矩阵和升降指标的关系。对于一个坐标变换会有【也就是变换并不改变metric】:
    \begin{equation}
      \tensor{\Lambda}{^a_\mu}\tensor{\Lambda}{^b_\nu} g_{ab} = g_{\mu\nu} 
      \label{eq:generalcoordinatetrans}
    \end{equation}
    对于这样的变换我们会发现升降指标之后的张量和逆矩阵满足关系【其实这个并不是逆矩阵,因为需要第二个指标和下一个的第一个指标求和才矩阵乘法,所以其实是逆矩阵再转制!】:
    \begin{equation}
      \Lambda^{-1 \sigma}{}_c = \Lambda_c{}^\sigma 
      \label{eq:liftandinverse}
    \end{equation}
    由于我们的Jacobi的逆矩阵定义为:$ J^{-1 a}{}_\mu =  \tensor{J}{^a_\mu} $所以也可以有一个等号。但是,我们不希望这样计算,太混淆了!!!
\end{enumerate}

\bigskip
\question{对于两个指标的量「不一定是张量」我们怎么翻译成为矩阵形式?}

指标的方程写作矩阵形式除了有的时候手动 or 输入电脑进行计算没有其他意义。【这个问题目前似乎没有意义,永远不要用矩阵来研究!】
\qed


\bigskip
\question{参考系的概念是怎么存在物理意义的??我们怎么确定一个观察者的随动参考系??}

Naively, 我们说一个参考系是belong to一个观察者的【也就是和观察者随动的】是说,这个参考系之中如果有一种运动是某个【行为很像空间】的坐标是固定【为0的】,某一个【行为很像时间】的坐标是正比于固有时间进行变化的。

但并不是每一个参考系都可以找到这样的观察者是随动的!!

\tip{GR之中广义参考系变换的物理意义}{
  在GR之中我们会进行任意广义的参考系变换。但是这个参考系变换并不一定具有物理意义。物理意义都是我们赋予的。
\bigskip

  对于Rindler Spacetime coordinate,很幸运,我们发现所有匀加速运动物体在这个参考系下都是静止的。所以我们赋予其物理意义为:「和所有匀加速运动物体随动的参考系」【note显然这个其实也不是我们经典意义上的加速参考系,因为我们牛顿力学不可能找到一个加速参考系看起来所有任意加速度加速运动的物体都是静止的。】
\bigskip

  但是我们手动赋予其这个「加速参考系」的意义。对于其具体概念我们需要回溯上面的话。这里的加速并不是某一个物体的随动参考系。而是上面「」之中描述的物理意义
\bigskip

  对于一般参考系,我们并不一定能够找到物理意义。
}

但是,对于每一个观察者。我们一定可以找到一个唯一的参考系是和其随动的!!【否则就不是物理的了】但我目前也不知道怎么写出这个参考系的坐标变换。
\qed 


\bigskip
\question{
  对于一个在某个已经知道metric的坐标系下运动的粒子,我们怎么给出这个坐标系和粒子的随动坐标系的变换关系??Rindler参考系是不是某种粒子的随动参考系??
}

我们首先知道,对于任意参考系下面一个运动轨迹的粒子,我们可以计算其随动参考系的时间。因为,我们知道$ d\tau^2 = - g_{\mu\nu}dx^\mu dx^\nu $

\newpage
\section{Metric Geometry and Geodesic}

\subsection{章节内容 take home message}

上一章我们知道了类比加速参考系我们描述引力(其实是,引力影响下物体的动力学)可以固定一个线元然后使用一个特殊的度规矩阵进行描述。下面我们讨论一般度规下面,粒子的动力学应该是什么样子的。
\rmk{注意,我们这里讨论的并不是引力场本身的动力学。或者度规本身的动力学。而是一个在一个特殊度规场之中的粒子的动力学。}

\hlr{Metric的基本介绍}

我们介绍了Metric的定义以及基本性质。【不再赘述】

并且我们知道,Metric可以通过有约束的embed在一个更加高维的时空进行构造!!(比如把一个球面embed在一个三维时空之中)这个操作几个重要的发现是:
\begin{itemize}
  \item 不同的metric可以描述同样的时空,仅仅是坐标系不同。
  \item 不同的embed方式可以得到同样的时空的不同坐标系的metric;同时不同的embed方式也可能得到不同的时空metric。
  \item 有些时空metric只能embed在minkowski之中产生;而有些只能Euclidean!
\end{itemize}
\bigskip

\hlr{Geodesic Equation作为运动方程的构造}

我们使用$ \tau $作为参数化,研究有质量粒子的作用量
\begin{align}
  S_0=-m\int d\tau\mathrm{~.}
\end{align}
可以得到有质量粒子的运动方程。也就是
\begin{align}
  \Large\frac{d^2x^\mu}{d\tau^2}+\Gamma_{\nu\lambda}^\mu\frac{dx^\nu}{d\tau}\frac{dx^\lambda}{d\tau}=0. 
\end{align}
最重要的是研究了Christoffel symbol的定义以及性质。发现这个量并非tensor但是geodesic equation是一个tensor equation。我们给出Christoffel symbol的变换关系:
\begin{align}
  \Gamma_{\beta\gamma}^\alpha=\Gamma_{\nu\lambda}^\mu\frac{\partial y^\alpha}{\partial x^\mu}\frac{\partial x^\nu}{\partial y^\beta}\frac{\partial x^\lambda}{\partial y^\gamma}+\frac{\partial y^\alpha}{\partial x^\mu}\frac{\partial^2x^\mu}{\partial y^\beta\partial y^\gamma},\Gamma_{\beta\gamma}^\alpha=J_\mu^\alpha J_\beta^\nu J_\gamma^\lambda\Gamma_{\nu\lambda}^\mu+J_\mu^\alpha\partial_\beta J_\gamma^\mu.
\end{align}
以及整个geodesic equation的变换关系:
\begin{align}
  \frac{d^2y^\alpha}{d\tau^2}+\Gamma_{\beta\gamma}^\alpha\frac{dy^\beta}{d\tau}\frac{dy^\gamma}{d\tau}=\frac{\partial y^\alpha}{\partial x^\mu}\left[\frac{d^2x^\mu}{d\tau^2}+\Gamma_{\nu\lambda}^\mu\frac{dx^\nu}{d\tau}\frac{dx^\lambda}{d\tau}\right]\quad.
\end{align}

\bigskip
\hlr{计算辅助作用量}

我们定义一个新的作用量辅助计算,这个作用量【和参数化有关!】但是【只会给出标准测地线方程】:
\begin{align}
  \mathcal{L}=\frac{1}{2}g_{\alpha\beta}\frac{dx^\alpha}{d\lambda}\frac{dx^\beta}{d\lambda}\\
  S_1[x]=\int d\lambda\left.\mathcal{L}\right.
\end{align}
\begin{itemize}
  \item 这个作用量对于有质量粒子,对于所有Alfine parameter都是有意义的并且等效的「虽然我们一般使用固有时参数化」
  \item 这个作用量可以描述无质量粒子;对于无质量粒子,这个作用量对于所有参数化都是有意义的并且等效的「虽然我们一般使用和动量匹配的参数化」。并且无质量粒子,无论什么参数化,标准测地线方程都是满足的,哪怕方程换了一个参数化,我们永远可以证明「只要新的参数化还是满足null的条件」这个多出来的部分是0。
\end{itemize}
同时一个重要的结论是,我们的测地线方程能够导出这个作用量的lagrangian是一个EoM的守恒量。
\begin{align}
  \frac{d^2x^\gamma}{d\lambda^2}+\Gamma_{\alpha\beta}^\gamma\frac{dx^\alpha}{d\lambda}\frac{dx^\beta}{d\lambda}=0\quad\Rightarrow\quad\frac{d}{d\lambda}\left(g_{\alpha\beta}\frac{dx^\alpha}{d\lambda}\frac{dx^\beta}{d\lambda}\right)=0\quad.
\end{align}
这个守恒量也可以理解为这个Action是对于参数化平移不变的。也就是$ \lambda \to \lambda +a $变换下参数化不变!!「可以理解为参数化一维流形的平移不变形」


\bigskip
\hlr{两个作用量原理的关系}

数学上这两个作用量原理可以通过一个生成函数互相联系起来。这个生成函数是【与参数化无关的】!!

\bigskip
\hlr{Alfine Parameter的选择}

\begin{itemize}
  \item 测地线方程对于参数化是敏感的,只有使用Alfine parameter这个方程才是有物理意义的「也就是作为运动方程的意义」。如果用其他参数化需要加一项。
  \item 作用量$ S_0 $对于参数化无关;作用量$ S_1 $和参数化有关,并且我们只有使用
    \begin{itemize}
    \item 有质量粒子使用Alfine parameter
    \item 无质量粒子使用切矢量正好给出动量的参数化
  \end{itemize}
\end{itemize}


\bigskip
\hlr{算点例子;不同metric下面的geodesic equation}

我们会发现一些特殊形式的metric的geodesic 是有某些特殊样子的!请看讲义!

\subsection{Questions and discussions}

\question{我们既然知道测地线方程是和Reparametrization有关的,那么三种给出测地线方程的Lagrangian的写法哪些和reparametrization有关?}

上面正文已经回答\qed 


\question{我们知道测地线方程是和Parameter有关的,那么我们最开始是怎么得到这个方程的?(sec:2.7)}

我们最开始是计算一个任意参考系下面的运动方程得到的。同时我们也证明对于一般参考系系统我们使用$ S = -m \int d\tau $的action进行变分,并同时【使用$ \tau $进行参数化】得到的。无论哪种其实我们都下意识的使用了一个固定的参数化也就是$ \tau $

2.7开始就是需要讨论这个参数化如果选择另一个参数会发生什么。
\qed 




\newpage
\section{Geodesic and Motion in Gravity}\label{sec:Geodesic and Motion in Gravity} % (fold)


\subsection{Euler Lagrange方程的结论}
\hlr{ S1 作用量导出的物理}

对于一个作用量原理描述的粒子,我们会知道如果有对称性那么会给出一个在运动方程上的守恒量。首先,一个很显然的对称性是,如果lagrangian不显含一个坐标「循环坐标」,那么变分原理告诉我们这个坐标对应的共轭动量是守恒的。或者我们说存在一个守恒的geodesic equation一次积分结果。
\begin{align}
  p_1=\partial\mathcal{L}/\partial\dot{x}^1=g_{1\nu}\dot{x}^\nu
\end{align}

比如不依赖时间坐标的metric自然会给出守恒量:
\begin{align}
  \large p_0=g_{0\nu}\dot{x}^\nu\equiv-E 
\end{align}
我们把这个守恒量称为能量「注意能量是一个Hamiltonian力学的概念,我们现在还没有引入广相之中的Hamiltonian力学」但是我们不妨这么定义这个守恒量捏,并且我们已经暗含了【propertime是时间方向】这个事实。

\bigskip
\hlr{Metric形式瞪眼Chris Symbol}

有一些metric的形式,我们可以直接看出Chris Symbol的形式!!比如对角metric...


\bigskip
\hlr{使用Euler-Lagrange方程给出Chris Symbol}

不要笨笨的用定义求Chris Symbol!用Euler Lagrange方程瞪眼出来!!

\subsection{引力中的粒子Lagrangian:对称性以及守恒量初探}

\hlr{守恒荷和Killing Vector:初探}

这一章的探讨和引入是混乱并且绕远的!!!请不要试图理解((但是结论是正确的,可以验证的((我们概括一个结论,我们就当突然说了这样一个结论,我们验证了一下就是对的!!
\begin{itemize}
  \item 如果一个矢量场$ V^\mu $满足Killing Equation $ \delta_V g = 0 $或者是$ L_V g = 0 $那么这个矢量场generate一个坐标变换。这个坐标变换是一个symmetry。这个变换写出来的形式是:
\begin{align}
  \delta x^\alpha &=\epsilon V^\alpha(x)\\ 
  \delta g_{\mu\nu} & = 0
\end{align}
  \item 使用noether's theorem这个变换会给出一个守恒量$ Q_V = p_\mu V^\mu $ 也就是粒子沿着$ V^\mu $方向的动量是守恒的。【本书中并没有仔细说明,我也不建议在这里搞清楚】
\end{itemize}

\tip{守恒量公式}{
  如果一个连续变换导致了Lagrangian最后只差了一个全微分的话,这就是一个对称性变换。守恒量是:
  \begin{equation}
    \begin{aligned}
      L\to L+\epsilon\frac{dF}{dt},\\ 
      Q = \frac{\partial L}{\partial \dot{q}^i}\delta q^i - F
    \end{aligned}
    \label{eq:conservedquantity}
  \end{equation}
}

btw: 考虑弦论里面的polyakov action也是可以考虑这么一个坐标变换的协变:
\begin{align}
  \begin{aligned}&\delta X^\mu=a^\mu{}_\nu X^\nu+b^\mu\\&\delta h_{\alpha\beta}=0\end{aligned}
\end{align}
如果我们考虑这么个变换,需要变换的向量正好满足Killing equation才能保证$ \delta h_{\mu\nu} = 0 $。


\subsection{牛顿引力极限}
\bigskip
\hlr{牛顿引力极限!!}

我们讨论GR和牛顿引力关系我们需要有三个假设:
\begin{itemize}
  \item 弱弱的场!
  \item 慢慢的运动
  \item 并且引力场不随着时间很快的变换「stationary field」
\end{itemize}
给出重要的结论就是;
\begin{align}
  \large g_{00}=-(1+2\phi)
\end{align}
一个合理的「满足Einstein 场方程」并且球形引力的就是:
\begin{align}
  ds^2=-\left(1-\frac{2G_NM}{c^2r}\right)c^2dt^2+\left(1-\frac{2G_NM}{c^2r}\right)^{-1}dr^2+r^2d\Omega^2. 
\end{align}

\bigskip
\hlr{复习Rindler时空}

也就是复习Rindler时空并且知道了很多奇奇怪怪的坐标系!

\subsection{引力红移以及引力之下无质量粒子运动}
\bigskip 
\hlr{Gravitational Redshift 推导1}

我们算一下发出两个光信号,收到的固有时差多少。比对一下就可以得到频率区别。


\bigskip
\hlr{Gravitational Redshift 推导2}

我们仔细说明一下书中Argument 2的推导。我们首先研究光波相对论描述性质:
\imp{光波描述性质}{
我们研究相对论语境下面光的行为我们可以知道。【选定一个坐标系】:
\begin{enumerate}
  \item 然后光的运动可以通过一个Null Vector进行描述$ k^a = (\omega,k) $,满足$ k^a k_a= 0 $
  \item 量子力学告诉我们这个null Vector正比于光的4-动量:$ p^a = \hbar k^a $。
  \item 同时因为波矢指向光子运动的方向,我们一般使用下面的参数化$ k^a = \displaystyle\frac{d}{d \lambda} x^a(\lambda)$这个$ x^a(\lambda) $描述了光子的运动。
\end{enumerate}
}
在上方基础性质的基础上我们讨论,这个参考系之中不同的观察者测量到的光波频率。我们推广狭义相对论的结果,认为,对于所有的参考者来说光的频率是:
\begin{equation}
  \omega=-u^\alpha k_\alpha\mathrm{~.}
  \label{eq:frequencyforgeneral}
\end{equation}
\line
在上面的讨论基础上我们考虑下面的红移问题。对于一个静止在某一个球对称metric $ r_a, r_B $位置的观察者来说。
\begin{align}
  ds^2=g_{00}(r)dt^2+g_{rr}(r)dr^2+r^2d\Omega^2\quad,
\end{align}
如果$ r_A $处的物体发射了一个光波给$ r_B $处的物体,那么我们分别考虑这两个观察者观察到的频率不同。
我们假设光波在这个参考系下面的波矢是$ k^a(\lambda) $并且运动轨迹可以用$ x^a(\lambda) $进行描述满足$ k^a = d/d \lambda \ x^a $。
为了计算频率我们计算$ \omega = -u^ak_a $:
\begin{itemize}
  \item 对于两个观察者来说,他们的4速度是:
    \begin{align}
      u_A^a = \left(\frac{1}{\sqrt{-g_{00}(r_A)}},0,0,0\right),\quad u_B^a = \left(\frac{1}{\sqrt{-g_{00}(r_B)}},0,0,0\right)\quad.
    \end{align}
  \item 对于两个观察者来说,他们测量到的频率是:
    \begin{align}
      \omega_A = -u_A^a k_a = \frac{k_0}{\sqrt{-g_{00}(r_A)}},\quad \omega_B = -u_B^a k_a = \frac{k_0}{\sqrt{-g_{00}(r_B)}}\quad.
    \end{align}
\end{itemize}
下面我们需要考虑$ k_0 $是什么。我们进行一个类光粒子的lagrangian处理,我们进行一个形式化的变分研究!我们使用之前讨论的参数化,考虑Lagrangian $ \mathcal{L}=\frac{1}{2}g_{\alpha\beta}x^{\prime\alpha}x^{\prime\beta} $然后进行变分。我们发现这个Lagrangian不依赖于$ t $,所以根据Noether theorem,我们知道:
\begin{align}
  Q_t = \frac{\partial \mathcal{L}}{\partial t'} = g_{00}t' = k_0 = \text{const}\quad.
\end{align}
也就是说,$ k_0 $是一个常数。

于是我们有:
\begin{align}
  \frac{\omega_B}{\omega_A} = \sqrt{\frac{g_{00}(r_A)}{g_{00}(r_B)}}\quad.
\end{align}
这个就是Gravitational Redshift的结果。

\tip{关于$ k_0 $的interpretation}{
  我们会发现这个量其实可以理解为我们选定的参考系下面观察到的光波频率。【但注意,这个参考系并不一定属于某个观察者捏!!】
}

\bigskip
\hlr{第三种推导:一般时空的两个static observer之间的redshift推导!}

首先,我们需要定义一个一般时空下更广义的“static observer”的概念。狭义上我们感觉这样的观察者在空间位置并不变换而时间位置发生变化。好像就是在且仅仅在一个「很像时间的方向」进行运动。下面我们的难题是,怎么定义「很像时间的方向」?

\imp{怎么定义「很像时间的方向」}{
  我们可以使用Killing vector field来定义「很像时间的方向」。我们说一个Killing vector field是timelike的,如果在某一个区域之中这个Killing vector field的模长是负的。也就是$ g_{\mu\nu}V^\mu V^\nu < 0 $。那么我们就说这个区域之中存在一个timelike Killing vector field。
}
所以下面我们定义static observer为:
\defi{
  static observer

  如果一个观察者的4速度和某一个timelike Killing vector field平行,那么我们说这个观察者是static observer。
}

下面我们推导这样的观察者是观测到的光是什么样子的。我们知道光子是必须满足测地线方程的,同时如果一个向量是Killing vector,根据之前的讨论意味着这个向量方向进行变分对于粒子运动的Lagrangian是一个Symmetry。所以,根据Noether theorem,我们知道沿着Killing vector方向的光子动量是守恒的:
\begin{align}
  Q_V = k_\mu V^\mu = \text{const}\quad.
\end{align}
接下来我们计算static observer测量到的频率,对于一个static observer来说:
\begin{align}
  u^\mu = \frac{V^\mu}{\sqrt{-V^\nu V_\nu}}\quad.
\end{align}
所以我们有:
\begin{align}
  \omega = -u^\mu k_\mu = -\frac{1}{\sqrt{-V^\nu V_\nu}}k_\mu V^\mu = -\frac{Q_V}{\sqrt{-V^\nu V_\nu}}\quad.
\end{align}
所以我们有结论,两个static observer之间的redshift是:
\begin{align}
  \frac{\omega_B}{\omega_A} = \sqrt{\frac{V^\nu V_\nu|_A}{V^\nu V_\nu|_B}}\quad.  
\end{align}

\subsection{寻找Local Inertial Frame}
\bigskip
\hlr{等价原理以及Local inertial frame}

EEP告诉我们对于任意有引力的时空,我们可以找到一个坐标系,保证这个坐标系locally是类似于Minkowski的:
\begin{itemize}
  \item $ g_{ab}(p) = \eta_{ab} $
  \item 同时$ \Gamma^a{}_{bc}  = 0$或者等价的说$ \partial_\mu g_{\alpha\beta}(p) = 0 $
\end{itemize}

下面有一些comment:
\begin{enumerate}
  \item 显然对于free fall observer来说其geodesic上面每一个点附近,可以找到一个这样子的坐标系【fermi normal coordinate】
  \item 我们仅仅能对于0,1阶导数进行限制,二阶导数是不能限制的。我们可以理解二阶导数是曲率的体现,曲率是张量不能随便消除的。
  \item EEP只能适用于local的,因为我们需要足够小来保证曲率不会起作用「tidal force」
\end{enumerate}
下面给出三个寻中这样的coordinate的方法以及comment

\bigskip
\hlr{强行展开坐标变换并寻找}

我们给出如果强行把一个坐标变换展开,local lorentz frame能给出多少的信息。我们的结论是能够固定前两阶的展开系数,后面有很大的任意性!
\begin{align}
  \xi^a(x)=\xi_0^a+(x-x_0)^\alpha e_\alpha^a+\frac{1}{2}(x-x_0)^\beta(x-x_0)^\gamma e_\alpha^a\Gamma_{\beta\gamma}^\alpha(x_0)+\ldots
\end{align}

\bigskip
\hlr{Riemann Normal Coordinate}

下面给出一个canonical的构造方法,保证满足local inertial frame的要求。这个方法是使用geodesic进行构造的。

我们对于一个geodesic的解进行taylor expension:
\begin{align}
  x^\alpha(\tau)=x_0^\alpha+\tau u_0^\alpha-\frac{1}{2}\tau^2\Gamma_{\beta\gamma}^\alpha(x_0)u_0^\beta u_0^\gamma+\ldots.
\end{align}
然后我们进行一个变量替换:
\begin{align}
  u_0^\alpha=\lambda^ae_a^\alpha\quad,\quad g_{\alpha\beta}(x_0)e_a^\alpha e_b^\beta=\eta_{ab}\quad.
\end{align}
通过这两部我们可以构造一个坐标变换,正好满足local inertial frame的要求。并且构造方法已经在选择一个geodesic的时候固定了所有的信息:
\begin{align}
  x^\alpha(\xi)=x_0^\alpha+(\xi-\xi_0)^ae_a^\alpha-\frac{1}{2}(\xi-\xi_0)^b(\xi-\xi_0)^c\Gamma_{\beta\gamma}^\alpha(x_0)e_b^\beta e_c^\gamma+\ldots
\end{align}


\bigskip
\hlr{数一下自由度}

我们可以通过数一下自由度理解为什么metric的0,1阶导数可以被固定但是第二阶就不行。

我们研究不同阶导数在坐标变换下面的行为。发现二阶导数如果要限制,写出来的方程数目大于变换矩阵的自由度数目。所以我们知道,metric的二阶导数在坐标变换下面是不能被固定的。




\subsection{Questions and discussions}


\question{怎么理解我们的$ \mathcal{L} = \displaystyle\frac{1}{2} g_{\mu\nu}\dot{x}^\mu \dot{x}^\nu $,这到是几何概念还是物理??}

这个lagrangian其实是为了得到Geodesic quation计算方便给出的工具。

几何上理解也就是这个Lagrangian对应的变分原理是某两点在一个流形上面的“最短距离”的曲线方程。我用引号是因为最短距离本身不是一个严格定义的东西,我们不妨就是把这个lagrangian的积分变分为0定义成为最短距离。


但是物理上理解是,对于选Alffine parameter作为参数化的粒子坐标。这个给出了任意metric下面的自由粒子的运动方程。虽然有时候我们会觉得这个lagrangian没有任何物理量。但是我们要记住,这可以是一个粒子的动力学等效lagrangian!!

所以,这样子我们发现我们统一了物理上的自由粒子运动曲线和几何上的“最短距离”的概念。所以可以信誓旦旦的说,自由离子在引力场中按照最短距离进行运动。
\qed 

\bigskip
\question{在对于$ L = g_{\mu\nu} \dot{x}^\mu \dot{x}^\nu $进行变分的过程之中,我们为什么认为$ g_{\mu\nu} $是不变的??}

本文之中,讨论其实不是一个约束系统的坐标变换。而是一个循环坐标我们的运动方程能够推出一个守恒量。呃呃呃呃呃...Blau哥成功避开了这个问题。

\question{对于标量力学的说明捏!!}

我们一般说分析力学是标量力学。因为我们说分析力学的动力学量的generator是标量。但是实际上!!!在相对论意义下,只有Lagrangian是标量,Hamiltonian不是标量!!!!每当我们说一个Hamiltonian,或者一个能量的时候,我们说的都是某一个参考系下等时间面的Hamiltonian!!
\qed 

\tip{所谓“时间坐标”}{
  我们在GR之中所有坐标都是等价的。没有说有所谓“时间坐标”。但是!!如果我们选择一个坐标系,然后发现其中一个坐标的行为和我们在狭义相对论之中理解的“时间”十分类似,那么我们可以说这个坐标表征了一个坐标系下看某一个粒子运动的时间。

  但是,问题是,我们选择的一般的坐标系并不一定拥有这么好的一个坐标方向捏!!
}


\bigskip
\question{
我们经常说的时间平移变换的generator 或者其他平移变换各种变换的generator是什么??$ \partial_t $到底是什么意思,是个什么数学object,对物理有什么用???
}


\bigskip
\question{什么叫 metric is addapted to the hyperboloid???什么叫度规正好适应了一种几何??}

我们有的时候会用这样的语言指代,比如我们对Minkowski 时空进行下面的变换:
\begin{equation}
  (t,r)=(\rho\sinh\tau,\rho\cosh\tau)
  \label{eq:trans}
\end{equation}
并不是所有$ (t,r) $都可以这样变换的,因为这样变换有一个自然约束条件就是$ r^2 - t^2 = \rho $,所以我们基本上使用$ \rho $“刻画”了不同的双曲面,对应着是不同的加速度的从负无穷进行加速到正无穷的物体。

这种情况下,我们认为这个metric自然适应了双曲面这样的几何。


\imp{怎么理解某一个坐标系下固有时间}{
  我们怎么理解下面这个式子:
  \begin{align}
    d\tau^2=-g_{\alpha\beta}(x)dx^\alpha dx^\beta 
  \end{align}
  我们可以说某一个参考系下面我们计算不同位置的粒子的固有时的方法和位置相关。

  \bigskip
  要记住:狭义相对论里面我们每一个惯性参考系都可以理解为一个观察者通过对钟的方法观测世界使用的结果的参考系。但是GR之中,我们的参考系更加任意,并不一定对应任何观察者的随动。所以,我们不能对一个参考系引入一个像是狭义相对论的“观测”定义,我们只能说使用一个参考系来描述粒子运动,这个粒子看自己的时间流逝可以通过上面方法计算出来!!
}


\tip{相对论中的测量结果}{
  在一般的物理之中,我们经常不经意的使用一个参考系得到的数字来预测另一个观察者的测量结果。这是显然不对的,能得出正确的结论是因为参考系变换是trivial的,所以我们不小心就对了。
\bigskip

但是,GR之中,如果我们希望预测一个观察者的测量结果,务必使用这个观察者的随动参考系!!!对于一切某个参考系下面得到的结果,我们务必变换到随动参考系之中才能得到正确的数值!!
}


\bigskip
\question{
  我们考虑Lagrangian $ L= \displaystyle\frac{1}{2} g_{\mu\nu}x'^\mu x'^\nu $ 是怎么描述类光粒子的运动的??更准确的说,我们应该使用哪个参数化??「显然似乎使用Alffine parameter看上去是没有意义的,因为这是一个常数参数化 $ \tau = 0 $」
}

\hlr{使用 L 研究类时粒子的运动}

我们选择一个Alfine parameter「一般都是使用$ \tau $」作为参数化,这个Lagrangian才能够正确描述类时粒子的运动。然后进行研究就好!!

\bigskip
\hlr{使用 L 研究类光粒子的运动}

对于类光粒子来说,我们也可以等效的使用这个lagangian进行描述。第一个问题是,参数化应该选什么?我们会发现,类光粒子没有Alfine parameter的定义,因为我们之前定义为$ \lambda = a \tau +b $的参数化,但是没有质量粒子来说固有时永远是0。

但是神奇的是,我们会发现类光粒子来说:$ S_1 = \int L d \lambda  $永远是一个0!说明这个action对于任意轨迹都是stationary的,也就是【测地线方程「原型」】【对于所有参数化,不仅仅是alfine parameter】【对于无质量粒子】都成立!!

但是对于类光粒子我们有一个特别好的参数化。在这个参考系之中我们粒子的能量是$ k^a $的话【对于有物理意义的参考系这个是可以具体测量得到的】,我们使用$ k^a = \displaystyle\frac{d}{d \lambda} x^a $成立的参数化!「前面的例子之中我们就是这么描述的!」

\imp{光波描述性质}{
我们研究相对论语境下面光的行为我们可以知道。【选定一个坐标系】:
\begin{enumerate}
  \item 然后光的运动可以通过一个Null Vector进行描述$ k^a = (\omega,k) $,满足$ k^a k_a= 0 $
  \item 量子力学告诉我们:这个null Vector正比于光的4-动量:$ p^a = \hbar k^a $。
  \item 因为波矢量指向光子运动方向,我们一般使用下面的参数化$ k^a = \displaystyle\frac{d}{d \lambda} x^a(\lambda)$这个$ x^a(\lambda) $描述了光子的运动。
\end{enumerate}
}
\imp{研究无质量粒子Lagrangian}{
  分为下面几个步骤:
  \begin{enumerate}
    \item 选择一下$ k^a = \displaystyle\frac{d}{d \lambda} x^a $成立的参数化!
    \item 形式化的使用各种原理就好!!我们只需要最后引入无质量的条件,否则一切都trivial了!!
  \end{enumerate}
}
\qed 
\bigskip

\question{我们说的local inertial frame不会就是粒子随动参考系吧?}

显然不是!!!!请不要混淆,local inertial frame是某一个点的一个等效locally没有引力的参考系!!这个参考系并不一定是某一个粒子的随动参考系!!
\qed 





\newpage
\section{Tensor Algebra}\label{sec:Tensor Algebra} % (fold)

\subsection{General Covariance原理}

\hlr{怎么推出Geodesic equation对于一切参考系成立}

我们已经知道了物理规律需要满足EEP。并且我们之前已经通过EEP给出了一般粒子甚至光在一般有引力是空的运动。现在我们希望一个更加一般的原理。

回顾Geodesic的讨论,我们知道:我们论证这个方程在SR成立;并且论证这个方程是general covariance的。所以我们有理由相信这个方程在GR的情况下也是成立的。我们下面给出证明:
\begin{itemize}
  \item 首先,我们对于一般情况写出方程;我们总可以变换称为一个local inertial coordinate下面的样子。
  \item 我们根据EEP知道,local inertial coordinate下面的方程和SR的方程是一样的。所以这个方程成立
  \item 根据General Covriance我们又知道这个方程对于所有坐标系都成立的。
\end{itemize}

我们给出结论\textbf{EEP保证了,我们可以数学上知道一个方程对于inertial coordinate成立,并且这个方程是general covariant的,那么这个方程对于所有coordinate都成立}。

\bigskip
\hlr{Principle of General Covariance}

上方的讨论我们知道应该对于一切物理方程都是合法的。我们把结论总结为:【这是一个推广出GR的方程的办法】
\imp{Principle of General Covariance}{
  一个物理方程如果在有引力的情况下成立需要满足下面的条件:
  \begin{enumerate}
    \item 方程在没有引力的情况下成立$ g = 0 $并且$ \Gamma = 0 $
    \item 方程是general covariant的,也就是方程形式在所有的坐标系下面不变。
  \end{enumerate}
}

对于第二个条件更正确的表述是,方程在一个坐标系下成立则在其他坐标系下都成立「方程形式可能会乘上一个可逆的矩阵」;

对于第一个要小心:我们其实local inertial frame并不是没有引力的情况,所以没有引力情况下的并不一定是local inertial frame下面的。但是很多情况下两者是很像的,所以我们这里的讨论就忽略这个区别。但是有的时候并不可以忽略的。【我们强调这一点,是因为我们上面的Principle of General Covariance利用EEP进行证明的时候一定需要从0 gravity过度到LLF,但是两者其实是有区别的。】

\subsection{Tensors:实现General Covariance的工具}
\bigskip
\hlr{Tensor Algebra}

为了满足Principle of General Covariance,我们需要使用tensor algebra来描述物理量。我们知道tensor是一个在坐标变换下面有良好变换性质的数学对象,他们自然满足第二个条件也就是general covariant。

\hlr{本章其余部分我们跳过阅读;建议阅读参考lcb学习其中tensor的内容}


\subsection{Vierbein以及orthonormal frame}




% section Tensor Algebra (end)



\newpage
\section{Tensor Analysis (Generally Covariant Differentiation)}\label{sec:Tensor Analysis (Generally Covariant Differentiation)} % (fold)

\hlr{跳过Covariant Derivative的一些内容。很多书籍会讲得更加清晰捏}

\subsection{Covariant Derivative (Levi-Civita Connection)的神奇性质}

% section Tensor Analysis (Generally Covariant Differentiation) (end)






\chapter{sketch book}
这个章节的主要目的是,如果fol一个知识点,不小心fol到了shit。但是又不忍心删除。不妨放在这里吃灰(((

\end{document}
